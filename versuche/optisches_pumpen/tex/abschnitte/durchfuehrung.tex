
% ==================================================
%	Durchführung
% ==================================================

\section{Durchführung}
\label{sub:durchfuehrung}
	
	Zunächst wird die Horizontalfeldspule (H) komplett herunter 
		gefahren, die Sweep-Spule (S) verfährt den maximal möglichen 
		Bereich in Perioden von $\SI{2}{\second}$ Dauer. Bei Magnetfeld 
		$0$ 
		entarten die $m_J$ Niveaus und 
		die Transmittivität sinkt stark. Dazu werden Vertikalfeldspule (V) 
		und die optische Achse des Experiments so eingestellt, dass 
		der auf dem Oszilloskop zu sehende Intensitäts-Dip möglichst 
		schmal wird. $\ddot \smile$
	
	
	Nun wird in der Bereich $\SI{100-1000}{\kilo\hertz}$ in 
		$\SI{100}{\kilo\hertz}$ Schritten vermessen. Dazu wird zunächst 
		bei jedem Schritt $|\vec{B}_\text{H}|$, das Magnetfeld von H, so
		 eingestellt, dass ein oder zwei Dips 
		auf dem Oszilloskop zu sehen sind. Diese entsprechen den 
		Resonanzmagnetfeldern gemäß \eqref{eq:B}. 
		Nun wird $B_\text{S}$ angehalten und manuell variiert, bis das 
		Resonanzmagnetfeld 
		erreicht ist. Die Potentiometerwerte für $|\vec{B}_\text{S}|$ und 
		$|\vec{B}_\text{H}|$ werden aufgenommen. Dabei ist 
		$\vec{B}_\text{S}$ das Magnetfeld von S.
	
	 Zuletzt wird ein typisches Signalbild mit dem 
		Digitalen Oszilloskop gespeichert.
	
