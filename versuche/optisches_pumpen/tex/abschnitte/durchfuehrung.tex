
% ==================================================
%	Durchführung
% ==================================================

\section{Durchführung}
	\begin{enumerate}
	\item Zunächst wird die Horizontalfeldspule (H) komplett herunter 
		gefahren, die Sweep-Spule (S) verfährt den maximal möglichen 
		Bereich in Perioden von $\SI{2}{\second}$ Dauer. Bei Magnetfeld 
		$0$ 
		entarten die $m_J$ Niveaus und 
		die Transmittivität sinkt stark. Dazu werden Vertikalfeldspule (V) 
		und die Raumausrichtung des Experiments so eingestellt, dass 
		der auf dem Oszilloskop zu sehende Dip möglichst scharf wird. 
	\item Nun wird in der Bereich $\SI{100-1000}{\kilo\hertz}$ in 
		$\SI{100}{\kilo\hertz}$ Schritten vermessen. Dazu wird zunächst 
		bei jedem Schritt H so eingestellt, dass ein oder zwei Dips 
		auf dem Oszilloskop zu sehen sind. Diese entsprechen den 
		Magnetfeldern für eine Zeeman-Aufspaltung gemäß \eqref{eq:B}. 
		Nun wird S angehalten und manuell variiert, bis das Magnetfeld 
		\eqref{eq:B} erreicht ist. Die Potentiometerwerte für S und H 
		werden aufgenommen.
	\item Ein typisches Signalbild aus Punkt 2 wird mit dem 
		Digitalen Oszilloskop gespeichert.
	\end{enumerate}