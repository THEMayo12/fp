
% ==================================================
%	Theorie
% ==================================================

\section{Theorie}
	In der Quantentheorie können die Elektronen einer Atomhülle eines 
	Atoms nur abzählbar viele Zustände annehmen, die durch die 
	Quantenzahlen 
	\begin{itemize}
	\item Elektronenspin $S$, für Elektronen ist $S=\nicefrac{1}{2}$
	\item Kernspin $I$, für $\Rb95$ ist $I=\nicefrac{5}{2}$
	\item Bahndrehimpuls $L\geq 0$
	\item Drehimpuls der Elektronenhülle $J\in\{ |S-L|,|S-L|+1,
	\ldots , S+L-1, S+L \}$
	\item Geesamtdrehimpuls $F\in \{|J-I|,|J-I|+1,\ldots , J+I-1, J+I\}$
	\end{itemize}
	charakterisiert werden. 
	Zusätzlich werden die Quantenzahlen $m_S$, $m_L \in 
	(-S\text{ bzw. }-L,
	S\text{ bzw. }L) \cap \mathbb{Z}$ und $m_F\in (-F,F)\cap 
	\mathbb{Z}$ eingeführt, sodass 
	ein eindeutiger  
	Elektronenzustand $|s,l,j\rangle$ definiert werden kann. Im 
	allgemeinen können die Zustände jedoch entartet sein, d.h. 
	verschiedenen Zuständen haben die gleiche Energie\footnote{Im 
	Folgenden ist ein Energieniveau die Menge der $g$ Zustände selber 
	Energie. Wird verschiedenen Zuständen die gleiche Energie 
	zugeordnet, so spricht man von \text{Entartung}.}.
	Aufgrund des Pauli-Prinzips kann 
	jeder Elektronenzustand nur einfach besetzt werden, sodass 
	Zustände niedriger Energie "`aufgefüllt"' sind, da hier  
	thermische Prozesse keine Rolle spielen. Für 
	zwei nicht aufgefüllte Energieniveaus $1$ und $2$ mit Energien 
	$W_1$, $W_2$ 
	und 	Besetzungszahlen $N_1$, $N_2$ gilt dagegen die Boltzmannsche 
	Gleichung \cite{Praktrikum}
	\begin{equation}
		\frac{N_2}{N_1} = 
		\frac{g_2}{g_1}\frac{e^\nicefrac{-W_2}{k_\text{B} T}}{
		e^\nicefrac{-W_1}{k_\text{B} T}}
		 \label{eq:Boltzmann} .
	\end{equation}	 
	Die magnetischen Momente werden mit dem Bohrschen Magneton 
	$\upmu_\text{B}$ über 
	\begin{align}
		|\vec{\mu}_S| & = g_S \upmu_\text{B} \sqrt{ S(S +1)} \\
		|\vec{\mu}_L| & = \upmu_\text{B} \sqrt{ L(L +1)} \\
		|\vec{\mu}_J| & =g_J \upmu_\text{B} \sqrt{ J(J +1)} 
	\end{align}
	definiert, wobei $g_S$ und $g_J$ die Lande-Faktoren des Spins 
	bzw. des Drehimpulses der Elektronenhülle sind. 
	
	Im Experiment äußern sich die Zustände des Atoms in 
	Elektromagnetischer Strahlung, die beim Übergang eines Elektrons 
	in ein anderes Energieniveau emittiert wird. Sind $W_\text{i}$ und 
	$W_\text{f}$  die Energien des Anfangs- und Endniveaus, so wird 
	mit dem Plankschen Wirkungsquantum $\text{h}$ durch 
	\begin{equation}
		\text{h} \nu = W_\text{f}- W_\text{i} \label{eq:hnu}
	\end{equation}
	die Frequenz $\nu$ der emittierten Strahlung gegeben.
	
	Folgende Effekte beeinflussen die Energie eines Zustandes und 
	führen so Teilweise zur Aufhebung der Entartung.
	\begin{itemize}
		\item Die LS-Kopplung beschreibt die Wechselwirkung zwischen 
			Spin $S$ und Bahndrehimpuls $L$ der Elektronenhülle. Dadurch 
			wird 
			die Entartung in $L$ aufgehoben. Dies liefert einen 
			Beitrag zur \textit{Feinstruktur} des Spektrums.
		\item Die Wechselwirkung zwischen dem Drehimpuls der 
			Elektronenhülle $J$ und dem Kernspin $I$, durch die  
			die Entartung in $F$ aufgehoben wird, wird als 
			\textit{Hyperfeinstruktur} bezeichnet.
		\item Im äußeren Magnetfeld $\vec{B}$ verschiebt sich die 
			Energie eines Zustandes um 
			\begin{equation}
				U = m_F g_F \upmu_\text{B} |\vec{B}| \quad ,
			\end{equation}
			diesen Effekt bezeichnet man als \textit{Zeeman-Effekt}
			 \cite{Praktikum}; er hebt die Entartung in 
			$m_f$ auf.
	\end{itemize}
	Für den dabei eingeführten Lande-Faktor gilt dabei
	\begin{equation}
		g_F \approx g_J \frac{F(F+1)+J(J+1)-I(I+1)}{2F(F+1)} \quad .
		\label{eq:g_F}
	\end{equation}
	
	Nun soll das Prinzip des Optischen Pumpens an einem hypothetischen 
	Alkali-Atom, d.h. mit einem Valenzelektron, und ohne Kernspin 
	erläutert werden\footnote{Für $I=0$ ist $F=J$, sodass ab 
	jetzt $J$ und $m_J$ die relevanten Quantenzahlen sind.}. Wie in 
	Abbildung \ref{fig:Termschema} zu sehen ist, gibt es 
	\begin{figure}
		\centering
		\includegraphics[scale=0.5]{bilder/Termschema}
		\caption{Beispielhaftes Termschema eines Alkali-Atoms im 
		äußeren Magnetfeld $|\vec{B}|$. \cite{Praktrikum}}
		\label{fig:Termschema}
	\end{figure}
	zwischen dem $S_{\nicefrac12}$ (d.h. $L=0,J=\nicefrac12$) und 
	$P_{\nicefrac12}$ (d.h. $L=1,J=\nicefrac12$) Energieniveau vier 
	mögliche Übergänge. Durch Anregung mit rechtszirkular polarisierter 
	Strahlung kann nun gezielt der $\sigma^-$ Übergang vom 
	$S_{\nicefrac12}$ in 
	ein $P_{\nicefrac12}$ hervorgerufen 
	werden. Durch spontane Emission fällt das Elektron wieder auf das 
	$S_{\nicefrac12}$ Niveau mit $m_J=\nicefrac12$ oder 
	$m_J=-\nicefrac12$ 
	zurück. Durch Wiederholung dieses Vorgangs in einem viele-Atome 
	System wird dadurch der Zustand $S_{\nicefrac12},m_J=\nicefrac12$ 
	angereichert. Effektiv ist die Transmittivität des Systems für 
	diese Art der Strahlung zunächst gering, steigt jedoch, sobald 
	der $S_{\nicefrac12},m_J=-\nicefrac12$ wenig bis nicht mehr besetzt 
	ist. 
	
	Durch das anlegen eines Hochfrequenzfeldes der Frequenz $\nu$ 
	kommt es, falls ein Elektron die Energie 
	$\text{h} }nu$ abgeben kann, zu induzierter Emission. Bei 
	festgehaltener Frequenz $\nu$ kann nun das Magnetfeld so 
	eingestellt werden, dass die Zeeman-Aufspaltung zwischen den 
	$m_J=\pm \nicefrac12$ Niveaus genau 
	\begin{equation}
		\text{h}\nu = g_J \upmu_\text{B} |\vec{B}| \label{eq:B} 
	\end{equation}
	entspricht. Dadurch wird das 
	$m_J=-\nicefrac12$ Niveau wieder gefüllt und die Transmittivität 
	nimmt ab.
	
	In höherer Ordnung und mit Kernspin ergibt sich außerdem 
	die Breit-Rabi Formel
	\begin{equation}
		U=g_F \uppmu_\text{B} |\vec{B}|+g_F^2 \upmu_\text{B}^2 
		|\vec{B}|^2 \frac{1-2 m_F}{\Delta E_\text{Hy}} 
		\label{eq:Breit-Rabi}
	\end{equation}
	wobei $\Delta E_\text{Hy}$ die Hyperfeinstrukturaufspaltung 
	zwischen den Energieniveaus $F$ und $F+1$ darstellt.
	