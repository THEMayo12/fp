
% ==================================================
%	Einleitung
% ==================================================

\section{Einleitung}
	In diesem Versuch wird das Verfahren des \textit{optischen Pumpens}
	verwandt um die Zeemanaufspaltung der Energieniveaus von \ce{^85Rb} und
	\ce{^87Rb} zu vermessen. Das optische Pumpen ist dabei eine Methode,
	bei angelegtem äußeren Magnetfeld
	eine Abweichung von den natürlichen Besetzungszuständen der
	Elektronenhüllen zu erzeugen, sodass mit induzierter Emission
	die Energiedifferenz zwischen den Zeemanniveaus nachgewiesen und
	vermessen werden kann.

	Ziel des Versuches ist es, das lokale
	Magnetfeld der Erde in vertikaler und horizontaler Nord-Süd
	Richtung zu bestimmen, die Lande-Faktoren für \ce{^85Rb} und
	\ce{^87Rb} zu bestimmen sowie die Kernspins dieser beiden Isotope.
	Außerdem wird die Isotopenzusammensetzung des
	verwandten Rubidium-Gasgemisches ermittelt. Zuletzt wird eine
	Abschätzung des quadratischen Zeeman-Effektes für diesen
	Versuchsaufbau gegenen.
