
% ==================================================
%	Diskussion
% ==================================================

% \clearpage
\section{Diskussion}

Die hier bestimmten Vertikal- und Horizontalkomponenten des Erdmagnetfeldes
können mit~\cite{MAG} verglichen werden, worin
\begin{equation}
  B_\text{vertikal} \approx \SI{44}{\micro\tesla}, \qquad
  B_\text{horizontal} \approx \SI{20}{\micro\tesla}
\end{equation}
angegeben werden.
Dabei stimmt die ermittelte Horizontalkomponente mit dem Literaturwert überein,
wobei die Vertikalkomponente mit
${B_\text{vertikal} = \SI{36.2}{\micro\tesla}}$ abweicht, was
die folgenden Messungen negativ beeinflusst hat.
So sind die Kernspins laut Theorie alle halbzahlig, wobei jedoch
\begin{equation}
  I_1 = \SI[parse-numbers = false]{1.760 \pm 0.021}{} \approx \frac{3}{2}, \qquad
  I_2 = \SI[parse-numbers = false]{2.757 \pm 0.035}{} \approx \frac{5}{2}
\end{equation}
gemessen wurde.
Jedoch lassen sich mit diesen geschätzten Spins und der Nuklidkarte von
\cite{NUK} die entsprechenden Isotope entsprechend
\begin{equation}
  I_1 \approx \frac{3}{2} \rightarrow \ce{^87Rb}, \qquad
  I_2 \approx \frac{5}{2} \rightarrow \ce{^85Rb}
\end{equation}
zuordnen.
Das natürliche Isotopenverhältnis kann aus \cite{NUK} ebenfalls zu
\begin{equation}
  \frac{A_1}{A_2} = \frac{\SI{72.165}{\percent}}{\SI{27.835}{\percent}} = 2.59
\end{equation}
bestimmt werden, was wiederum deutlich von dem hier bestimmten Wert von
$A_1 / A_2 = 2.0$ abweicht.
