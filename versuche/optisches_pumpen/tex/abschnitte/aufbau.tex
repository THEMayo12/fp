
% ==================================================
%	Aufbau
% ==================================================

\section{Aufbau}
	Eine Skizze des Versuchsaufbaus ist in Abbildung \ref{fig:Aufbau} zu 
	\begin{figure}
		\centerin
		\includegraphics[scale=0.5]{bilder/Aufbau}
		\caption{Schematische Abbildung des Versuchsaufbaus. 
		\cite{Praktikum}}
		\label{fig:Aufbau}
	\end{figure}
	sehen. Der Aufbau aus Spektrallampe, Sammellinse und $D_1$ Filter 
	erzeugt Licht der Wellenlänge 
	$\lambda = \SI{794.8}{\nano\meter}$, welches genau dem $D_1$
	 Übergang 
	des Rb-Spektrums ($S_\nicefrac12$ nach $P_\nicefrac12$) entspricht. 
	Durch den linear Polarisator und die anschließende $\lambda/4$ 
	Platte kann zirkular polarisiertes Licht erzeugt werden. Dieses 
	tritt durch das Rubidium-Gasgemisch und fällt durch eine 
	weitere Sammellinse auf einen Photodetektor, mit dem die einfallende 
	Lichtintensität gemessen werden kann. Das Gasgemisch wird dabei auf 
	$\SI{50}{\celsius}$ geheizt, um einen optimalen Gasdruck zu
	erreichen. Das Gasgemisch besteht aus $\Rb95$ und $\Rb97$ Atomen in 
	einem unbekannten Verhältnis, sodass zwei Resonanzfrequenzen 
	mit unterschiedlich stark ausgeprägten Transmittivitätseinbrüchen 
	erwartet werden. 
	Die Apparatur wird so aufgebaut, dass die Strahlrichtung auf der 
	Nord-Süd Achse steht, sodass das Erdmagnetfeld keine dazu senkrechte 
	horizontale Komponente besitzt. 
	
	Die Dampfzelle ist von drei Helmholtzspulen und einer RF-Spule  
	umgeben. Es ist 
	jeweils $R$ der Radius und $N$ die Windungszahl der Spulen.
	\begin{itemize}
		\item Die Vertikalfeldspule gleicht das Erdmagnetfeld in 
			vertikaler Richtung aus. 
			$N_\text{V}=20 ,R_\text{V}=11.735$.
		\item Die Horizontalfeldspule erzeugt ein horizontales 
			Magnetfeld entlang der Strahlachse.
			$N_\text{H}= 154,R_\text{H}=15.79$.
		\item Die Sweep-Spule erzeugt ebenfalls ein horizontales 
			Magnetfeld entlang der Strahlachse. Sie sorgt dafür, dass 
			das Magnetfeld einen Bereich um das durch die 
			Horizontalfeldspule erzeugten Magnetfeldes verfährt.
			$N_\text{S}=11 ,R_\text{S}=16.39$.
		\item Die RF-Spule liegt an der Dampfzelle. An ihr 
			wird die Frequenz $\nu$ für die induzierte Emission 
			eingestellt. Da hier nur die Frequenz entscheidend ist, 
			sind Radius und Windungszahl unerheblich.
	\end{itemize}
	Über ein Steuergerät wird der Spulenstrom von Sweep- und 
	Vertikalfeldspule mit $\SI{0.1}{\ampere}$ pro Umdrehung und bei 
	der Horizontalfeldspule mit $\SI{0.3}{\ampere}$ pro Umdrehung 
	eingestellt. 
	
	Die RF-Spule wird an einen Frequenzgenerator im Bereich 
	$\SI{100-1000}{\kilo\hertz}$ angeschlossen. Auf einem Oszilloskop 
	werden im XY-Modus die Lichtintensität an der Photodiode (Y) gegen 
	die Sweep-Feldstärke (X) dargestellt.