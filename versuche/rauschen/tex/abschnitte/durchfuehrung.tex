
% ==================================================
%	Durchführung
% ==================================================

\section{Durchführung}

\subsection{Einfaches Spektrometer}
\label{sub:einfaches_spektrometer}

\subsubsection{Eigenrauschen des Rauschspektrometers}
\label{ssub:eigenrauschen_des_rauschspektrometers}

Zur Bestimmung des Eigenrauschens des einfachen Rauschspektrometers wird die
Verstärkerschaltung aus Abbildung~\ref{fig:spektrometer} verwandt, wobei der
Widerstand durch einen Kurzschluss ersetzt wird.
Der Bandpass wird dabei so eingestellt, dass nur Frequenzen zwischen
\SI{20}{\kilo\hertz}und \SI{30}{\kilo\hertz} durchgelassen
werden.  Die Ausgangsspannung besteht nun lediglich aus dem Rauschen der
Verstärkerschaltung, das bei allen anderen Messungen den Untergrund darstellt.

\subsubsection{Durchlasskurve des Bandfilters}
\label{ssub:durchlasskurve_des_bandfilters}

Die Durchlasskurve des Bandfilters wird mit dem Aufbau aus
Abbildung~\ref{fig:spektrometer} vermessen.
Hierbei wird anstelle des Widerstands ein Frequenzgenerator angeschlossen.
Der Bandpass wird hierbei aus der vorigen Messung nicht verstellt.
Die Frequenzen werden nun logarithmisch über drei Zehnerpotenzen variiert.
Die Ausgangsspannungen und die Verstärkungen des Nachverstärkers werden notiert.

\subsubsection{Rauschmessung des Widerstands}
\label{ssub:rauschmessung_des_widerstands}

Es wird nun die Ausgangsspannung in Abhängigkeit des Widerstandes gemessen und
die jeweilige Nachverstärkung notiert.
Hierzu werden zwei Potentiometer verwandt mit den Bereichen
\SIrange[range-phrase=--, range-units=single]{0}{1000}{\ohm}
und
\SIrange[range-phrase=--, range-units=single]{1}{10}{\kilo\ohm}.

\subsection{Korrelatorschaltung}
\label{sub:korrelatorschaltung}

\subsubsection{Eigenrauschen des Spektrometers}
\label{ssub:eigenrauschen_des_spektrometers}

Auch bei der Korrelatorschaltung wird zunächst das Eigenrauschen der 
Verstärkerschaltung gemessen.

\subsubsection{Durchlasskurve des Selektivverstärkers}
\label{ssub:durchlasskurve_des_selektivverstärkers}

Zur Bestimmung der Durchlasskurve des Selektivverstärkers wird die Schaltung
aus Abbildung~\ref{fig:korrelator} verwandt, wobei anstelle des Widerstands
wieder ein Frequenzgenerator angeschlossen wird.
Anschließend wird die Ausgangsspannung zu Frequenzen zwischen
\SIrange{3}{60}{\kilo\hertz} vermessen.

\subsubsection{Rauschmessung des Widerstands mit der Korrelatorschaltung}
\label{ssub:rauschmessung_des_widerstands_mit_der_korrelatorschaltung}

Die Rauschmessung erfolgt analog zu
Abschnitt~\ref{ssub:rauschmessung_des_widerstands}, wobei jedoch
Schaltung~\ref{fig:korrelator} verwandt wird.

\subsection{Vermessung der Hochvakuumdiode mit Reinmetallkathode}
\label{sub:vermessung_der_hochvakuumdiode_mit_reinmetallkathode}

\subsubsection{Kennlinie der Diode}
\label{ssub:kennlinie_der_diode}

Zur Vermessung der Kennlinie wird die Schaltung aus Abbildung~\ref{fig:schrot}
verwandt. Hierbei wird nun die Heizspannung varriert und anschließend die
Ausgangsspannung in Abhängigkeit der Anodenspannung gemessen.

\subsubsection{Messung zur Bestimmung der Elementarladung}
\label{ssub:messung_zur_bestimmung_der_elementarladung}

Hierbei wird die Anodenstrom auf \SI{130}{\volt} eingestellt und die
Ausgangsspannung in Abhängigkeit des Anodenstroms gemessen.
Für diese Messung wird der Bandpassfilter mit $\Delta\nu = \SI{20}{\kilo\hertz}$
in dem Bereich \SIrange{100}{120}{\kilo\hertz} benutzt. Zudem werden die
Nachverstärkungen notiert.

\subsubsection{Messung des Frequenzspektrums}
\label{ssub:messung_des_frequenzspektrums}

Das Frequenzspektrum wird bei einem konstanten Heizstrom von \SI{0.8}{\ampere}
einer Anodenspannung von \SI{130}{\volt} und einem Anodenstrom von
\SI{1}{\milli\ampere} aufgenommen. Hierbei arbeitet die Diode im
Sättigungsbetrieb.
Der Innenwiderstand der Diode beträgt hier \SI{4680}{\ohm}.
Bei der Messung werden die Frequenzen über vier Zehnerpotenzen
\SIrange{0.010}{460}{\kilo\hertz} variiert. Bei Frequenzen von
\SIrange{100}{460}{\kilo\hertz} wird ein Bandpassfilter verwandt und sonst der
Selektivverstärker. Es werden die Ausgangsspannungen und die Nachverstärkungen
notiert.

\subsection{Vermessung der Hochvakuumdiode mit Oxydkathode}
\label{sub:vermessung_der_hochvakuumdiode_mit_oxydkathode}

Die Messung wird hier analog zu der in
Abschnitt~\ref{ssub:messung_des_frequenzspektrums} durchgeführt.
Der Innenwiderstand beträgt hier \SI{2200}{\ohm}.
