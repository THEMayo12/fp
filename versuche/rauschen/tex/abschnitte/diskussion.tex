
% ==================================================
%	Diskussion
% ==================================================
\clearpage
\section{Diskussion}
In Tabelle \ref{tab:ergebnisse} sind noch einmal die Ergebnisse 
dieser Versuchsdurchführung und die entsprechenden Literaturwerte 
aufgelistet.

\begin{table}[h]
\centering
\begin{tabular}{crc}
\toprule \midrule
Physikalische Größe & Ermittelter Wert & Literaturwert \\
\midrule
$k_\text{B}$ 	& 41.8+/-0.8
$\times 10^{-23}\frac{\text{J}}{\text{K}}$ & $1.380 648 
52(79)\times 10^{-23}\frac{\text{J}}{\text{K}}$\cite{pdg}		\\
"  				& 0.171+/-0.005
$\times 10^{-23}\frac{\text{J}}{\text{K}}$ & "		\\ 
"  				& 1.3+/-0.4
$\times 10^{-23}\frac{\text{J}}{\text{K}}$ & "		\\
"  				& 9.6+/-3.2
$\times 10^{-23}\frac{\text{J}}{\text{K}}$ & "		\\
$\text{e}_0$		& (6.39+/-0.14)e-25
$\times 10^{-19}\text{C}$ & $1.602 176 6208(98)\times 
10^{-19}$C\cite{pdg} \\
$\alpha$			& 0.66+/-0.24 
&$\mathcal{O}(1)$\\
\midrule
\bottomrule
\end{tabular}
\caption{Zusammenstellung der Versuchsergebnisse.}
\label{tab:ergebnisse}
\end{table}


%cite{pdg} http://pdg.lbl.gov/2015/reviews/rpp2015-rev-phys-constants.pdf


Bei den ermittelten Werten für die Boltzmankonstante $k_\text{B}$ 
zeigen sich für die Messungen am einfachen Rauschspektrometer 
(erste zwei Werte) eine deutlich größere Abweichung vom Literaturwert 
im vergleich zu den Messungen am Rauschspektrometer nach 
Korrelationsprinzip. Aus den in der Theorie genannten Gründen 
war dieses Verhalten bereits zu erwarten. Insbesondere der Wert 
$k_\text{B}=1.3+/-0.4$ am korrelierten Spektrometer 
mit kleinem Widerstand $R$ bis $1000\Omega$ enthält den Literaturwert 
in seinem Toleranzbereich. 

Der ermittelte Wert für die Elementarladung $\text{e}_0$ weicht 
geringfügig nach unten ab. Eine mögliche Begründung für diese 
Verschiebung ist, dass während der Messreihe die Kathode wieder 
an eine externe Stromquelle angeschlossen war und nicht allein 
vom Bleiakkumulator betrieben wurde, sodass Störsignale den 
Rauscheffekt beeinflusst haben könnten.

Der Exponent des Funkeleffekts, der als Repräsentant eines $1/f$ 
Rauschens untersucht wurde, liegt wie erwartet in der Größenordnung 
von $1$, sodass hier wirklich von einem $1/\nu^\alpha\approx 1/\nu$ 
Rauschen gesprochen werden kann. Wie ebenfalls erwartet trat der 
Funkeleffekt bei der Oxydkathode deutlich stärker auf als bei 
der Reinmetallkathode (vgl. dazu Abb. \ref{fig:kathode_rein} und 
\ref{fig:kathode_oxyd})  .