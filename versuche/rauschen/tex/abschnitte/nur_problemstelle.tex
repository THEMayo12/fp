\section{Reinmetall Kathode}
	\begin{table}[h]
		\centering
		\begin{tabular}{cccccccc}
		\toprule \midrule
		$\nu$ in kHz	 & $U_a^2$ in $\text{V}^2$	& $\Delta U_a^2$ in $\text{V}^2$ &
		$V_\text{N}$	 & $\Delta \nu$ in kHz 		& $\ln \{\nu\}$				     &
		$W\times 10^{18}$ in $\nicefrac{\text{A}^2}{\text{kHz}}$&
		$\ln \{W\}$	\\
		\midrule
		460.000           & 0.487             & 0.002             & \phantom{0}20\phantom{.} & 21.500            & \phantom{0}6.131  & \phantom{00}0.259 & --42.799         \\
400.000           & 0.519             & 0.030             & \phantom{0}20\phantom{.} & 24.300            & \phantom{0}5.991  & \phantom{00}0.244 & --42.858         \\
360.000           & 0.630             & 0.040             & \phantom{0}20\phantom{.} & 24.400            & \phantom{0}5.886  & \phantom{00}0.295 & --42.668         \\
300.000           & 0.555             & 0.005             & \phantom{0}20\phantom{.} & 23.400            & \phantom{0}5.704  & \phantom{00}0.271 & --42.753         \\
260.000           & 0.460             & 0.004             & \phantom{0}20\phantom{.} & 20.600            & \phantom{0}5.561  & \phantom{00}0.255 & --42.813         \\
220.000           & 0.505             & 0.005             & \phantom{0}20\phantom{.} & 21.000            & \phantom{0}5.394  & \phantom{00}0.274 & --42.739         \\
180.000           & 0.405             & 0.005             & \phantom{0}20\phantom{.} & 16.100            & \phantom{0}5.193  & \phantom{00}0.287 & --42.694         \\
160.000           & 0.379             & 0.005             & \phantom{0}20\phantom{.} & 14.400            & \phantom{0}5.075  & \phantom{00}0.300 & --42.649         \\
120.000           & 0.302             & 0.002             & \phantom{0}20\phantom{.} & 11.600            & \phantom{0}4.787  & \phantom{00}0.297 & --42.660         \\
100.000           & 2.020             & 0.010             & \phantom{0}50\phantom{.} & 12.100            & \phantom{0}4.605  & \phantom{00}0.305 & --42.634         \\
\phantom{0}64.000 & 1.390             & 0.020             & \phantom{0}50\phantom{.} & \phantom{0}8.170  & \phantom{0}4.159  & \phantom{00}0.311 & --42.615         \\
\phantom{0}33.000 & 0.770             & 0.010             & \phantom{0}50\phantom{.} & \phantom{0}4.500  & \phantom{0}3.497  & \phantom{00}0.312 & --42.610         \\
\phantom{0}17.000 & 0.400             & 0.010             & \phantom{0}50\phantom{.} & \phantom{0}2.350  & \phantom{0}2.833  & \phantom{00}0.311 & --42.615         \\
\phantom{0}10.000 & 0.240             & 0.002             & \phantom{0}50\phantom{.} & \phantom{0}1.400  & \phantom{0}2.303  & \phantom{00}0.313 & --42.608         \\
\phantom{00}6.400 & 0.140             & 0.005             & \phantom{0}50\phantom{.} & \phantom{0}0.890  & \phantom{0}1.856  & \phantom{00}0.287 & --42.694         \\
\phantom{00}3.300 & 0.070             & 0.003             & \phantom{0}50\phantom{.} & \phantom{0}0.460  & \phantom{0}1.194  & \phantom{00}0.278 & --42.727         \\
\phantom{00}1.700 & 0.030             & 0.005             & \phantom{0}50\phantom{.} & \phantom{0}0.240  & \phantom{0}0.531  & \phantom{00}0.228 & --42.924         \\
\phantom{00}1.000 & 0.084             & 0.005             & 100\phantom{.}    & \phantom{0}0.140  & \phantom{0}0.000  & \phantom{00}0.274 & --42.741         \\
\phantom{00}0.640 & 0.050             & 0.001             & 100\phantom{.}    & \phantom{0}0.090  & --0.446           & \phantom{00}0.254 & --42.818         \\
\phantom{00}0.330 & 0.130             & 0.010             & 200\phantom{.}    & \phantom{0}0.047  & --1.109           & \phantom{00}0.316 & --42.599         \\
\phantom{00}0.170 & 0.079             & 0.050             & 200\phantom{.}    & \phantom{0}0.025  & --1.772           & \phantom{00}0.361 & --42.466         \\
\phantom{00}0.100 & 0.040             & 0.010             & 200\phantom{.}    & \phantom{0}0.015  & --2.303           & \phantom{00}0.304 & --42.636         \\
\phantom{00}0.064 & 0.080             & 0.010             & 200\phantom{.}    & \phantom{0}0.009  & --2.749           & \phantom{00}0.982 & --41.465         \\
\phantom{00}0.033 & 0.250             & 0.010             & 200\phantom{.}    & \phantom{0}0.005  & --3.411           & \phantom{00}6.203 & --39.621         \\
\phantom{00}0.017 & 0.700             & 0.050             & 200\phantom{.}    & \phantom{0}0.002  & --4.075           & \phantom{0}34.739 & --37.899         \\
\phantom{00}0.010 & 1.800             & 0.100             & 200\phantom{.}    & \phantom{0}0.001  & --4.605           & 171.214           & --36.304         \\
		\midrule
		\bottomrule
		\end{tabular}
		\caption{Messwerte für
		die Reinmetallkathode. Dabei ist $\nu$ die Frequenz und $\Delta \nu$ die
		Durchlassbreite. $U_\text{a}$ ist die gemessene Ausgangsspannung, mit der
		(vernachlässigbar kleinen) Ungenauigkeit $\Delta U_\text{a}$. $V_\text{N}$
		ist der Nachverstärkungsfaktor. $W$ ist das Frequenzspektrum
		$W(\nu)=U_\text{a}(\nu)^2 R^{-2} \nu^{-1}$. Es bedeutet $\{\nu \}$ den
		Zahlenwert von $\nu$ in der Einheit kHz und $\ln\{W\}$ den Zahlenwert
		von $W$ in den Einheiten $\text{V}^2$, $\Omega$ und kHz.}
		\label{tab:kathode_rein}
	\end{table}

	\clearpage
\section{Oxyd Kathode}
	\begin{table}[h]
		\centering
		\begin{tabular}{ccccccccc}
		\toprule \midrule
		$\nu$ in kHz	 & $U_a^2$ in $\text{V}^2$	& $\Delta U_a^2$ in $\text{V}^2$ &
		$V_\text{N}$	 & $\Delta \nu$ in kHz 		& $\ln \{\nu\}$				     &
		$W\times 10^{21}$ in $\nicefrac{\text{A}^2}{\text{kHz}}$&$\ln \{W\}$	  \\
		\midrule
		460.000           & 1.435             & 0.002             & \phantom{0}50\phantom{.} & 34.300            & 13.039            & \phantom{00}0.346 & --33.298          & --32.050         \\
440.000           & 1.697             & 0.001             & \phantom{0}50\phantom{.} & 35.200            & 12.995            & \phantom{00}0.398 & --33.156          & --32.769         \\
400.000           & 1.610             & 0.003             & \phantom{0}50\phantom{.} & 35.800            & 12.899            & \phantom{00}0.372 & --33.226          & --31.687         \\
340.000           & 1.392             & 0.002             & \phantom{0}50\phantom{.} & 33.500            & 12.737            & \phantom{00}0.343 & --33.305          & --32.026         \\
300.000           & 1.303             & 0.003             & \phantom{0}50\phantom{.} & 29.500            & 12.612            & \phantom{00}0.365 & --33.244          & --31.494         \\
260.000           & 1.038             & 0.002             & \phantom{0}50\phantom{.} & 24.500            & 12.468            & \phantom{00}0.350 & --33.286          & --31.713         \\
220.000           & 0.992             & 0.004             & \phantom{0}50\phantom{.} & 23.600            & 12.301            & \phantom{00}0.347 & --33.294          & --30.983         \\
180.000           & 0.810             & 0.002             & \phantom{0}50\phantom{.} & 18.500            & 12.101            & \phantom{00}0.362 & --33.253          & --31.433         \\
160.000           & 0.671             & 0.002             & \phantom{0}50\phantom{.} & 16.300            & 11.983            & \phantom{00}0.340 & --33.314          & --31.306         \\
140.000           & 0.614             & 0.002             & \phantom{0}50\phantom{.} & 14.400            & 11.849            & \phantom{00}0.352 & --33.279          & --31.182         \\
120.000           & 0.545             & 0.002             & \phantom{0}50\phantom{.} & 12.200            & 11.695            & \phantom{00}0.369 & --33.233          & --31.016         \\
100.000           & 2.720             & 0.100             & 100\phantom{.}    & 12.550            & 11.513            & \phantom{00}0.448 & --33.040          & --27.132         \\
\phantom{0}70.000 & 2.030             & 0.100             & 100\phantom{.}    & \phantom{0}9.100  & 11.156            & \phantom{00}0.461 & --33.011          & --26.811         \\
\phantom{0}64.000 & 2.010             & 0.100             & 100\phantom{.}    & \phantom{0}8.400  & 11.067            & \phantom{00}0.494 & --32.941          & --26.731         \\
\phantom{0}33.000 & 1.170             & 0.010             & 100\phantom{.}    & \phantom{0}4.500  & 10.404            & \phantom{00}0.537 & --32.858          & --28.409         \\
\phantom{0}17.000 & 0.650             & 0.010             & 100\phantom{.}    & \phantom{0}2.300  & \phantom{0}9.741  & \phantom{00}0.584 & --32.774          & --27.738         \\
\phantom{0}10.000 & 0.410             & 0.010             & 100\phantom{.}    & \phantom{0}1.400  & \phantom{0}9.210  & \phantom{00}0.605 & --32.739          & --27.242         \\
\phantom{00}6.400 & 1.280             & 0.010             & 200\phantom{.}    & \phantom{0}0.890  & \phantom{0}8.764  & \phantom{00}0.743 & --32.533          & --26.789         \\
\phantom{00}3.300 & 0.800             & 0.010             & 200\phantom{.}    & \phantom{0}0.460  & \phantom{0}8.102  & \phantom{00}0.898 & --32.343          & --26.129         \\
\phantom{00}1.700 & 0.750             & 0.010             & 200\phantom{.}    & \phantom{0}0.240  & \phantom{0}7.438  & \phantom{00}1.614 & --31.757          & --25.478         \\
\phantom{00}1.000 & 0.190             & 0.030             & 200\phantom{.}    & \phantom{0}0.140  & \phantom{0}6.908  & \phantom{00}0.701 & --32.591          & --23.841         \\
\phantom{00}0.640 & 0.365             & 0.002             & 200\phantom{.}    & \phantom{0}0.093  & \phantom{0}6.461  & \phantom{00}2.027 & --31.530          & --26.140         \\
\phantom{00}0.330 & 0.332             & 0.004             & 200\phantom{.}    & \phantom{0}0.047  & \phantom{0}5.799  & \phantom{00}3.656 & --30.940          & --24.762         \\
\phantom{00}0.170 & 0.310             & 0.010             & 200\phantom{.}    & \phantom{0}0.025  & \phantom{0}5.136  & \phantom{00}6.536 & --30.359          & --23.196         \\
\phantom{00}0.100 & 0.290             & 0.010             & 200\phantom{.}    & \phantom{0}0.015  & \phantom{0}4.605  & \phantom{0}10.190 & --29.915          & --22.685         \\
\phantom{00}0.064 & 0.290             & 0.010             & 200\phantom{.}    & \phantom{0}0.009  & \phantom{0}4.159  & \phantom{0}16.107 & --29.457          & --22.228         \\
\phantom{00}0.033 & 0.310             & 0.010             & 200\phantom{.}    & \phantom{0}0.005  & \phantom{0}3.497  & \phantom{0}34.435 & --28.697          & --21.534         \\
\phantom{00}0.017 & 0.320             & 0.020             & 200\phantom{.}    & \phantom{0}0.002  & \phantom{0}2.833  & \phantom{0}73.462 & --27.939          & --20.115         \\
\phantom{00}0.010 & 0.250             & 0.020             & 200\phantom{.}    & \phantom{0}0.001  & \phantom{0}2.303  & 107.610           & --27.558          & --19.487         \\
		\midrule
		\bottomrule
		\end{tabular}
		\caption{Messwerte für
		die Oxydkathode. Dabei ist $\nu$ die Frequenz und $\Delta \nu$ die
		Durchlassbreite. $U_\text{a}$ ist die gemessene Ausgangsspannung, mit der
		(vernachlässigbar kleinen) Ungenauigkeit $\Delta U_\text{a}$. $V_\text{N}$
		ist der Nachverstärkungsfaktor. $W$ ist das Frequenzspektrum
		$W(\nu)=U_\text{a}^2(\nu) R^{-2} \nu^{-1}$. Es bedeutet $\{\nu \}$ den
		Zahlenwert von $\nu$ in der Einheit kHz und $\ln\{W\}$ den Zahlenwert
		von $W$ in den Einheiten $\text{V}^2$, $\Omega$ und kHz.}
		\label{tab:kathode_oxyd}
	\end{table}
