
% ==================================================
%	Auswertung
% ==================================================

\section{Auswertung}

\subsection{Bestimmung von $k_\text{B}$ mittels Spannungsrauschens am 
			einfachen Rauschspektrometer}
		
	
	\subsubsection{Eichmessung}
	
		Die Messwerte zum kurzgeschlossenen Rauschspektrometer sind 
		in Tabelle \ref{tab:eichung_eigenrauschen_einfach} zu sehen. 
		Es wurde das Ausgangssignal $U^2_\text{a}$ in Abhängigkeit von 
		der Nachverstärkung $V_\text{N}$ gemessen.
		
		\begin{table}[h]
		\centering
			\begin{tabular}{cc}
				\toprule \midrule
				$V_\text{N}$ & $U^2_\text{a}$ in $\text{V}^2$
				\\
				\midrule
				\phantom{000}1.000 & --12.400         \\
\phantom{000}2.000 & --12.100         \\
\phantom{000}5.000 & --12.300         \\
\phantom{00}10.000 & --12.300         \\
\phantom{00}20.000 & --12.100         \\
\phantom{00}50.000 & --11.800         \\
\phantom{0}100.000 & --10.500         \\
\phantom{0}200.000 & \phantom{0}--4.600\\
\phantom{0}500.000 & \phantom{0}35.100\\
1000.000          & \phantom{0}88.800\\
				\midrule \bottomrule
			\end{tabular}
			\caption{Messwerte zum kurzgeschlossenen 
			Rauschspektrometer. Dabei ist $U_\text{a}$ die 
			Ausgangsspannung und $V_\text{N}$ die Nachverstärkung.}
			\label{tab:eichung_eigenrauschen_einfach}
		\end{table}
		
		Die Messwerte der Eichmessung sind in Tabelle 
		\ref{tab:eichung_einfach} zu sehen. Während der Messung 
		betrug $V_\text{N}=20$ und die Ausgangsspannung des 
		Frequenzgenerators $U_\text{ein} = 120 \text{ mV}$.
		
		\begin{table}[h]
		\centering
			\begin{tabular}{cc}
				\toprule \midrule
				$\nu \text{ in} \text{ kHz}$ & $U^2_\text{a}$ in $\text{V}^2$ 
				\\
				\midrule
				\phantom{0}1.188  & 0.116            \\
\phantom{0}3.800  & 0.117            \\
\phantom{0}4.169  & 0.117            \\
\phantom{0}5.161  & 0.118            \\
\phantom{0}8.745  & 0.137            \\
\phantom{0}9.261  & 0.140            \\
12.608            & 0.750            \\
14.920            & 1.750            \\
16.747            & 3.310            \\
17.098            & 1.850            \\
18.787            & 5.370            \\
20.192            & 7.240            \\
20.535            & 7.210            \\
21.809            & 8.390            \\
23.009            & 8.870            \\
24.434            & 8.770            \\
26.209            & 8.150            \\
27.809            & 7.230            \\
30.821            & 5.570            \\
32.526            & 3.860            \\
33.041            & 3.320            \\
35.554            & 2.160            \\
37.925            & 1.340            \\
39.882            & 1.020            \\
42.107            & 0.720            \\
45.715            & 0.440            \\
48.860            & 0.310            \\
51.061            & 0.257            \\
52.345            & 0.234            \\
55.561            & 0.190            \\
58.132            & 0.169            \\
60.934            & 0.150            \\
65.829            & 0.141            \\
71.522            & 0.133            \\
75.763            & 0.130            \\
				\midrule \bottomrule
			\end{tabular}
			\caption{Messwerte zur Eichmessung. Dabei ist $U_\text{a}$ 
			die 
			Ausgangsspannung und $\nu$ die Frequenz des 
			Eingangssignals. }
			\label{tab:eichung_einfach}
		\end{table}
		
		Die Durchlasskurve ist in Abbildung \ref{fig:eichung_einfach} zu sehen. Dabei wurde die Größe 
		\begin{equation}
		\beta := \frac{U^2_\text{a, Durchlasskurve}-U_\text{a, Eigenrauschen}^2}
		{V_= V_\text{N}^2 V_\text{V}^2  U^2_\text{ein}}
		\end{equation}
		eingeführt, wobei $V_= =10$, $V_\text{V}=1000$ und $V_\text{N}=20$ (während dieser Messung)
		die Verstärkungsfaktoren des Gleichspannungsverstärkers, des Vorverstärkers beziehungsweise 
		des Nachverstärkers sind.
		
		\begin{figure}
			\centering
			\includegraphics[scale=0.7]{bilder/eichung_einfach.pdf}
			\caption{Durchlasskurve der einfachen Rauschverstärkerschaltung.}
			\label{fig:eichung_einfach}
		\end{figure}
		
		Der Eichfaktor 
		\begin{equation}
			A:=\int\limits_0^\infty V_= V_\text{V}^2(\nu) V_\text{N}^2(\nu) \text{ d}\nu		
		\end{equation}				
		kann dann als
		\begin{equation}
			A = \int \limits_0^\infty \beta(\nu) \text{ d}\nu
		\end{equation}
		über ein Trapezintegrationsverfahren berechnet werden,
		\begin{equation}
			A = \sum\limits_{i=2}^n \frac{\left(\beta(\nu_i)+
					\beta(\nu_{i-1})\right)(\nu_i-\nu_{i-1})}{2}
		\end{equation}
		wobei $\nu_i$ den $i$-ten Messwert darstellt. Es ergibt sich
		\begin{equation}
			A = 18.14\text{ kHz}  \quad . % 18.15+/-0.32 
		\end{equation}


	
	\subsubsection{Rauschmessung}
		Die Messwerte für die Rauschmessung sind in den Tabellen \ref{tab:rauschen_einfach1} 
		und \ref{tab:rauschen_einfach2} sowie den Abbildungen 
		\ref{fig:rauschen_einfach1} und \ref{fig:rauschen_einfach2} zu sehen.
		
		
	\begin{figure}[htbp]

	\begin{minipage}{0.3\textwidth} 

			\centering
			\begin{tabular}{cc}
				\toprule \midrule
				$R$ in $\Omega$ & $U_\text{a}^2$ in $V^2$ \\
				\midrule
				100.000           & 0.057            \\
150.000           & 0.127            \\
200.000           & 0.222            \\
251.000           & 0.336            \\
300.000           & 0.461            \\
350.000           & 0.598            \\
400.000           & 0.758            \\
450.000           & 0.923            \\
500.000           & 1.083            \\
550.000           & 1.263            \\
600.000           & 1.440            \\
650.000           & 1.631            \\
700.000           & 1.810            \\
750.000           & 2.000            \\
800.000           & 2.190            \\
850.000           & 2.390            \\
900.000           & 2.570            \\
950.000           & 2.730            \\			
				\midrule \bottomrule
			\end{tabular}
			\caption{Messwerte zum Spannungsrauschen am einfachen 
			Rauschspektrometer vor der Normierung. $R_\text{max}=1000 \Omega$.}
			\label{tab:rauschen_einfach1}

	\end{minipage}
	% Auffüllen des Zwischenraums
	\hfill
	% minipage mit Grafik
	\begin{minipage}{0.7\textwidth}

			\centering
			\includegraphics[scale=0.7]{bilder/rauschen_einfach1.pdf}
			\caption{Spannungsrauschkurve am einfachen Rauschspektrometer für einen 
			Regelbaren Widerstand bis $1000\Omega$. Die Werte wurden bereits auf eine Verstärkung von 
			$1$ normiert, mit $V_= =10$, $V_\text{N}=200$ und $V_\text{V}=1000$. }
			\label{fig:rauschen_einfach1}
			
	\end{minipage}
	% \caption{noch eine Caption}
	\end{figure}		
	
	
	
	
		\begin{figure}[htbp]

	\begin{minipage}{0.3\textwidth} 

			\centering
			\begin{tabular}{cc}
				\toprule \midrule
				$R$ in $\Omega$ & $U_\text{a}^2$ in $V^2$ \\
				\midrule
				\phantom{0}2.0    & 0.019            \\
\phantom{0}4.3    & 0.059            \\
\phantom{0}6.7    & 0.095            \\
\phantom{0}8.3    & 0.120            \\
\phantom{0}9.0    & 0.145            \\
12.2              & 0.180            \\
14.1              & 0.212            \\
16.4              & 0.242            \\
19.5              & 0.287            \\
22.9              & 0.330            \\
25.5              & 0.373            \\
30.2              & 0.419            \\
34.9              & 0.474            \\
35.8              & 0.480            \\
40.0              & 0.525            \\
45.5              & 0.571            \\
50.6              & 0.602            \\
65.1              & 0.656            \\
62.2              & 0.682            \\
68.0              & 0.698            \\
75.0              & 0.715            \\
78.9              & 0.732            \\
85.4              & 0.727            \\
92.1              & 0.745            \\
97.2              & 0.805            \\			
				\midrule \bottomrule
			\end{tabular}
			\caption{Messwerte zum Spannungsrauschen am einfachen 
			Rauschspektrometer. $R_\text{max}=100 \text{k}\Omega$.}
			\label{tab:rauschen_einfach2}

	\end{minipage}
	% Auffüllen des Zwischenraums
	\hfill
	% minipage mit Grafik
	\begin{minipage}{0.7\textwidth}

			\centering
			\includegraphics[scale=0.7]{bilder/rauschen_einfach2.pdf}
			\caption{Spannungsrauschkurve am einfachen Rauschspektrometer für einen 
			Regelbaren Widerstand bis $100\text{k}\Omega$. Die Werte wurden bereits auf eine 
			Verstärkung von 	$1$ normiert, mit $V_= =10$, $V_\text{N}=200$ und $V_\text{V}=1000$. }
			\label{fig:rauschen_einfach2}
			
	\end{minipage}

	\end{figure}		
		

		Die Ausgleichsgeraden sind durch
		\begin{equation}
			G_1(R) = (5.45 \pm 0.30) \times 10^{-15} \frac{\text{V}^2}{\Omega} R 
					- (xxx)\times 10^{-3}\text{V}^2 		
					% G_2(R) = \SI[parse-numbers = false]{\left(3.72 \pm 0.08\right) \times 10^{-17}}{\volt^2\per\ohm}\, \cdot \,R\, + \SI[parse-numbers = false]{\left(3.024994 \pm 0.000011\right) \times 10^{-9}}{\volt^2}
		\end{equation}
		und
		\begin{equation}
			G_2(R) = (0.4 \pm 3.0) \times 10^{-16} \frac{\text{V}^2}{\Omega} R 
					- (xxx)\times 10^{-3}\text{V}^2 		
					% G_2(R) = \SI[parse-numbers = false]{\left(3.72 \pm 0.08\right) \times 10^{-17}}{\volt^2\per\ohm}\, \cdot \,R\, + \SI[parse-numbers = false]{\left(3.024994 \pm 0.000011\right) \times 10^{-9}}{\volt^2}
		\end{equation}
		
		Mit der Nyquist-Beziehung erhält man
		\begin{equation}
			U_\text{a}^2 =4k_\text{B}T A R \quad , 
		\end{equation}
		sodass mit den Steigungen $m_1$ und $m_2$ der Geradengleichungen $G1(R)$ und $G2(R)$ 
		die Boltzmannkonstante als
		\begin{equation}
		k_\text{B}=\frac{m_i}{4 T A}
		\end{equation}
		berechnet werden kann. Für eine Raumtemperatur von $300$K folgt
		\begin{align}
		k_\text{B}^{1000\Omega}			=  (25\pm 1.4)\times 10^{-23} \frac{\text{J}}{\text{s}} \\
		%input{tabellen/k_einfach1.tex}\\ 
		k_\text{B}^{100\text{k}\Omega}	=	(0.2\pm 1.4)\times 10^{-23} \frac{\text{J}}{\text{s}}  
		%input{tabellen/k_einfach2.tex}
		\end{align}
	
	\subsubsection{Ermittlung der Rauschzahl}
		Für das einfache Rauschspektrometer wird die Rauschzahl 
		\begin{equation}
			F = \frac{U_\text{a}^2}{4 k_\text{B}T R A}
		\end{equation}
		bei $R\approx 500\Omega$ berechnet. Mit den bereits berechneten Größen und 
		dem Wertepaar $(500\Omega,1.083 \text{V}^2)$ ergibt sich 
		\begin{equation}
			F = (1.11 \pm 0.22) \times 10^{3}% 0.665+/-0.005
		\end{equation}
		
		
		
		
		
		
		
\subsection{Bestimmung von $k_\text{B}$ mittels Spannungsrauschens am 
			Korrelator-Rauschspektrometer}
			
		\subsubsection{Eichmessung}
	
		Die Messwerte zum kurzgeschlossenen Rauschspektrometer sind 
		in Tabelle \ref{tab:eichung_eigenrauschen_korr} zu sehen. 
		Es wurde das Ausgangssignal $U^2_\text{a}$ in Abhängigkeit von 
		der Nachverstärkung $V_\text{N}$ gemessen.
		
		\begin{table}[h]
		\centering
			\begin{tabular}{cc}
				\toprule \midrule
				$V_\text{N}$ & $U^2_\text{a}$ in $\text{V}^2$
				\\
				\midrule
				\phantom{000}1\phantom{.} & --0.014          \\
\phantom{000}2\phantom{.} & --0.014          \\
\phantom{000}5\phantom{.} & --0.014          \\
\phantom{00}10\phantom{.} & --0.014          \\
\phantom{00}20\phantom{.} & --0.014          \\
\phantom{00}50\phantom{.} & --0.014          \\
\phantom{0}100\phantom{.} & --0.012          \\
\phantom{0}200\phantom{.} & --0.007          \\
\phantom{0}500\phantom{.} & \phantom{0}0.033 \\
1000\phantom{.}   & \phantom{0}0.174 \\
				\midrule \bottomrule
			\end{tabular}
			\caption{Messwerte zum kurzgeschlossenen  
			Rauschspektrometer nach Korrelationspinzip. Dabei ist $U_\text{a}$ die 
			Ausgangsspannung und $V_\text{N}$ die Nachverstärkung.}
			\label{tab:eichung_eigenrauschen_einfach}
		\end{table}
		
		Die Messwerte der Eichmessung sind in Tabelle 
		\ref{tab:eichung_korr} zu sehen. Während der Messung 
		betrug $V_\text{N}=2$ und die Ausgangsspannung des 
		Frequenzgenerators $U_\text{ein} = 120 \text{ mV}$.
		
		\begin{table}[h]
		\centering
			\begin{tabular}{cc}
				\toprule \midrule
				$\nu \text{ in} \text{ kHz}$ & $U^2_\text{a}$ in $\text{V}^2$ 
				\\
				\midrule
				\phantom{0}3.416  & --0.008          \\
\phantom{0}8.255  & \phantom{0}0.005 \\
13.634            & \phantom{0}0.022 \\
17.100            & \phantom{0}0.073 \\
17.868            & \phantom{0}0.100 \\
19.176            & \phantom{0}0.152 \\
20.538            & \phantom{0}0.263 \\
21.815            & \phantom{0}0.490 \\
21.848            & \phantom{0}0.509 \\
22.098            & \phantom{0}0.580 \\
22.248            & \phantom{0}0.640 \\
22.473            & \phantom{0}0.850 \\
23.036            & \phantom{0}0.960 \\
24.040            & \phantom{0}2.150 \\
25.720            & \phantom{0}4.960 \\
26.197            & \phantom{0}4.280 \\
28.222            & \phantom{0}1.180 \\
29.218            & \phantom{0}1.350 \\
30.067            & \phantom{0}0.520 \\
32.006            & \phantom{0}0.260 \\
33.972            & \phantom{0}0.160 \\
35.967            & \phantom{0}0.100 \\
40.406            & \phantom{0}0.052 \\
44.480            & \phantom{0}0.031 \\
47.013            & \phantom{0}0.019 \\
52.664            & \phantom{0}0.011 \\
				\midrule \bottomrule
			\end{tabular}
			\caption{Messwerte zur Eichmessung der Korrelatorschaltung. Dabei ist $U_\text{a}$ 
			die 
			Ausgangsspannung und $\nu$ die Frequenz des 
			Eingangssignals. }
			\label{tab:eichung_korr}
		\end{table}
		
		Die Durchlasskurve ist in Abbildung \ref{fig:eichung_korr} zu sehen. Dabei wurde die Größe  
		\begin{equation}
		\beta := \frac{U^2_\text{a, Durchlasskurve}-U_\text{a, Eigenrauschen}^2}
		{V_= V_\text{selektiv}^2 V_\text{N}^2 V_\text{V}^2  U^2_\text{ein}}
		\end{equation}
		eingeführt, wobei $V_= =10$, $V_\text{selektiv}=10$ und $V_\text{V}=1000$
		die Verstärkungsfaktoren des Gleichspannungsverstärkers, des Selektivverstärkers und des 
		Vorverstärkers sind. Die ersten 13 Werte wurden mit $V_\text{N}=50$, die weiteren 
		mit $V_\text{N}=20$ aufgenommen.
				
		\begin{figure}
			\centering
			\includegraphics[scale=0.7]{bilder/eichung_korr.pdf}
			\caption{Durchlasskurve der korrelierten Rauschverstärkerschaltung.}
			\label{fig:eichung_korr}
		\end{figure}
		
		Der Eichfaktor ergibt sich als
		\begin{equation}
			A =  0.39+/-0.13 %0.386 \text{ kHz}  \quad .  
		\end{equation}


	
	\subsubsection{Rauschmessung}
		Die Messwerte für die Rauschmessung sind in den Tabellen \ref{tab:rauschen_korr1} 
		und \ref{tab:rauschen_korr2} sowie den Abbildungen 
		\ref{fig:rauschen_korr1} und \ref{fig:rauschen_korr2} zu sehen.
		
		
	\begin{figure}[htbp]

	\begin{minipage}{0.3\textwidth} 

			\centering
			\begin{tabular}{cc}
				\toprule \midrule
				$R$ in $\Omega$ & $U_\text{a}^2$ in $V^2$ \\
				\midrule
				\phantom{0}50\phantom{.} & 0.046             & 1.40184          \\
100\phantom{.}    & 0.242             & 1.40968          \\
150\phantom{.}    & 0.538             & 1.42151          \\
200\phantom{.}    & 0.950             & 1.43799          \\
250\phantom{.}    & 1.390             & 1.45558          \\
300\phantom{.}    & 1.910             & 1.47636          \\
350\phantom{.}    & 2.550             & 1.50194          \\
400\phantom{.}    & 3.150             & 1.52592          \\
450\phantom{.}    & 3.890             & 1.55549          \\
500\phantom{.}    & 4.580             & 1.58306          \\
550\phantom{.}    & 5.310             & 1.61222          \\
600\phantom{.}    & 6.090             & 1.64337          \\
650\phantom{.}    & 6.750             & 1.66972          \\
700\phantom{.}    & 7.520             & 1.70047          \\
750\phantom{.}    & 1.370             & 1.74210          \\
800\phantom{.}    & 1.470             & 1.76704          \\
850\phantom{.}    & 1.600             & 1.79947          \\
900\phantom{.}    & 1.720             & 1.82939          \\
950\phantom{.}    & 1.850             & 1.86181          \\			
				\midrule \bottomrule
			\end{tabular}
			\caption{Messwerte zum Spannungsrauschen am korrelierten 
			Rauschspektrometer vor der Normierung. $R_\text{max}=1000 \Omega$.}
			\label{tab:rauschen_korr1}

	\end{minipage}
	
	\hfill
	
	\begin{minipage}{0.7\textwidth}

			\centering
			\includegraphics[scale=0.7]{bilder/rauschen_korr1.pdf}
			\caption{Spannungsrauschkurve am einfachen Rauschspektrometer für einen 
			Regelbaren Widerstand bis $1000\Omega$. Die Werte wurden bereits auf eine Verstärkung von 
			$1$ normiert, mit $V_= =10$, $V_\text{N}=20 bzw. V_\text{N}=50$, 
			$V_\text{selektiv}=10$ und $V_\text{V}=1000$. }
			\label{fig:rauschen_korr1}
			
	\end{minipage}

	\end{figure}		
	
	
	
	
		\begin{figure}[htbp]

	\begin{minipage}{0.3\textwidth} 

			\centering
			\begin{tabular}{cc}
				\toprule \midrule
				$R$ in $\Omega$ & $U_\text{a}^2$ in $V^2$ \\
				\midrule
				\phantom{0}0.1    & --0.016          \\
\phantom{0}0.9    & \phantom{0}0.000 \\
\phantom{0}2.4    & \phantom{0}0.063 \\
\phantom{0}3.3    & \phantom{0}0.063 \\
\phantom{0}4.6    & \phantom{0}0.088 \\
\phantom{0}7.9    & \phantom{0}0.160 \\
\phantom{0}8.5    & \phantom{0}0.176 \\
10.3              & \phantom{0}0.216 \\
12.0              & \phantom{0}0.241 \\
14.0              & \phantom{0}0.280 \\
16.2              & \phantom{0}0.311 \\
18.3              & \phantom{0}0.338 \\
20.7              & \phantom{0}0.372 \\
24.7              & \phantom{0}0.415 \\
29.1              & \phantom{0}0.448 \\
40.0              & \phantom{0}0.487 \\
49.6              & \phantom{0}0.488 \\
60.1              & \phantom{0}0.480 \\
70.3              & \phantom{0}0.455 \\
81.2              & \phantom{0}0.433 \\
96.5              & \phantom{0}0.385 \\			
				\midrule \bottomrule
			\end{tabular}
			\caption{Messwerte zum Spannungsrauschen am korrelierten  
			Rauschspektrometer. $R_\text{max}=100 \text{k}\Omega$.}
			\label{tab:rauschen_korr2}

	\end{minipage}
	% Auffüllen des Zwischenraums
	\hfill
	% minipage mit Grafik
	\begin{minipage}{0.7\textwidth}

			\centering
			\includegraphics[scale=0.7]{bilder/rauschen_korr2.pdf}
			\caption{Spannungsrauschkurve am korrelierten Rauschspektrometer für einen 
			regelbaren Widerstand bis $100\text{k}\Omega$. Die Werte wurden bereits auf eine 
			Verstärkung von 	$1$ normiert, mit $V_= =10$, $V_\text{selektiv}=10$, $V_\text{N}=200$ und 
			$V_\text{V}=1000$. }
			\label{fig:rauschen_korr2}
			
	\end{minipage}

	\end{figure}		
		

		Die Ausgleichsgeraden sind durch
		\begin{equation}
			G_1(R) = (6.14 \pm 0.06) \times 10^{-15} \frac{\text{V}^2}{\Omega} R 
					- (xxx)\times 10^{-3}\text{V}^2 		
					% G_2(R) = \SI[parse-numbers = false]{\left(4.48 \pm 0.17\right) \times 10^{-17}}{\volt^2\per\ohm}\, \cdot \,R\, + \SI[parse-numbers = false]{\left(3.523 \pm 0.022\right) \times 10^{-12}}{\volt^2}
		\end{equation}
		und
		\begin{equation}
			G_2(R) = (4.5 \pm 6.0) \times 10^{-17} \frac{\text{V}^2}{\Omega} R 
					- (xxx)\times 10^{-3}\text{V}^2 		
					% G_2(R) = \SI[parse-numbers = false]{\left(4.48 \pm 0.17\right) \times 10^{-17}}{\volt^2\per\ohm}\, \cdot \,R\, + \SI[parse-numbers = false]{\left(3.523 \pm 0.022\right) \times 10^{-12}}{\volt^2}
		\end{equation}
		
		Mit der Nyquist-Beziehung erhält man die Boltzmannkonstante erneut als
		\begin{equation}
		k_\text{B}=\frac{m_i}{4 T A} \quad .
		\end{equation}
 		Für eine Raumtemperatur von $300$K folgt
		\begin{align}
		k_\text{B}^{1000\Omega}			=  (1320.0\pm 40.0)\times 10^{-23} \frac{\text{J}}{\text{s}} 
		\\
		%input{tabellen/k_korr1.tex}\\ 
		k_\text{B}^{100\text{k}\Omega}	=	(10\pm 13)\times 10^{-23} \frac{\text{J}}{\text{s}}  
		%input{tabellen/k_korr2.tex}
		\end{align}
	
	



\subsection{Diodenkennlinien}


\subsection{Bestimmung von $\text{e}_0$ mittels Stromrauschens}

\subsection{Frequenzspektrum einer Reinmetallkathode}

\subsection{Frequenzspektrum einer Oxydkathode}