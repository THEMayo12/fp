
% ==================================================
%	Auswertung
% ==================================================

\section{Auswertung}

\subsection{Bestimmung von $k_\text{B}$ mittels Spannungsrauschens am 
			einfachen Rauschspektrometer}
		
	
	\subsubsection{Eichmessung}
	
		Die Messwerte zum kurzgeschlossenen Rauschspektrometer sind 
		in Tabelle \ref{tab:eichung_eigenrauschen_einfach} zu sehen. 
		Es wurde das Ausgangssignal $U^2_\text{a}$ in Abhängigkeit von 
		der Nachverstärkung $V_\text{N}$ gemessen.
		
		\begin{table}[h]
		\centering
			\begin{tabular}{cc}
				\toprule \midrule
				$V_\text{N}$ & $U^2_\text{a}$ in $\text{V}^2$
				\\
				\midrule
				\phantom{000}1.000 & --12.400         \\
\phantom{000}2.000 & --12.100         \\
\phantom{000}5.000 & --12.300         \\
\phantom{00}10.000 & --12.300         \\
\phantom{00}20.000 & --12.100         \\
\phantom{00}50.000 & --11.800         \\
\phantom{0}100.000 & --10.500         \\
\phantom{0}200.000 & \phantom{0}--4.600\\
\phantom{0}500.000 & \phantom{0}35.100\\
1000.000          & \phantom{0}88.800\\
				\midrule \bottomrule
			\end{tabular}
			\caption{Messwerte zum kurzgeschlossenen 
			Rauschspektrometer. Dabei ist $U_\text{a}$ die 
			Ausgangsspannung und $V_\text{N}$ die Nachverstärkung.}
			\label{tab:eichung_eigenrauschen_einfach}
		\end{table}
		
		Die Messwerte der Eichmessung sind in Tabelle 
		\ref{tab:eichung_einfach} zu sehen. Während der Messung 
		betrug $V_\text{N}=20$ und die Ausgangsspannung des 
		Frequenzgenerators $U_\text{ein} = 120 \text{ mV}$.
		
		\begin{table}[h]
		\centering
			\begin{tabular}{cccc}
				\toprule \midrule
				$\nu \text{ in} \text{ kHz}$ & $U^2_\text{a}$ in $\text{V}^2$ &$\nu \text{ in} \text{ 
				kHz}$ & $U^2_\text{a}$ in $\text{V}^2$
				\\
				\midrule
				\phantom{0}1.188  & 0.116            \\
\phantom{0}3.800  & 0.117            \\
\phantom{0}4.169  & 0.117            \\
\phantom{0}5.161  & 0.118            \\
\phantom{0}8.745  & 0.137            \\
\phantom{0}9.261  & 0.140            \\
12.608            & 0.750            \\
14.920            & 1.750            \\
16.747            & 3.310            \\
17.098            & 1.850            \\
18.787            & 5.370            \\
20.192            & 7.240            \\
20.535            & 7.210            \\
21.809            & 8.390            \\
23.009            & 8.870            \\
24.434            & 8.770            \\
26.209            & 8.150            \\
27.809            & 7.230            \\
30.821            & 5.570            \\
32.526            & 3.860            \\
33.041            & 3.320            \\
35.554            & 2.160            \\
37.925            & 1.340            \\
39.882            & 1.020            \\
42.107            & 0.720            \\
45.715            & 0.440            \\
48.860            & 0.310            \\
51.061            & 0.257            \\
52.345            & 0.234            \\
55.561            & 0.190            \\
58.132            & 0.169            \\
60.934            & 0.150            \\
65.829            & 0.141            \\
71.522            & 0.133            \\
75.763            & 0.130            \\
				\midrule \bottomrule
			\end{tabular}
			\caption{Messwerte zur Eichmessung. Dabei ist $U_\text{a}$ 
			die 
			Ausgangsspannung und $\nu$ die Frequenz des 
			Eingangssignals. }
			\label{tab:eichung_einfach}
		\end{table}
		
		Die Durchlasskurve ist in Abbildung \ref{fig:eichung_einfach} zu sehen. Dabei wurde die Größe 
		\begin{equation}
		\beta := \frac{U^2_\text{a, Durchlasskurve}-U_\text{a, Eigenrauschen}^2}
		{V_= V_\text{N}^2 V_\text{V}^2  U^2_\text{ein}}
		\end{equation}
		eingeführt, wobei $V_= =10$, $V_\text{V}=1000$ und $V_\text{N}=20$ (während dieser Messung)
		die Verstärkungsfaktoren des Gleichspannungsverstärkers, des Vorverstärkers beziehungsweise 
		des Nachverstärkers sind.
		
		\begin{figure}
			\centering
			\includegraphics[scale=0.7]{bilder/eichung_einfach.pdf}
			\caption{Durchlasskurve der einfachen Rauschverstärkerschaltung.}
			\label{fig:eichung_einfach}
		\end{figure}
		
		Der Eichfaktor 
		\begin{equation}
			A:=\int\limits_0^\infty V_= V_\text{V}^2(\nu) V_\text{N}^2(\nu) \text{ d}\nu		
		\end{equation}				
		kann dann als
		\begin{equation}
			A = \int \limits_0^\infty \beta(\nu) \text{ d}\nu
		\end{equation}
		über eine Berechnung der Ober- und Untersumme berechnet werden,
		\begin{equation}
			A_\text{ober} = \sum\limits_{\nu_<} \beta_i (\nu_i-\nu_{i-1}) +
			\sum\limits_{\nu_>} \beta_{i-1} (\nu_i-\nu_{i-1})
		\end{equation}
		\begin{equation}
			A_\text{unter} = \sum\limits_{\nu_<} \beta_{i-1} (\nu_i-\nu_{i-1}) +
			\sum\limits_{\nu_>} \beta_{i} (\nu_i-\nu_{i-1})
		\end{equation}
		wobei $\nu_i$ den $i$-ten Messwert darstellt. Außerdem 
		steht $\nu_<$ für die Frequenzen unterhalb und $\nu_>$ für die 
		Frequenzen oberhalb des Maximums. Als Mittelwert ergibt sich
		\begin{equation}
			A =  18.15+/-0.32   \quad . 
		\end{equation}


	
	\subsubsection{Rauschmessung}
		Die Messwerte für die Rauschmessung sind in den Tabellen \ref{tab:rauschen_einfach1} 
		und \ref{tab:rauschen_einfach2} sowie den Abbildungen 
		\ref{fig:rauschen_einfach1} und \ref{fig:rauschen_einfach2} zu sehen.
		
		
	\begin{figure}[htbp]

	\begin{minipage}{0.3\textwidth} 

			\centering
			\begin{tabular}{cc}
				\toprule \midrule
				$R$ in $\Omega$ & $U_\text{a}^2$ in $V^2$ \\
				\midrule
				100.000           & 0.057            \\
150.000           & 0.127            \\
200.000           & 0.222            \\
251.000           & 0.336            \\
300.000           & 0.461            \\
350.000           & 0.598            \\
400.000           & 0.758            \\
450.000           & 0.923            \\
500.000           & 1.083            \\
550.000           & 1.263            \\
600.000           & 1.440            \\
650.000           & 1.631            \\
700.000           & 1.810            \\
750.000           & 2.000            \\
800.000           & 2.190            \\
850.000           & 2.390            \\
900.000           & 2.570            \\
950.000           & 2.730            \\			
				\midrule \bottomrule
			\end{tabular}
			\caption{Messwerte zum Spannungsrauschen am einfachen 
			Rauschspektrometer vor der Normierung. $R_\text{max}=1000 \Omega$.}
			\label{tab:rauschen_einfach1}

	\end{minipage}
	% Auffüllen des Zwischenraums
	\hfill
	% minipage mit Grafik
	\begin{minipage}{0.7\textwidth}

			\centering
			\includegraphics[scale=0.7]{bilder/rauschen_einfach1.pdf}
			\caption{Spannungsrauschkurve am einfachen Rauschspektrometer für einen 
			Regelbaren Widerstand bis $1000\Omega$. Die Werte wurden bereits auf 
			eine Verstärkung von 
			$1$ normiert, mit $V_= =10$, $V_\text{N}=200$ und $V_\text{V}=1000$. }
			\label{fig:rauschen_einfach1}
			
	\end{minipage}
	% \caption{noch eine Caption}
	\end{figure}		
	
	
	
	
		\begin{figure}[htbp]

	\begin{minipage}{0.3\textwidth} 

			\centering
			\begin{tabular}{cc}
				\toprule \midrule
				$R$ in $\Omega$ & $U_\text{a}^2$ in $V^2$ \\
				\midrule
				\phantom{0}2.0    & 0.019            \\
\phantom{0}4.3    & 0.059            \\
\phantom{0}6.7    & 0.095            \\
\phantom{0}8.3    & 0.120            \\
\phantom{0}9.0    & 0.145            \\
12.2              & 0.180            \\
14.1              & 0.212            \\
16.4              & 0.242            \\
19.5              & 0.287            \\
22.9              & 0.330            \\
25.5              & 0.373            \\
30.2              & 0.419            \\
34.9              & 0.474            \\
35.8              & 0.480            \\
40.0              & 0.525            \\
45.5              & 0.571            \\
50.6              & 0.602            \\
65.1              & 0.656            \\
62.2              & 0.682            \\
68.0              & 0.698            \\
75.0              & 0.715            \\
78.9              & 0.732            \\
85.4              & 0.727            \\
92.1              & 0.745            \\
97.2              & 0.805            \\			
				\midrule \bottomrule
			\end{tabular}
			\caption{Messwerte zum Spannungsrauschen am einfachen 
			Rauschspektrometer. $R_\text{max}=100 \text{k}\Omega$.}
			\label{tab:rauschen_einfach2}

	\end{minipage}
	% Auffüllen des Zwischenraums
	\hfill
	% minipage mit Grafik
	\begin{minipage}{0.7\textwidth}

			\centering
			\includegraphics[scale=0.7]{bilder/rauschen_einfach2.pdf}
			\caption{Spannungsrauschkurve am einfachen Rauschspektrometer für einen 
			Regelbaren Widerstand bis $100\text{k}\Omega$. Die Werte wurden bereits auf eine 
			Verstärkung von 	$1$ normiert, mit $V_= =10$, $V_\text{N}=200$ und $V_\text{V}=1000$. }
			\label{fig:rauschen_einfach2}
			
	\end{minipage}

	\end{figure}		
		

		Die Ausgleichsgeraden sind durch
		\begin{equation}
			G_1(R) = \SI[parse-numbers = false]{\left(9.10 \pm 0.07\right) \times 10^{-15}}{\volt^2\per\ohm}\, \cdot \,R\, + \SI[parse-numbers = false]{\left(9.68 \pm 0.05\right) \times 10^{-12}}{\volt^2}
		\end{equation}
		und
		\begin{equation}
			G_2(R) = \SI[parse-numbers = false]{\left(3.72 \pm 0.08\right) \times 10^{-17}}{\volt^2\per\ohm}\, \cdot \,R\, + \SI[parse-numbers = false]{\left(3.024994 \pm 0.000011\right) \times 10^{-9}}{\volt^2}
		\end{equation}
		
		Mit der Nyquist-Beziehung erhält man
		\begin{equation}
			U_\text{a}^2 =4k_\text{B}T A R \quad , 
		\end{equation}
		sodass mit den Steigungen $m_1$ und $m_2$ der Geradengleichungen $G1(R)$ und $G2(R)$ 
		die Boltzmannkonstante als
		\begin{equation}
		k_\text{B}=\frac{m_i}{4 T A}
		\end{equation}
		berechnet werden kann. Für eine Raumtemperatur von $300$K folgt
		\begin{align}
		k_\text{B}^{1000\Omega}			=  41.8+/-0.8\times 10^{-23}\frac{\text{J}}
																			{\text{K}}  \\ 
		k_\text{B}^{100\text{k}\Omega}	=  0.171+/-0.005\times 10^{-23}\frac{\text{J}}
																			{\text{K}} 
		\end{align}
	
	\subsubsection{Ermittlung der Rauschzahl}
		Für das einfache Rauschspektrometer wird die Rauschzahl 
		\begin{equation}
			F = \frac{U_\text{a}^2}{4 k_\text{B}T R A}
		\end{equation}
		bei $R\approx 500\Omega$ berechnet. Mit den bereits berechneten Größen und 
		dem Wertepaar $(500\Omega,1.083 \text{V}^2)$ ergibt sich 
		\begin{equation}
			F = 0.665+/-0.005 \times 10^{3}
		\end{equation}
		
		
		
		
		
		
\clearpage		
\subsection{Bestimmung von $k_\text{B}$ mittels Spannungsrauschens am 
			Korrelator-Rauschspektrometer}
			
		\subsubsection{Eichmessung}
	
		Die Messwerte zum kurzgeschlossenen Rauschspektrometer sind 
		in Tabelle \ref{tab:eichung_eigenrauschen_korr} zu sehen. 
		Es wurde das Ausgangssignal $U^2_\text{a}$ in Abhängigkeit von 
		der Nachverstärkung $V_\text{N}$ gemessen.
		
		\begin{table}[h]
		\centering
			\begin{tabular}{cc}
				\toprule \midrule
				$V_\text{N}$ & $U^2_\text{a}$ in $\text{V}^2$
				\\
				\midrule
				\phantom{000}1\phantom{.} & --0.014          \\
\phantom{000}2\phantom{.} & --0.014          \\
\phantom{000}5\phantom{.} & --0.014          \\
\phantom{00}10\phantom{.} & --0.014          \\
\phantom{00}20\phantom{.} & --0.014          \\
\phantom{00}50\phantom{.} & --0.014          \\
\phantom{0}100\phantom{.} & --0.012          \\
\phantom{0}200\phantom{.} & --0.007          \\
\phantom{0}500\phantom{.} & \phantom{0}0.033 \\
1000\phantom{.}   & \phantom{0}0.174 \\
				\midrule \bottomrule
			\end{tabular}
			\caption{Messwerte zum kurzgeschlossenen  
			Rauschspektrometer nach Korrelationspinzip. Dabei ist $U_\text{a}$ die 
			Ausgangsspannung und $V_\text{N}$ die Nachverstärkung.}
			\label{tab:eichung_eigenrauschen_korr}
		\end{table}
		
		Die Messwerte der Eichmessung sind in Tabelle 
		\ref{tab:eichung_korr} zu sehen. Während der Messung 
		betrug $V_\text{N}=2$ und die Ausgangsspannung des 
		Frequenzgenerators $U_\text{ein} = 120 \text{ mV}$.
		
		\begin{table}[h]
		\centering
			\begin{tabular}{cccc}
				\toprule \midrule
				$\nu \text{ in} \text{ kHz}$ & $U^2_\text{a}$ in $\text{V}^2$ &
				$\nu \text{ in} \text{ kHz}$ & $U^2_\text{a}$ in $\text{V}^2$ 
				\\
				\midrule
				\phantom{0}3.416  & --0.008          \\
\phantom{0}8.255  & \phantom{0}0.005 \\
13.634            & \phantom{0}0.022 \\
17.100            & \phantom{0}0.073 \\
17.868            & \phantom{0}0.100 \\
19.176            & \phantom{0}0.152 \\
20.538            & \phantom{0}0.263 \\
21.815            & \phantom{0}0.490 \\
21.848            & \phantom{0}0.509 \\
22.098            & \phantom{0}0.580 \\
22.248            & \phantom{0}0.640 \\
22.473            & \phantom{0}0.850 \\
23.036            & \phantom{0}0.960 \\
24.040            & \phantom{0}2.150 \\
25.720            & \phantom{0}4.960 \\
26.197            & \phantom{0}4.280 \\
28.222            & \phantom{0}1.180 \\
29.218            & \phantom{0}1.350 \\
30.067            & \phantom{0}0.520 \\
32.006            & \phantom{0}0.260 \\
33.972            & \phantom{0}0.160 \\
35.967            & \phantom{0}0.100 \\
40.406            & \phantom{0}0.052 \\
44.480            & \phantom{0}0.031 \\
47.013            & \phantom{0}0.019 \\
52.664            & \phantom{0}0.011 \\
				\midrule \bottomrule
			\end{tabular}
			\caption{Messwerte zur Eichmessung der Korrelatorschaltung. Dabei ist $U_\text{a}$ 
			die 
			Ausgangsspannung und $\nu$ die Frequenz des 
			Eingangssignals. }
			\label{tab:eichung_korr}
		\end{table}
		
		Die Durchlasskurve ist in Abbildung \ref{fig:eichung_korr} zu sehen. Dabei wurde die Größe  
		\begin{equation}
		\beta := \frac{U^2_\text{a, Durchlasskurve}-U_\text{a, Eigenrauschen}^2}
		{V_= V_\text{selektiv}^2 V_\text{N}^2 V_\text{V}^2  U^2_\text{ein}}
		\end{equation}
		eingeführt, wobei $V_= =10$, $V_\text{selektiv}=10$ und $V_\text{V}=1000$
		die Verstärkungsfaktoren des Gleichspannungsverstärkers, des Selektivverstärkers und des 
		Vorverstärkers sind. Die ersten 13 Werte wurden mit $V_\text{N}=50$, die weiteren 
		mit $V_\text{N}=20$ aufgenommen.
				
		\begin{figure}
			\centering
			\includegraphics[scale=0.7]{bilder/eichung_korr.pdf}
			\caption{Durchlasskurve der korrelierten Rauschverstärkerschaltung.}
			\label{fig:eichung_korr}
		\end{figure}
		
		Der Eichfaktor ergibt sich als
		\begin{equation}
			A =  0.39+/-0.13  \quad .  
		\end{equation}


	\clearpage
	\subsubsection{Rauschmessung}
		Die Messwerte für die Rauschmessung sind in den Tabellen \ref{tab:rauschen_korr1} 
		und \ref{tab:rauschen_korr2} sowie den Abbildungen 
		\ref{fig:rauschen_korr1} und \ref{fig:rauschen_korr2} zu sehen.
		
		
	\begin{figure}[htbp]

	\begin{minipage}{0.3\textwidth} 

			\centering
			\begin{tabular}{cc}
				\toprule \midrule
				$R$ in $\Omega$ & $U_\text{a}^2$ in $V^2$ \\
				\midrule
				\phantom{0}50\phantom{.} & 0.046             & 1.40184          \\
100\phantom{.}    & 0.242             & 1.40968          \\
150\phantom{.}    & 0.538             & 1.42151          \\
200\phantom{.}    & 0.950             & 1.43799          \\
250\phantom{.}    & 1.390             & 1.45558          \\
300\phantom{.}    & 1.910             & 1.47636          \\
350\phantom{.}    & 2.550             & 1.50194          \\
400\phantom{.}    & 3.150             & 1.52592          \\
450\phantom{.}    & 3.890             & 1.55549          \\
500\phantom{.}    & 4.580             & 1.58306          \\
550\phantom{.}    & 5.310             & 1.61222          \\
600\phantom{.}    & 6.090             & 1.64337          \\
650\phantom{.}    & 6.750             & 1.66972          \\
700\phantom{.}    & 7.520             & 1.70047          \\
750\phantom{.}    & 1.370             & 1.74210          \\
800\phantom{.}    & 1.470             & 1.76704          \\
850\phantom{.}    & 1.600             & 1.79947          \\
900\phantom{.}    & 1.720             & 1.82939          \\
950\phantom{.}    & 1.850             & 1.86181          \\			
				\midrule \bottomrule
			\end{tabular}
			\caption{Messwerte zum Spannungsrauschen am korrelierten 
			Rauschspektrometer vor der Normierung. $R_\text{max}=1000 \Omega$.}
			\label{tab:rauschen_korr1}

	\end{minipage}
	
	\hfill
	
	\begin{minipage}{0.6\textwidth}

			\centering
			\includegraphics[scale=0.6]{bilder/rauschen_korr1.pdf}
			\caption{Spannungsrauschkurve am einfachen Rauschspektrometer für einen 
			Regelbaren Widerstand bis $1000\Omega$. Die Werte wurden bereits auf eine Verstärkung von 
			$1$ normiert, mit $V_= =10$, $V_\text{N}=20 bzw. V_\text{N}=50$, 
			$V_\text{selektiv}=10$ und $V_\text{V}=1000$. }
			\label{fig:rauschen_korr1}
			
	\end{minipage}

	\end{figure}		
	
	
	
	
		\begin{figure}[htbp]

	\begin{minipage}{0.3\textwidth} 

			\centering
			\begin{tabular}{cc}
				\toprule \midrule
				$R$ in $\Omega$ & $U_\text{a}^2$ in $V^2$ \\
				\midrule
				\phantom{0}0.1    & --0.016          \\
\phantom{0}0.9    & \phantom{0}0.000 \\
\phantom{0}2.4    & \phantom{0}0.063 \\
\phantom{0}3.3    & \phantom{0}0.063 \\
\phantom{0}4.6    & \phantom{0}0.088 \\
\phantom{0}7.9    & \phantom{0}0.160 \\
\phantom{0}8.5    & \phantom{0}0.176 \\
10.3              & \phantom{0}0.216 \\
12.0              & \phantom{0}0.241 \\
14.0              & \phantom{0}0.280 \\
16.2              & \phantom{0}0.311 \\
18.3              & \phantom{0}0.338 \\
20.7              & \phantom{0}0.372 \\
24.7              & \phantom{0}0.415 \\
29.1              & \phantom{0}0.448 \\
40.0              & \phantom{0}0.487 \\
49.6              & \phantom{0}0.488 \\
60.1              & \phantom{0}0.480 \\
70.3              & \phantom{0}0.455 \\
81.2              & \phantom{0}0.433 \\
96.5              & \phantom{0}0.385 \\			
				\midrule \bottomrule
			\end{tabular}
			\caption{Messwerte zum Spannungsrauschen am korrelierten  
			Rauschspektrometer. $R_\text{max}=100 \text{k}\Omega$.}
			\label{tab:rauschen_korr2}

	\end{minipage}

	\hfill

	\begin{minipage}{0.6\textwidth}

			\centering
			\includegraphics[scale=0.6]{bilder/rauschen_korr2.pdf}
			\caption{Spannungsrauschkurve am korrelierten Rauschspektrometer für einen 
			regelbaren Widerstand bis $100\text{k}\Omega$. Die Werte wurden bereits auf eine 
			Verstärkung von 	$1$ normiert, mit $V_= =10$, $V_\text{selektiv}=10$, $V_\text{N}=200$ und 
			$V_\text{V}=1000$. }
			\label{fig:rauschen_korr2}
			
	\end{minipage}

	\end{figure}		
		

		Die Ausgleichsgeraden sind durch
		\begin{equation}
			G_1(R) = \SI[parse-numbers = false]{\left(6.07 \pm 0.06\right) \times 10^{-15}}{\volt^2\per\ohm}\, \cdot \,R\, + \SI[parse-numbers = false]{\left(1.282 \pm 0.004\right) \times 10^{-11}}{\volt^2}
		\end{equation}
		und
		\begin{equation}
			G_2(R) = \SI[parse-numbers = false]{\left(4.48 \pm 0.17\right) \times 10^{-17}}{\volt^2\per\ohm}\, \cdot \,R\, + \SI[parse-numbers = false]{\left(3.523 \pm 0.022\right) \times 10^{-12}}{\volt^2}
		\end{equation}
		
		Mit der Nyquist-Beziehung erhält man die Boltzmannkonstante erneut als
		\begin{equation}
		k_\text{B}=\frac{m_i}{4 T A} \quad .
		\end{equation}
 		Für eine Raumtemperatur von $300$K folgt
		\begin{align}
		k_\text{B}^{1000\Omega}			=  1.3+/-0.4 \times 10^{-20}\frac{\text{J}}
																			{\text{K}} \\ 
		k_\text{B}^{100\text{k}\Omega}	=  9.6+/-3.2 \times 10^{-23}\frac{\text{J}}
																			{\text{K}} 
		\end{align}
	
	


\clearpage
\subsection{Diodenkennlinien}
	Zwei Kennlinien der in diesem Versuch benutzten Kathode sind in 
	den Tabellen \ref{tab:kennlinie1} und \ref{tab:kennlinie2} sowie 
	den Abbildungen \ref{fig:kennlinie1} und \ref{fig:kennlinie2} zu 
	sehen.
	
	\begin{figure}[htbp]
	\begin{minipage}{0.3\textwidth} 
			\centering
			\begin{tabular}{ccc}
				\toprule \midrule
				$U_\text{Anode}$ in V & $U_\text{a}^2$ in $V^2$ 
				& $V_\text{N}$\\
				\midrule
				\phantom{0}10.000 & 0.240             & 5.000            \\
\phantom{0}25.000 & 0.390             & 5.000            \\
\phantom{0}45.000 & 0.750             & 2.000            \\
\phantom{0}60.000 & 1.100             & 2.000            \\
\phantom{0}80.000 & 1.100             & 2.000            \\
\phantom{0}95.000 & 1.170             & 2.000            \\
110.000           & 1.170             & 2.000            \\
130.000           & 1.150             & 2.000            \\
150.000           & 1.130             & 2.000            \\			
				\midrule \bottomrule
			\end{tabular}
			\caption{Diodenkennlinie bei einem Heizstrom von 
			$0.8$A.}
			\label{tab:kennlinie1}
	\end{minipage}
	\hfill
	\begin{minipage}{0.7\textwidth}
			\centering
			\includegraphics[scale=0.7]{bilder/kennlinie1.pdf}
			\caption{Diodenkennlinie bei einem Heizstrom von 
			$0.8$A.Die Werte wurden bereits um die 
			Verstärkungsfaktoren $V_\text{N}$ normiert. }
			\label{fig:kennlinie1}		
	\end{minipage}
	\end{figure}		
	
	
	\begin{figure}[htbp]
	\begin{minipage}{0.3\textwidth} 
			\centering
			\begin{tabular}{ccc}
				\toprule \midrule
				$U_\text{Anode}$ in V & $U_\text{a}^2$ in $V^2$ 
				& $V_\text{N}$\\
				\midrule
				\phantom{0}10.000 & 0.240             & 5.000            \\
\phantom{0}25.000 & 0.390             & 5.000            \\
\phantom{0}45.000 & 0.750             & 2.000            \\
\phantom{0}60.000 & 1.100             & 2.000            \\
\phantom{0}80.000 & 1.100             & 2.000            \\
\phantom{0}95.000 & 1.170             & 2.000            \\
110.000           & 1.170             & 2.000            \\
130.000           & 1.150             & 2.000            \\
150.000           & 1.130             & 2.000            \\			
				\midrule \bottomrule
			\end{tabular}
			\caption{Diodenkennlinie bei einem Heizstrom von 
			$0.9$A.}
			\label{tab:kennlinie2}
	\end{minipage}
	\hfill
	\begin{minipage}{0.7\textwidth}
			\centering
			\includegraphics[scale=0.7]{bilder/kennlinie2.pdf}
			\caption{Diodenkennlinie bei einem Heizstrom von 
			$0.9$A. Die Werte wurden bereits um die 
			Verstärkungsfaktoren $V_\text{N}$ normiert.}
			\label{fig:kennlinie2}		
	\end{minipage}
	\end{figure}		
	
	
	
\clearpage
\subsection{Bestimmung von $\text{e}_0$ mittels Stromrauschens}

	Die Messwerte sind in Tabelle \ref{tab:elementarladung_messwerte} 
	zu sehen.
	\begin{table}[h]
	\centering
		\begin{tabular}{cccc}
		\toprule 
		\midrule
			$I_0$ in mA & $U^2_\text{a}$ in $\text{V}^2$ &
			$\Delta U_\text{a}^2$ in $\text{V}^2$ & $V_\text{N}$ \\
			\midrule
			0.5               & 1.14              & 0.00              & 50\phantom{.}    \\
1.0               & 2.87              & 0.02              & 50\phantom{.}    \\
1.5               & 4.10              & 0.02              & 50\phantom{.}    \\
2.0               & 5.45              & 0.02              & 50\phantom{.}    \\
2.5               & 1.11              & 0.02              & 20\phantom{.}    \\
3.0               & 1.42              & 0.02              & 20\phantom{.}    \\
3.5               & 1.69              & 0.02              & 20\phantom{.}    \\
4.0               & 1.91              & 0.02              & 20\phantom{.}    \\
			\midrule 
			\bottomrule
		\end{tabular}
		\caption{Messwerte zur Bestimmung der Elementarladung 
		and der Reinmetallkathode. Dabei ist $I_0$ der eigestellte 
		Anodenstrom.}
		\label{tab:elementarladung_messwerte}
	\end{table}
	
	Über die Schottkybeziehung für das Schrotrauschen 
	\begin{equation}
		I^2 = 2 \text{e}_0 I_0 \Delta \nu
	\end{equation}		
	ergibt sich die Elementarladung als 
	\begin{equation}
	\text{e}_0 = \frac{m}{2 \Delta \nu} \quad ,
	\end{equation}
	wobei $m$ die Steigung der Ausgleichsgeraden 
	\begin{equation}
	G(I_0) = \SI[parse-numbers = false]{\left(5.65 \pm 0.12\right) \times 10^{-15}}{\ampere}\, \cdot \,I_0\, - \SI[parse-numbers = false]{\left(8.6 \pm 3.2\right) \times 10^{-19}}{\ampere^2}
	\end{equation}
	darstellt. Die über das Ohmsche Gesetz und die 
	Verstärkungsfaktoren $V_= =10$, $V_\text{V}=1000$ und 
	$V_\text{N}$ in das Rauschstromquadrat $I^2$ umgerechneten 
	Messwerte sowie die Ausgleichsgerade sind in Abbildung 
	\ref{fig:elementarladung} zu sehen. Außerdem wurde das 
	Frequenzband ($100$-$120$)kHz gewählt, sodass in dieser 
	Versuchsdurchführung $\nu = 20000$Hz beträgt.
		
	\begin{figure}
		\centering
		\includegraphics[scale=0.7]{bilder/elementarladung.pdf}
		\caption{Darstellung der Messwerte zur Ermittlung der 
		Elementarladung $\text{e}_0$. Die gemessene Ausgangsspannung 
		wurde bereits in den Rauschstrom umgerechnet.}
		\label{fig:elementarladung}
	\end{figure}
	
	Für den Wert der Elementarladung ergibt sich 
	\begin{equation}
	\text{e}_0 = (6.39+/-0.14)e-25 \times 
				10^{-19}\text{C} \quad .
	\end{equation}
	
\clearpage
\subsection{Frequenzspektrum einer Reinmetallkathode}

	Die aufgenommenen Messwerte sind in Tabelle 
	\ref{tab:kathode_rein} zu sehen. Es wurde bereits das 
	entsprechende $\Delta \nu$ gemäß der beiliegenden Tabelle 
	ausgerechnet. Eine Darstellung der Ausgangsspannung $U_a^2$ in 
	Abhängikeit von der Rauschfrequenz $\nu$ ist in Abbildung 
	\ref{fig:kathode_rein} zu sehen.
	\begin{table}
		\centering
		\begin{tabular}{ccccc}
		\toprule \midrule
		$\nu$ in kHz & $U_a^2$ in $\text{V}^2$& $\Delta U_a^2$ in   
		$\text{V}^2$& $V_\text{N}$ & $\Delta \nu$ in kHz\\
		\midrule
		460.000           & 0.487             & 0.002             & \phantom{0}20\phantom{.} & 21.500            & \phantom{0}6.131  & \phantom{00}0.259 & --42.799         \\
400.000           & 0.519             & 0.030             & \phantom{0}20\phantom{.} & 24.300            & \phantom{0}5.991  & \phantom{00}0.244 & --42.858         \\
360.000           & 0.630             & 0.040             & \phantom{0}20\phantom{.} & 24.400            & \phantom{0}5.886  & \phantom{00}0.295 & --42.668         \\
300.000           & 0.555             & 0.005             & \phantom{0}20\phantom{.} & 23.400            & \phantom{0}5.704  & \phantom{00}0.271 & --42.753         \\
260.000           & 0.460             & 0.004             & \phantom{0}20\phantom{.} & 20.600            & \phantom{0}5.561  & \phantom{00}0.255 & --42.813         \\
220.000           & 0.505             & 0.005             & \phantom{0}20\phantom{.} & 21.000            & \phantom{0}5.394  & \phantom{00}0.274 & --42.739         \\
180.000           & 0.405             & 0.005             & \phantom{0}20\phantom{.} & 16.100            & \phantom{0}5.193  & \phantom{00}0.287 & --42.694         \\
160.000           & 0.379             & 0.005             & \phantom{0}20\phantom{.} & 14.400            & \phantom{0}5.075  & \phantom{00}0.300 & --42.649         \\
120.000           & 0.302             & 0.002             & \phantom{0}20\phantom{.} & 11.600            & \phantom{0}4.787  & \phantom{00}0.297 & --42.660         \\
100.000           & 2.020             & 0.010             & \phantom{0}50\phantom{.} & 12.100            & \phantom{0}4.605  & \phantom{00}0.305 & --42.634         \\
\phantom{0}64.000 & 1.390             & 0.020             & \phantom{0}50\phantom{.} & \phantom{0}8.170  & \phantom{0}4.159  & \phantom{00}0.311 & --42.615         \\
\phantom{0}33.000 & 0.770             & 0.010             & \phantom{0}50\phantom{.} & \phantom{0}4.500  & \phantom{0}3.497  & \phantom{00}0.312 & --42.610         \\
\phantom{0}17.000 & 0.400             & 0.010             & \phantom{0}50\phantom{.} & \phantom{0}2.350  & \phantom{0}2.833  & \phantom{00}0.311 & --42.615         \\
\phantom{0}10.000 & 0.240             & 0.002             & \phantom{0}50\phantom{.} & \phantom{0}1.400  & \phantom{0}2.303  & \phantom{00}0.313 & --42.608         \\
\phantom{00}6.400 & 0.140             & 0.005             & \phantom{0}50\phantom{.} & \phantom{0}0.890  & \phantom{0}1.856  & \phantom{00}0.287 & --42.694         \\
\phantom{00}3.300 & 0.070             & 0.003             & \phantom{0}50\phantom{.} & \phantom{0}0.460  & \phantom{0}1.194  & \phantom{00}0.278 & --42.727         \\
\phantom{00}1.700 & 0.030             & 0.005             & \phantom{0}50\phantom{.} & \phantom{0}0.240  & \phantom{0}0.531  & \phantom{00}0.228 & --42.924         \\
\phantom{00}1.000 & 0.084             & 0.005             & 100\phantom{.}    & \phantom{0}0.140  & \phantom{0}0.000  & \phantom{00}0.274 & --42.741         \\
\phantom{00}0.640 & 0.050             & 0.001             & 100\phantom{.}    & \phantom{0}0.090  & --0.446           & \phantom{00}0.254 & --42.818         \\
\phantom{00}0.330 & 0.130             & 0.010             & 200\phantom{.}    & \phantom{0}0.047  & --1.109           & \phantom{00}0.316 & --42.599         \\
\phantom{00}0.170 & 0.079             & 0.050             & 200\phantom{.}    & \phantom{0}0.025  & --1.772           & \phantom{00}0.361 & --42.466         \\
\phantom{00}0.100 & 0.040             & 0.010             & 200\phantom{.}    & \phantom{0}0.015  & --2.303           & \phantom{00}0.304 & --42.636         \\
\phantom{00}0.064 & 0.080             & 0.010             & 200\phantom{.}    & \phantom{0}0.009  & --2.749           & \phantom{00}0.982 & --41.465         \\
\phantom{00}0.033 & 0.250             & 0.010             & 200\phantom{.}    & \phantom{0}0.005  & --3.411           & \phantom{00}6.203 & --39.621         \\
\phantom{00}0.017 & 0.700             & 0.050             & 200\phantom{.}    & \phantom{0}0.002  & --4.075           & \phantom{0}34.739 & --37.899         \\
\phantom{00}0.010 & 1.800             & 0.100             & 200\phantom{.}    & \phantom{0}0.001  & --4.605           & 171.214           & --36.304         \\
		\midrule
		\bottomrule
		\end{tabular}
		\caption{Messwerte und Durchlassbreiten $\Delta \nu$ für 
		die Reinmetallkathode.}
		\label{tab:kathode_rein}
	\end{table}

	\begin{figure}
		\centering
		\includegraphics[scale=0.7]{bilder/kathode_rein.pdf}
		\caption{Darstellung des Frequenzspektrums $W(\nu)$ in 
		Abhängikeit von der Rauschfrequenz $\nu$.}
		\label{fig:kathode_rein}
	\end{figure}

\clearpage
\subsection{Frequenzspektrum einer Oxydkathode}

	Die aufgenommenen Messwerte sind in Tabelle 
	\ref{tab:kathode_oxyd} zu sehen. Es wurde bereits das 
	entsprechende $\Delta \nu$ gemäß der beiliegenden Tabelle 
	ausgerechnet. Eine Darstellung der Ausgangsspannung $U_a^2$ in 
	Abhängikeit von der Rauschfrequenz $\nu$ ist in Abbildung 
	\ref{fig:kathode_oxyd} zu sehen.
	\begin{table}
		\centering
		\begin{tabular}{ccccc}
		\toprule \midrule
		$\nu$ in kHz & $U_a^2$ in $\text{V}^2$& $\Delta U_a^2$ in   
		$\text{V}^2$& $V_\text{N}$ & $\Delta \nu$ in kHz\\
		\midrule
		460.000           & 1.435             & 0.002             & \phantom{0}50\phantom{.} & 34.300            & 13.039            & \phantom{00}0.346 & --33.298          & --32.050         \\
440.000           & 1.697             & 0.001             & \phantom{0}50\phantom{.} & 35.200            & 12.995            & \phantom{00}0.398 & --33.156          & --32.769         \\
400.000           & 1.610             & 0.003             & \phantom{0}50\phantom{.} & 35.800            & 12.899            & \phantom{00}0.372 & --33.226          & --31.687         \\
340.000           & 1.392             & 0.002             & \phantom{0}50\phantom{.} & 33.500            & 12.737            & \phantom{00}0.343 & --33.305          & --32.026         \\
300.000           & 1.303             & 0.003             & \phantom{0}50\phantom{.} & 29.500            & 12.612            & \phantom{00}0.365 & --33.244          & --31.494         \\
260.000           & 1.038             & 0.002             & \phantom{0}50\phantom{.} & 24.500            & 12.468            & \phantom{00}0.350 & --33.286          & --31.713         \\
220.000           & 0.992             & 0.004             & \phantom{0}50\phantom{.} & 23.600            & 12.301            & \phantom{00}0.347 & --33.294          & --30.983         \\
180.000           & 0.810             & 0.002             & \phantom{0}50\phantom{.} & 18.500            & 12.101            & \phantom{00}0.362 & --33.253          & --31.433         \\
160.000           & 0.671             & 0.002             & \phantom{0}50\phantom{.} & 16.300            & 11.983            & \phantom{00}0.340 & --33.314          & --31.306         \\
140.000           & 0.614             & 0.002             & \phantom{0}50\phantom{.} & 14.400            & 11.849            & \phantom{00}0.352 & --33.279          & --31.182         \\
120.000           & 0.545             & 0.002             & \phantom{0}50\phantom{.} & 12.200            & 11.695            & \phantom{00}0.369 & --33.233          & --31.016         \\
100.000           & 2.720             & 0.100             & 100\phantom{.}    & 12.550            & 11.513            & \phantom{00}0.448 & --33.040          & --27.132         \\
\phantom{0}70.000 & 2.030             & 0.100             & 100\phantom{.}    & \phantom{0}9.100  & 11.156            & \phantom{00}0.461 & --33.011          & --26.811         \\
\phantom{0}64.000 & 2.010             & 0.100             & 100\phantom{.}    & \phantom{0}8.400  & 11.067            & \phantom{00}0.494 & --32.941          & --26.731         \\
\phantom{0}33.000 & 1.170             & 0.010             & 100\phantom{.}    & \phantom{0}4.500  & 10.404            & \phantom{00}0.537 & --32.858          & --28.409         \\
\phantom{0}17.000 & 0.650             & 0.010             & 100\phantom{.}    & \phantom{0}2.300  & \phantom{0}9.741  & \phantom{00}0.584 & --32.774          & --27.738         \\
\phantom{0}10.000 & 0.410             & 0.010             & 100\phantom{.}    & \phantom{0}1.400  & \phantom{0}9.210  & \phantom{00}0.605 & --32.739          & --27.242         \\
\phantom{00}6.400 & 1.280             & 0.010             & 200\phantom{.}    & \phantom{0}0.890  & \phantom{0}8.764  & \phantom{00}0.743 & --32.533          & --26.789         \\
\phantom{00}3.300 & 0.800             & 0.010             & 200\phantom{.}    & \phantom{0}0.460  & \phantom{0}8.102  & \phantom{00}0.898 & --32.343          & --26.129         \\
\phantom{00}1.700 & 0.750             & 0.010             & 200\phantom{.}    & \phantom{0}0.240  & \phantom{0}7.438  & \phantom{00}1.614 & --31.757          & --25.478         \\
\phantom{00}1.000 & 0.190             & 0.030             & 200\phantom{.}    & \phantom{0}0.140  & \phantom{0}6.908  & \phantom{00}0.701 & --32.591          & --23.841         \\
\phantom{00}0.640 & 0.365             & 0.002             & 200\phantom{.}    & \phantom{0}0.093  & \phantom{0}6.461  & \phantom{00}2.027 & --31.530          & --26.140         \\
\phantom{00}0.330 & 0.332             & 0.004             & 200\phantom{.}    & \phantom{0}0.047  & \phantom{0}5.799  & \phantom{00}3.656 & --30.940          & --24.762         \\
\phantom{00}0.170 & 0.310             & 0.010             & 200\phantom{.}    & \phantom{0}0.025  & \phantom{0}5.136  & \phantom{00}6.536 & --30.359          & --23.196         \\
\phantom{00}0.100 & 0.290             & 0.010             & 200\phantom{.}    & \phantom{0}0.015  & \phantom{0}4.605  & \phantom{0}10.190 & --29.915          & --22.685         \\
\phantom{00}0.064 & 0.290             & 0.010             & 200\phantom{.}    & \phantom{0}0.009  & \phantom{0}4.159  & \phantom{0}16.107 & --29.457          & --22.228         \\
\phantom{00}0.033 & 0.310             & 0.010             & 200\phantom{.}    & \phantom{0}0.005  & \phantom{0}3.497  & \phantom{0}34.435 & --28.697          & --21.534         \\
\phantom{00}0.017 & 0.320             & 0.020             & 200\phantom{.}    & \phantom{0}0.002  & \phantom{0}2.833  & \phantom{0}73.462 & --27.939          & --20.115         \\
\phantom{00}0.010 & 0.250             & 0.020             & 200\phantom{.}    & \phantom{0}0.001  & \phantom{0}2.303  & 107.610           & --27.558          & --19.487         \\
		\midrule
		\bottomrule
		\end{tabular}
		\caption{Messwerte und Durchlassbreiten $\Delta \nu$ für 
		die Oxydkathode.}
		\label{tab:kathode_oxyd}
	\end{table}

	\begin{figure}
		\centering
		\includegraphics[scale=0.7]{bilder/kathode_oxyd.pdf}
		\caption{Darstellung des Frequenzspektrums $W(\nu)$ in 
		Abhängikeit von der Rauschfrequenz $\nu$.}
		\label{fig:kathode_oxyd}
	\end{figure}
	
	Die Ausgleichsgerade durch die ersten acht Werte wird durch 
	\begin{equation}
	G(f) = \SI[parse-numbers = false]{-0.979 \pm 0.018}{}\, \cdot \,f\, - \SI[parse-numbers = false]{24.6 \pm 0.5}{}
	\end{equation}
	beschrieben, wobei Steigung $\alpha$ der Ausgleichsgeraden 
	\begin{equation}
	\alpha = 0.66+/-0.24
	\end{equation}
	den Koeffizienten des Funkelrauschens darstellt.