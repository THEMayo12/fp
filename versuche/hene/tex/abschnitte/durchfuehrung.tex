\section{Aufbau und Justage}
Der Laser besteht aus den Komponenten Spiegel 1 (S1), Spiegel 2 (S2) und der 
HeNe-Röhre (R), die auf eine optische Bank angebracht werden können. Auf deren 
einem Ende befindet sich außerdem ein weiterer Laser (JL) sowie am anderen eine 
Photodiode. R schließt an beiden Seiten mit einem Brewster-Fenster ab, sodass die Energieverluste für die entsprechende Polarisationsrichtung deutlich größer sind als diejenigen zur dazu senkrechten Polarisationsrichtung, wodurch der Laser. Dadurch beginnt der Laser bereits bei kleinen Leistungen  in einer Polarisationsrichtung zu lasen, während für die andere eine deutlich höhere Leistung nötig ist. Der Laser wird bei einer solchen Leistung betrieben, dass das Licht vollständig linearpolarisiert ist. Der Laser wird wie folgt aufgebaut.
\begin{enumerate}
\item R wird auf die optische Bank gebracht. Mit Hilfe von JL wird R mittels 
ihrer Justierschrauben so eingestellt, dass der Laserstrahl ohne Verluste 
durch sie Verläuft.
\item Der teildurchlässige Spiegel S2 wird unmittelbar hinter R auf die 
optische Bank gebracht, sodass der Laserstrahl nach Durchqueren von R 
reflektiert wird. Mit den Justierschrauben von S2 wird dessen Ausrichtung so 
eingestellt, dass der reflektierte Strahl möglichst auf dem eintreffenden 
liegt.
\item S1 wird auf der anderen Seite von R auf die optische Bank gebracht und
ebenfalls so einjustiert, dass der reflektierte Strahl auf dem eintreffenden 
liegt.
\item Die Laserröhre R wird angeschaltet, sodass durch elektrische Entladungen 
eine Besetzungsinversion hervorgerufen wird. 
\item Durch leichtes Variieren von S1 kann es zum Lasen kommen.
\end{enumerate}

\section{Durchführung}
\subsection{Überprüfen der Stabilitätsbedingung}
Für zwei Spiegelkonfigurationen wird die Stabilitätsbedingung überprüft. Dazu 
wird der Laser für einen kleinen Spiegelabstand $L$ zum lasen gebracht. Nun 
wird $L$ schrittweise erhöht und mit der Photodiode die Leistung des 
Laserstrahles gemessen. Dabei wird S1 jedes mal leicht nachjustiert, um 
möglichst lange ein Laserverthalten zu erreichen.
\subsection{Vermessen von TEM-Moden}
Der Laserstrahl wird nach dem Austreten durch S2 mit einer Streulinse vergrößert. Mit der Photodiode wird nun die Intensität in Abhängigkeit vom 
Abstand zur Strahlachse gemessen. Dann wird ein dünner Wolframdraht innerhalb des Resonators mittig in den Strahl gebracht. Dadurch wird die TM$_01q$-Mode 
erzeugt. Diese wird ebenfalls vermessen.
\subsection{Vermessen der Polarisation des Laserstrahles}
Nachdem der Strahl ausgekoppelt ist wird ein Polarisationsfilter in den 
Strahlgang gebracht. Mit der Photodiode wird die Strahlintensität in 
Abhängigkeit von der Ausrichtung des Polarisationsfilters gemessen.
\subsection{Bestimmen der Wellenlänge}
Ein Gitter (200 Linien/mm) wird nachdem der Strahl ausgekoppelt ist in den 
Strahlgang gebracht. Das Beugungsbild ist auf einem Schirm, der $l=60\text{ 
cm}$ vom Gitter entfernt steht, zu sehen. Hier werden nun die Abstände der 
Intensitätsmaxima zum Hauptmaximum gemessen.