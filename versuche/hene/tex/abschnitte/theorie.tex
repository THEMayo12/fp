\section{Zielsetzung}
Das Ziel dieses Versuches ist es, die Funktionsweise eines HeNe-Lasers zu
untersuchen. Dazu werden die Stbilitätsbedingungen des Lasers sowie zwei TEM-
Moden untersucht. Außerdem wird die Polarisationsrichtung des Laserstrahles
sowie seine Wellenlänge bestimmt.\cite{AP}


\section{Theorie}
\subsection{Photonwechselwirkung mit einem Atom}
Besitzt ein Atom zwei Energieniveaus 1 und 2 der Energien $E_1$ und $E_2>E_1$, so kann ein Photon der Energie $E=E_2-E_1$ auf folgende zwei Arten damit
wechselwirken.
\begin{description}
\item[Absorption:] Befindet sich ein Elektron in 1 und ist 2 nicht voll
besetzt, so kann dieses das Photon absorbieren wodurch es in 2 übergeht.
\item[Stimulierte Emission] Befindet sich ein Elektron in 2 und ist 1 nicht voll besetzt, so kann das Elektron unter Emission eines weiteren Photons, welches in seinen Eigenschaften identisch mit dem wechselwirkenden ist, in 1 wechseln.
\end{description}
Die Wahrscheinlichkeiten für das Auftreten der beiden Fälle sind gleich.
Außerdem kann es zur spontanen Emission kommen. Dabei wird ein Photon der Energie $E$ mit zufälligen Eigenschaften emittiert.
\subsection{Photonwechselwirkung im Mehratomesystem}
In einem Mehratomesystem mit Energieniveaus 1 (Grundzustand) und 2 (angeregter
Zustand) der Energien $E_1$ und
$E_2$ und den Besetzungdichten $n_1$ und $n_2$ überwiegt gemäß der Maxwell-
Boltzmann-Verteilung der Grundzustand, also $n_1>n_2$. Um einen kohärenten
Lichtstrahl zu erhalten muss die stimulierte Emission überwiegen, es muss
demnach der Zustand 2 höher Besetzt sein als der Zustand 1. Erreicht wird
dies durch eine Besetzungsinversion.

Beim HeNe-Laser ist Neon das aktive Material; die Besetzungsinversion
geschieht wie folgt. Durch elektrische Entladungen werden die Helium Atome in
einen energetisch höheren Zustand gebracht. Durch Stöße zweiter Art mit den
Neonatomen werden diese in einen angeregten Zustand überführt, was einer
Besetzungsinversion entspricht.

Da die Besetzung der Zustände längs des aktiven Mediums homogen und die
Wahrscheinlichkeit der Wechselwirkung eines Photons mit einem Elektron für
alle Elektronen gleich ist, wächst die Verstärkung exponentiell zur Länge des
durch das Medium verlaufenden Weges an. Um diesen Weg zu verlängern und damit
die Verstärkung zu erhöhen wird mit zwei Spiegeln ein Resonator erzeugt,
sodass der Lichtstrahl das aktive Medium oft durchläuft. Um den Lichtstrahl
auszukoppeln ist einer der Spiegel teildurchlässig. Um einen selbsterregenden
Oszillator zu erhalten, müssen die Verluste kleiner als die Verstärkung sein.
Dazu muss insbesondere die optische Bedingung
\begin{equation}
0\leq g_1 \cdot g_2 <1 \label{eq:Resonanzbedingung}
\end{equation}
mit den Resonatorparametern $g_\text{i}=1-L/r_\text{i}$ erfüllt sein. Dabei
steht $L$ für die Länge des Resonators und $r_\text{i}$ für den
Krümmungsradius des $i$. Spiegels.

Durch Unebenheiten und Verkippungen der Spiegel wird werden im Experiment
allgemein transversale Schwingungsmoden TM$_{lpq}$ auftreten, wobei $l$, $p$
und $q$ die Anzahl der Knoten in x-, y- (transversal) und z-Richtung
(longitudinal) angeben. Die entsprechende Feldverteilung für einen konfokalen
Resonator kann durch
\begin{equation}
E_{lpq}\propto \cos(l\varphi)\frac{4\varrho^2}{(1+Z^2)^{(1+l)/2}}L_p^q\left(\frac{4\varrho^2}{1+Z^2}\right) \text{e}^{-\frac{\varrho^2}{1+Z^2}-\text{i}\left( \frac{(1+Z)\uppi R}{\lambda} +\frac{\varrho^2 Z}{1+Z^2}-(l+1p+1)\left(\frac{\uppi}{2}-\arctan\left( \frac{1-Z}{1+Z} \right) \right) \right)}  \label{eq:Feldverteilung}
\end{equation}
beschrieben werden, wobei $\varrho^2=\frac{2\uppi}{R\lambda}$ und $RZ=2z$. Es
ist $L_p^q$ das zugeordnete Laguerrepolynom.
Für die TEM$_{00q}$-Mode ergibt sich damit
\begin{equation}
I(r)=I_0 e^{-\frac{2r^2}{w^2}} \label{eq:TEM00} \quad ,
\end{equation}
wobei $r$ der Abstand zur optischen Achse ist.

\subsection{Dopplereffekt bei longitudinalen Moden}
Neben der Bedingung \eqref{eq:Resonanzbedingung} zur optischen Stabilität
eines Resonators muss außerdem die Bedingung
\begin{equation}
L=n\frac{\text{c}}{2 \nu}
\end{equation}
erfüllt sein, die sicherstellt, dass eine stehende Welle der Ordnung $n$ mit
der Frequenz $\nu$ und Ausbreitungsgeschwindigkeit c entsteht. Zu einer
gegebenen Resonatorlänge $L_0$ entsteht also ein Linienspektrum möglicher Frequenzen $\nu_n=\frac{n \text{c}}{2L_0}$. Eine alternative Sichtweise ist,  dass, da der Laser
auf dem 3s-2p Übergang im Neon arbeitet, die Frequenz durch $\nu_0=632.8 \text{ nm}$ fest ist und die Resonatorlängen durch $L_n=\frac{n\text{c}}{2\nu_0}$
festgelegt werden.

Da den Geschwindigkeiten der Neonatome eine Maxwell-Boltzmann-Verteilung
zugrunde liegt, kommt es zum Dopplereffekt. Während ein (im Laborsystem)
ruhendes Atom die Frequenz $\nu_0$ verstärkt, so wird ein dazu bewegtes
Atom ein Photon der Frequenz $\nu$ abstrahlen, welche leicht von $\nu_0$
verschieden ist. Auf diese Weise entsteht eine gaußförmige Verstärkungskurve.

Laut der Maxwell-Boltzmann-Verteilung hat ein Teilchen der Masse $m$ in einem 
System der Temperatur $T$ mit einer Wahrscheinlichkeit von
\begin{equation}
f(v)=4\pi \left( \frac{m}{2\pi \text{k}_\text{B} T} \right)^\frac{3}{2} v^2 
\text{e}^{-\frac{mv^2}{2\text{k}_\text{B}T}}
\end{equation}
die (betragsmäßige) Geschwindigkeit $v$. Dabei stellt $\text{k}_\text{B}$ die 
Boltzmann-Konstante dar. Hieraus ergibt sich eine mittlere Geschwindigkeit von 
\begin{equation}
\bar{v}=\sqrt{\frac{8\text{k}_\text{B}T}{\pi m}} \quad .
\end{equation}
Gemäß dem Doppler-Gesetz, zeigt eine mit der Geschwindikeit $v$ gegenüber einem 
ruhenden Beobachter bewegte Signalquelle eine verschobene Frequenz von 
\begin{equation}
\nu = \frac{\nu_0}{1-\frac{v}{c}}
\end{equation}
gegenüber der Frequenz $\nu_0$ im Ruhesystem der Quelle, wobei $c$ die 
Ausbreitungsgeschwindigkeit der Welle darstellt.


Um die Frequenzverschiebung durch den Doppler-Effekt beurteilen zu können, werden 
nun der Abstand zwischen zwei erlaubten Frequenzen bei gegebener Resonatorlänge,
sowie die Verschiebung durch den Doppler-Effekt berechnet:
\begin{align}
\Delta \nu &= \nu_{n+1}-\nu_n=\frac{c}{2 L_0} \approx 300\text{ MHz} \quad , \\
\nu _1 -\nu_0&=  \frac{\nu_0}{1-\frac{\bar{v}}{c}}-\nu_0 \approx 890 \text{ MHz} \quad .
\end{align}
Der Berechnung wurden die Werte $m=20.18 \text{ u}$, $L_0\approx 0.5 \text{ m}$ und  
die Lichtgeschwindigkeit $c$ zu Grunde gelegt. Offensichtlich ist der Abstand 
zwischen zwei erlaubten Frequenzen viel Größer als die Verschiebung durch den 
Doppler-Effekt. Es kann also davon ausgegangen werden, dass der Laser in einer 
bestimmten Mode arbeitet.