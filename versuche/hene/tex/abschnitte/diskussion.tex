
% ==================================================
%	Diskussion
% ==================================================

\section{Diskussion}

In den Abbildungen \ref{fig:S1} und \ref{fig:S2} lassen sich die
Stabilitätsbedingungen für den Resonator mit den Radien $r_1 = r_2 =
\SI{1400}{\mm}$ und den Resonantor mit
$r_1 = \SI{1000}{\mm}$ und $r_2 = \SI{1400}{\mm}$ verifizieren.
In den Bereichen der Stabilitätsbedingung, wo die Kurve negativ wird, wird auch
kein Diodenstrom gemessen.

Desweiteren wurden die transversalen Moden \TEMN und \TEME vermessen.
An diesen Daten wurde die theoretische Verteilung \ref{eq:Feldverteilung}
gefittet.
Die entsprechenden Daten und Fits sind in \ref{fig:M1} und \ref{fig:M2} zu
finden. Es zeigt sich, dass der Fit die gemessenen Werte gut nähert.
Es sind jedoch leichte Abweichungen, überwiegend in dem Fit zur \TEME-Mode aus
Abbildung \ref{fig:M1}, zu erkennen.
% Erwartet wird die Überlagerung eines Polynoms von Grad 2 mit einem Gauß.
% Betrachtung der Abbildung \ref{fig:M1} in dem Bereich von $r = \SI{8}{\mm}$ bis
% \SI{15}{\mm}, so
Grund hierfür wird experimenteller Natur, wie z.B. das zusätzliche Messen von
Streulicht oder fehlerhafte Justage, sein.
Zudem ist zu erwähnen, dass die beiden Peaks der vermessenen \TEME-Mode
unterschiedlich groß sind.
Dies kann daran liegen, dass der Draht nicht exakt mittig in den Strahlengang
des Lasers gebracht wurde.

Weiterhin wurde die Polarisation des Lasers vermessen.
Die gemessenen Werte und der Fit sind in Abbildung \ref{fig:P} zu sehen.
Hierin sind ebenfalls leichte Abweichungen zu erkennen, wobei der Verlauf
entsprechend $\sim \sin^2\varphi$ zu erkennen ist.
Der Grund für die Polarisation des Lasers liegt an den Brewsterfenstern.
An den Brewsterfenstern wird s-polarisiertes Licht reflektiert und
p-polarisiertes Licht transmittiert.
Das transmittierte Licht ist das Licht, was schließlich als Laser verwendet wird.
Das s-polarisierte Licht wird über das Brewsterfenster aus dem Resonator
ausgekoppelt, sodass überwiegend p-polarisiertes Licht im Resonator
über die stimulierte Emission verstärkt wird.

Zuletzt wurde noch die Wellenlänge des Laser mit Hilfe eines Gitters bestimmt.
Diese beträgt
\begin{equation}
	\lambda = \SI[parse-numbers = false]{(622.3 \pm 25.9)}{\nano\meter} ~.
\end{equation}
