
% ==================================================
%	Einleitung
% ==================================================

\section{Einleitung}
Gegenstand dieses Versuches sind die Leitungseigenschaften von Koaxialkabeln sowie 
das Verhalten von Signalen bei der Reflexion an verschiedenen Abschlussimpedanzen. 
Zunächst werden die Leitungskonstanten $R$, $G$, $L$ und $C$ eingeführt, um eine 
Beschreibung eines realen Leiters zu ermöglichen und damit das Verhalten eines 
Signals bei Ausbreitung und Reflexion modelliert, Dann wird die Methode der 
Smith-Diagramme erklärt. Nach einer Beschreibung von Aufbau und Versuchsdurchführung 
werden die Messergebnisse zu den Leitungskonstanten, der Dämpfungskonstante und zur 
Mehrfachreflexion an verschiedenen Abschlüssen vorgestellt.