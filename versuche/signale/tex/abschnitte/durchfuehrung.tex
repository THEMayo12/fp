
% ==================================================
%	Durchführung
% ==================================================

\section{Durchführung}
In diesem Versuch stehen drei Koaxialkabel (20m-,50 $\Omega$; 50m-,75 $\Omega$) 
zur 
verfügung. Außerdem ein Oszilloskop, ein Nim-Pulser und diverse 
Abschlusswiderstände.

\begin{enumerate}
\item 	Es wird ein LRC-Messgerät an das 20m-Koaxialkabel angeschlossen, und die 
		Leitungskonstanten $R$, $L$ und $C$ in Abhängigkeit von der angelegten 
		Frequenz gemessen. Dabei ist das andere Ende des Kabels kurzgeschlossen.

\item 	Der Aufbau geschieht nun nach Abbildung \ref{fig:Aufbau}. 
		Die Dämpfungskonstante $\alpha$ wird für die drei Koaxialkabel bestimmt. 
		Dazu wird ein Eingangsimpuls (Nim-Pulser) mit hohem Oberwellenanteil  
		benutzt, und die 
		Fourierkoeffizienten der Oberwellen gemessen.
		
\item 	An einem Koaxialkabel mit zuerst offenem, dann kurzgeschlossenem Ende 
		werden die Eingangsimpedanzen gemessen.	
		
\item	Es werden 
		nacheinander drei Abschlusswiderstände unbekannter Eingenschaft 
		angeschlossen und das Bild am Oszilloskop aufgenommen.

\item	Das $50 \,\Omega$ und $75 \, \Omega$ Kabel werden in Reihe geschaltet und 
		das	Oszilloskopbild aufgenommen.
		
		
\end{enumerate}

