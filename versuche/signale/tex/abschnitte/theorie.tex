
% ==================================================
%	Theorie
% ==================================================

\section{Theorie}
\subsection{Die verlustfreie und reale Leitung}
Um die Eigenschaften einer verlustfreien Leitung zu modellieren kann das in Abbildung 
\ref{fig:verlustfrei} dargestellte Ersatzschaltbild benutzt werden, wobei $L$ eine 
Spule und $C$ einen Kondensator darstellen. Um nun eine reale, verlustbehaftete 
Leitung zu simulieren werden die Widerstände $R$ und $G$ gemäß Abbildung 
\ref{fig:verlustbehaftet} in das Ersatzschaltbild integriert. Eine reale Leitung 
wird nun durch den Kapazitätsbelag $C$, den Induktivitätsbelag $L$, den ohmschen 
Belag $R$ und den Querleitfähigkeitsbelag $G$ charakterisiert. Die (komplexe) 
Spannung $U(t,z)$ zur Zeit $t$ am Ort $z$ kann dann durch

%fig

\begin{equation}
U(t,z)=U_0 \e^{-\gamma z}\e^{\i\omega t} \label{eq:Spannung}
\end{equation}
beschrieben werden, wobei $\gamma=\alpha + \beta \i =\sqrt{(R+\i \omega L)(G+\i 
\omega C)}$ die Ausbreitungskonsante mit dem Dämpfungsbelag $\alpha$ und dem 
Phasenbelag $\beta$ ist und $\omega$ die Kreisfrequenz der angelegten Spannung mit 
Amplitude $U_0$. Daraus 
lässt sich der Wellenwiderstand $Z_0$ ableiten als das Verhältnis von 
Spannungsamplitude zu Stromamplitude
\begin{equation}
Z_0:= \frac{U(\omega)}{I(\omega)} = \sqrt{\frac{R+\i \omega L}{G+\i \omega C}}
\label{eq:Wellenwiderstand} \quad ,
\end{equation}
hier für den Spezialfall, dass das Leitungskabel an jeder Stelle die gleichen 
Eigenschaften hat. Durch die Frequenzabhängigkeit von \eqref{eq:Spannung} und 
\eqref{eq:Wellenwiderstand} kommt es zur Dispersion und somit zur Verzerrung von 
Signalen, die auf die Leitung gegeben werden.
\subsection{Spannungsimpulse auf Leitungen}
Oft ist es nützlich, das Verhalten eines Spannungsimpulses auf einer Leitung zu 
untersuchen. Hierzu werden neben $Z_0$ noch die Quellenimpedanz $Z_\text{g}$ und 
die Lastimpedanz $Z_\text{L}$ benötigt. Ist $U_0$ die Spannung des eingehenden 
Impulses, und $U_\text{r}$ die des am anderen Ende reflektierten, so gilt für die 
Spannung $U_\text{L}$ an der Lastimpedanz
\begin{equation}
U_\text{L}=U_0+U_\text{r} \quad .
\end{equation}
Außerdem lässt sich der Reflexionsfaktor $\Gamma$ als
\begin{equation}
\Gamma := \frac{U_\text{r}}{U_0} =\frac{Z_\text{L}-Z_0}{Z_L + Z_0} =|\Gamma|e^{\i 
\varphi_\Gamma} \label{eq:Gamma}
\end{equation}
definieren. Die Signalspannung in Ortsdarstellung lässt sich dann mit einer 
Laplacetransformation $\mathfrak{L}$ aus der 
Impulsdarstellung gemäß
\begin{equation}
U_\text{r}(t)=\mathfrak{L}^{-1}(U_\text{r}(p))=\mathfrak{L}^{-1}(\Gamma(p)U_\text{h}
(p))
\end{equation}
bestimmen, wobei $\Gamma$ der Reflexionsfaktor in Impulsdarstellung ist, und 
$U_\text{r}(p)$ und $U_\text{h}(p)$ die Spannung des reflektierten bzw. hinlaufenden 
Spannungspulses ist. Diese Rechnung entspricht der Näherung, dass die Leitung nahezu 
verlustfrei ist.
\subsection{Der Skin-Effekt bei Koaxialkabeln}
In diesem Verusuch werden Koaxialkabel verwendet. Diese bestehen aus einem inneren 
Leiter als Kern. Dieser wird von einem Dielektrikum und dieses wiederum von einem 
äußeren Leiter umgeben. Durch die magnetischen Felder innerhalb des Kabels werden 
Wirbelströme induziert, welche den eigentlichen Wechselstrom an den äußeren Rand der 
Leiter drängt, sodass der Leitungsquerschnitt verkleinert und somit der Widerstand 
vergrößert wird. Aus diesem Grund wird der Widerstand $R$ eines Koaxialkabels bei 
Kreisfrequenzen ab $100 \text{ kHz}$ durch ein $\sqrt{\omega}$-Gesetzt angemessen 
beschrieben.
\subsection{Smith-Diagramme}
Smith-Diagramme sind ein Hilfsmittel, um zu einem gegebenen Lastwiderstand 
$Z_\text{L}$ und Wellenwiderstand $Z_0$ den Reflexionsfaktor $\Gamma$ numerisch 
zu ermitteln. Dazu wird \eqref{eq:Gamma} mit $z_\text{L}:=Z_\text{L}/Z_0$ 
umgeschrieben zu
\begin{equation}
\Gamma = \frac{z_\text{L}-1}{z_\text{L}+1} \quad .
\end{equation}
Interpretiert man nun $\Gamma(z_\text{L})$ als komplexe Funktion ist dies eine 
Möbiustransformation, sodass, da diese Abbildung konform ist, die Bilder der 
Koordinatenachsen im Zielraum ebenfalls ein Koordinatessystem bilden. In Abbildung 
\ref{fig:Smith} werden über ein (nicht eingezeichnetes) Kartesisches 
Koordinatensystem die Bilder der Koordinatenlinien eingezeichnet. Um nun hieraus 
einen Reflexionskoeffizienten zu bestimmen, muss lediglich $z_\text{L}$ in das 
transformierte Koordinatensystem eingetragen werden. Dieser Punkt entspricht, wenn 
er in Kartesischen Koordinaten abgelesen wird gerade $\Gamma(z_\text{L})$.

%fig
