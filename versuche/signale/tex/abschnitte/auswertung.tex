
% ==================================================
%	Definitionen
% ==================================================

\newcommand{\CU}{RG-58C/U}
\newcommand{\BU}{RG-58B/U}

% ==================================================
%	Auswertung
% ==================================================

\section{Auswertung}

%%%%%%%%%%%%%%%%%%%%%%%%%%%%%%%%%%%%%%%%%%%%%%%%%%%%%%%%%%%%%%%%%%%%%%%%%%%
%                           Leitungskonstanten                            %
%%%%%%%%%%%%%%%%%%%%%%%%%%%%%%%%%%%%%%%%%%%%%%%%%%%%%%%%%%%%%%%%%%%%%%%%%%%

\subsection{Leitungskonstanten}
\label{sub:leitungskonstanten}

Es sollen die mit einem LCU-Messgerät bestimmen Leitungskonstanten
$R$, $L$ und $C$ in Abhängigkeit der Frequenz $\nu$ von drei Koaxialkabel,
einem langen und einem kurzen \CU\ sowie einem kurzen \BU\ dargestellt werden.
Des weiteren soll der Leitwert $G$ berechnet und ebenfalls graphisch
dargestellt werden.
Die im Versuch gemessenen Werte befinden sich dabei in den
Tabellen~\ref{tab:Leitungskonstanten_50k},~\ref{tab:Leitungskonstanten_50l}
und~\ref{tab:Leitungskonstanten_75k}.
Die in den Tabellen angegebenen Daten sind in den Abbildungen~\ref{fig:R_50k}
bis~\ref{fig:G_75k} graphisch dargestellt.

% ==================================================
% 	Tabellen
% ==================================================

\begin{table}[htpb]
  \centering
  \begin{tabular}{ccccc}
    \midrule
    \midrule
    $\nu / \si{kHz}$        & $C / \si{\pico\farad}$     & $R / \si{\ohm}$ &
    $L / \si{\micro\henry}$ & $G / \si{\milli\siemens}$ \\
    \midrule
    \phantom{0}0.2    & 990.14            & 0.5228            & 2.50              & 207.1            \\
\phantom{0}0.4    & 990.07            & 0.5220            & 3.10              & 166.7            \\
\phantom{0}0.6    & 990.10            & 0.5217            & 3.10              & 166.6            \\
\phantom{0}0.8    & 990.14            & 0.5215            & 3.20              & 161.4            \\
\phantom{0}1.0    & 990.04            & 0.5213            & 3.20              & 161.3            \\
\phantom{0}2.0    & 989.90            & 0.5212            & 3.18              & 162.2            \\
\phantom{0}3.0    & 989.93            & 0.5210            & 3.18              & 162.2            \\
\phantom{0}4.0    & 989.88            & 0.5210            & 3.19              & 161.7            \\
\phantom{0}5.0    & 989.90            & 0.5211            & 3.18              & 162.2            \\
\phantom{0}6.0    & 989.91            & 0.5211            & 3.18              & 162.2            \\
\phantom{0}7.0    & 989.93            & 0.5212            & 3.18              & 162.2            \\
\phantom{0}8.0    & 989.94            & 0.5213            & 3.18              & 162.3            \\
\phantom{0}9.0    & 989.92            & 0.5215            & 3.18              & 162.3            \\
10.0              & 989.95            & 0.5216            & 3.18              & 162.4            \\
11.0              & 990.01            & 0.5218            & 3.18              & 162.4            \\
12.0              & 990.05            & 0.5219            & 3.18              & 162.5            \\
13.0              & 990.07            & 0.5223            & 3.18              & 162.6            \\
14.0              & 990.12            & 0.5225            & 3.18              & 162.7            \\
15.0              & 990.13            & 0.5220            & 3.18              & 162.5            \\
16.0              & 990.01            & 0.5216            & 3.18              & 162.4            \\
17.0              & 990.01            & 0.5215            & 3.18              & 162.4            \\
18.0              & 990.04            & 0.5219            & 3.18              & 162.5            \\
19.0              & 989.98            & 0.5222            & 3.18              & 162.6            \\
20.0              & 989.96            & 0.5226            & 3.18              & 162.7            \\
    \midrule
    \midrule
  \end{tabular}
  \caption{Darstellung der in Abhängigkeit der Frequenz $\nu$ gemessenen
      Leitungskonstanten $R$, $L$ und $C$ für das kurze \CU-Kabel.}
\label{tab:Leitungskonstanten_50k}
\end{table}

\begin{table}[htpb]
  \centering
  \begin{tabular}{ccccc}
    \midrule
    \midrule
    $\nu / \si{kHz}$        & $C / \si{\nano\farad}$     & $R / \si{\ohm}$ &
    $L / \si{\micro\henry}$ & $G / \si{\siemens}$ \\
    \midrule
    \phantom{0}0.2    & 8.5013            & 4.1958            & 24.9              & 1.4              \\
\phantom{0}0.4    & 8.5015            & 4.1992            & 26.4              & 1.4              \\
\phantom{0}0.6    & 8.5016            & 4.1989            & 26.2              & 1.4              \\
\phantom{0}0.8    & 8.5014            & 4.1988            & 27.0              & 1.3              \\
\phantom{0}1.0    & 8.5012            & 4.1988            & 26.0              & 1.4              \\
\phantom{0}2.0    & 8.5004            & 4.1991            & 26.0              & 1.4              \\
\phantom{0}3.0    & 8.5003            & 4.1979            & 25.9              & 1.4              \\
\phantom{0}4.0    & 8.5003            & 4.2010            & 25.9              & 1.4              \\
\phantom{0}5.0    & 8.5003            & 4.2014            & 25.9              & 1.4              \\
\phantom{0}6.0    & 8.5007            & 4.2022            & 25.9              & 1.4              \\
\phantom{0}7.0    & 8.5011            & 4.2058            & 25.9              & 1.4              \\
\phantom{0}8.0    & 8.5012            & 4.2062            & 25.9              & 1.4              \\
\phantom{0}9.0    & 8.5018            & 4.2094            & 25.9              & 1.4              \\
10.0              & 8.5022            & 4.2093            & 25.9              & 1.4              \\
11.0              & 8.5028            & 4.2114            & 25.9              & 1.4              \\
12.0              & 8.5034            & 4.2135            & 25.9              & 1.4              \\
13.0              & 8.5040            & 4.2177            & 25.9              & 1.4              \\
14.0              & 8.5048            & 4.2191            & 25.9              & 1.4              \\
15.0              & 8.5056            & 4.2202            & 25.9              & 1.4              \\
16.0              & 8.5064            & 4.2235            & 25.9              & 1.4              \\
17.0              & 8.5071            & 4.2266            & 25.9              & 1.4              \\
18.0              & 8.5079            & 4.2297            & 25.9              & 1.4              \\
19.0              & 8.5075            & 4.2345            & 25.9              & 1.4              \\
20.0              & 8.5082            & 4.2690            & 25.9              & 1.4              \\
    \midrule
    \midrule
  \end{tabular}
  \caption{Darstellung der in Abhängigkeit der Frequenz $\nu$ gemessenen
      Leitungskonstanten $R$, $L$ und $C$ für das lange \CU-Kabel.}
\label{tab:Leitungskonstanten_50l}
\end{table}

\begin{table}[htpb]
  \centering
  \begin{tabular}{ccccc}
    \midrule
    \midrule
    $\nu / \si{kHz}$        & $C / \si{\pico\farad}$     & $R / \si{\ohm}$ &
    $L / \si{\micro\henry}$ & $G / \si{\milli\siemens}$ \\
    \midrule
    \phantom{0}0.20   & 674.3300          & 1.9534            & 5.0               & 263.4            \\
\phantom{0}0.40   & 674.4100          & 1.9521            & 6.2               & 212.3            \\
\phantom{0}0.60   & 674.4700          & 1.9517            & 6.1               & 215.8            \\
\phantom{0}0.80   & 674.4600          & 1.9515            & 6.1               & 215.8            \\
\phantom{0}1.00   & 674.4400          & 1.9512            & 6.1               & 215.7            \\
\phantom{0}2.00   & 674.2300          & 1.9526            & 6.1               & 216.2            \\
\phantom{0}3.00   & 674.1900          & 1.9554            & 6.0               & 217.9            \\
\phantom{0}4.00   & 674.1900          & 1.9590            & 6.0               & 219.8            \\
\phantom{0}5.00   & 674.2000          & 1.9636            & 6.0               & 221.8            \\
\phantom{0}6.00   & 674.2200          & 1.9690            & 5.9               & 224.2            \\
\phantom{0}7.00   & 674.2300          & 1.9755            & 5.9               & 226.1            \\
\phantom{0}8.00   & 674.1800          & 1.9822            & 5.8               & 230.0            \\
\phantom{0}9.00   & 674.2000          & 1.9895            & 5.8               & 233.3            \\
10.00             & 674.1700          & 1.9971            & 5.7               & 236.6            \\
11.00             & 674.1900          & 2.0048            & 5.6               & 240.1            \\
12.00             & 674.2100          & 2.0127            & 5.6               & 243.6            \\
13.00             & 674.2500          & 2.0205            & 5.5               & 247.7            \\
14.00             & 674.2600          & 2.0281            & 5.4               & 251.4            \\
15.00             & 674.2700          & 2.0354            & 5.4               & 255.1            \\
16.00             & 674.2600          & 2.0428            & 5.3               & 258.4            \\
17.00             & 674.2600          & 2.0500            & 5.3               & 262.3            \\
18.00             & 674.3500          & 2.0570            & 5.2               & 266.2            \\
19.00             & 674.4000          & 2.0636            & 5.2               & 269.7            \\
20.00             & 674.3800          & 2.0699            & 5.1               & 273.2            \\
    \midrule
    \midrule
  \end{tabular}
  \caption{Darstellung der in Abhängigkeit der Frequenz $\nu$ gemessenen
      Leitungskonstanten $R$, $L$ und $C$ für das \BU-Kabel.}
\label{tab:Leitungskonstanten_75k}
\end{table}

% ==================================================
% 	Plots
% ==================================================

\begin{figure}[t]
	\centering
	\includegraphics[scale=1.0]{bilder/R_50k.pdf}
	\caption{Darstellung des Widerstandes $R$ in Abhängigkeit der Frequenz $\nu$
	für das kurze \CU-Kabel.}
	\label{fig:R_50k}
	\includegraphics[scale=1.0]{bilder/R_50l.pdf}
	\caption{Darstellung des Widerstandes $R$ in Abhängigkeit der Frequenz $\nu$
	für das lange \CU-Kabel.}
	\label{fig:R_50l}
\end{figure}

\begin{figure}[h!]
	\centering
	\includegraphics[scale=1.0]{bilder/R_75k.pdf}
	\caption{Darstellung des Widerstandes $R$ in Abhängigkeit der Frequenz $\nu$
	für das \BU-Kabel.}
	\label{fig:R_75k}
	\includegraphics[scale=1.0]{bilder/C_50k.pdf}
	\caption{Darstellung der Kapazität $C$ in Abhängigkeit der Frequenz $\nu$
	für das kurze \CU-Kabel.}
	\label{fig:C_50k}
\end{figure}

\begin{figure}[h!]
	\centering
	\includegraphics[scale=1.0]{bilder/C_50l.pdf}
	\caption{Darstellung der Kapazität $C$ in Abhängigkeit der Frequenz $\nu$
	für das lange \CU-Kabel.}
	\label{fig:C_50l}
	\includegraphics[scale=1.0]{bilder/C_75k.pdf}
	\caption{Darstellung der Kapazität $C$ in Abhängigkeit der Frequenz $\nu$
	für das \BU-Kabel.}
	\label{fig:C_75k}
\end{figure}

\begin{figure}[h!]
	\centering
	\includegraphics[scale=1.0]{bilder/L_50k.pdf}
	\caption{Darstellung der Induktivität $L$ in Abhängigkeit der Frequenz $\nu$
	für das kurze \CU-Kabel.}
	\label{fig:L_50k}
	\includegraphics[scale=1.0]{bilder/L_50l.pdf}
	\caption{Darstellung der Induktivität $L$ in Abhängigkeit der Frequenz $\nu$
	für das lange \CU-Kabel.}
	\label{fig:L_50l}
\end{figure}

\begin{figure}[h!]
	\centering
	\includegraphics[scale=1.0]{bilder/L_75k.pdf}
	\caption{Darstellung der Induktivität $L$ in Abhängigkeit der Frequenz $\nu$
	für das \BU-Kabel.}
	\label{fig:L_75k}
	\includegraphics[scale=1.0]{bilder/G_50k.pdf}
	\caption{Darstellung des Leitwerts $G$ in Abhängigkeit der Frequenz $\nu$
	für das kurze \CU-Kabel.}
	\label{fig:G_50k}
\end{figure}

\begin{figure}[h!]
	\centering
	\includegraphics[scale=1.0]{bilder/G_50l.pdf}
	\caption{Darstellung des Leitwerts $G$ in Abhängigkeit der Frequenz $\nu$
	für das lange \CU-Kabel.}
	\label{fig:G_50l}
	\includegraphics[scale=1.0]{bilder/G_75k.pdf}
	\caption{Darstellung des Leitwerts $G$ in Abhängigkeit der Frequenz $\nu$
	für das \BU-Kabel.}
	\label{fig:G_75k}
\end{figure}

%%%%%%%%%%%%%%%%%%%%%%%%%%%%%%%%%%%%%%%%%%%%%%%%%%%%%%%%%%%%%%%%%%%%%%%%%%%
%                           Dämpfungskonstante                            %
%%%%%%%%%%%%%%%%%%%%%%%%%%%%%%%%%%%%%%%%%%%%%%%%%%%%%%%%%%%%%%%%%%%%%%%%%%%

\clearpage
\subsection{Dämpfungskonstante}
\label{sub:d_mpfungskonstante}

Es soll die Frequenzabhängigkeit der Dämpfungskonstante des langen
\CU-Kabels bestimmt werden. %TODO
Dazu wird ein Signal auf das kurze \CU-Kabel gegeben und die Dämpfung $L_0$
der ersten Oberwelle bestimmt.
Anschließend wird das Signal auf das zu analysierende lange
\CU-Kabel gegeben und ebenfalls wieder die Dämpfung $L_1$ der ersten Oberwelle
bestimmt.
Die Dämpfung wird dabei in \si{dBm} gemessen. Der Pegel ist damit durch
\begin{equation}
	L = 10 \log(\frac{P}{1\si{\milli\watt}})
\end{equation}
gegeben. Daher kann die Dämpfung in \si{\dB} des \CU-Kabel durch
\begin{equation}
	A = L_1 - L_0
\end{equation}
bestimmt werden. In Tabelle \ref{tab:Attenuation} sind die gemessenen Pegel und
die berechnete Dämpfung angegeben. Die berechneten Werte sind in
\ref{fig:bilder/alpha} graphisch dargestellt.

\begin{table}[hb]
  \centering
  \begin{tabular}{ccc|c}
    \midrule
    \midrule
    $\nu / \si{\kHz}$ & $L_0 / \si{dBm}$ & $L_1 / \si{dBm}$ &
    $A / \si{\dB}$ \\
    \midrule
    \phantom{0}2\phantom{.} & --8.3             & \phantom{0}--8.9  & --0.6            \\
\phantom{0}4\phantom{.} & --6.1             & \phantom{0}--7.8  & --1.7            \\
\phantom{0}6\phantom{.} & --5.4             & \phantom{0}--7.9  & --2.5            \\
\phantom{0}8\phantom{.} & --4.5             & \phantom{0}--7.5  & --3.0            \\
10\phantom{.}     & --4.6             & \phantom{0}--8.0  & --3.4            \\
12\phantom{.}     & --4.9             & \phantom{0}--8.5  & --3.6            \\
14\phantom{.}     & --4.5             & \phantom{0}--8.8  & --4.3            \\
16\phantom{.}     & --5.7             & \phantom{0}--9.9  & --4.2            \\
18\phantom{.}     & --5.5             & --10.1            & --4.6            \\
20\phantom{.}     & --5.7             & --10.6            & --4.9            \\
22\phantom{.}     & --5.8             & --11.0            & --5.2            \\
24\phantom{.}     & --5.5             & --11.2            & --5.7            \\
26\phantom{.}     & --5.7             & --11.6            & --5.9            \\
28\phantom{.}     & --5.9             & --12.0            & --6.1            \\
30\phantom{.}     & --5.9             & --12.2            & --6.3            \\
32\phantom{.}     & --5.7             & --12.3            & --6.6            \\
34\phantom{.}     & --6.0             & --12.9            & --6.9            \\
36\phantom{.}     & --6.0             & --13.2            & --7.2            \\
38\phantom{.}     & --5.9             & --13.4            & --7.5            \\
40\phantom{.}     & --6.0             & --13.8            & --7.8            \\
42\phantom{.}     & --6.0             & --14.1            & --8.1            \\
44\phantom{.}     & --6.1             & --14.4            & --8.3            \\
46\phantom{.}     & --4.9             & --13.8            & --8.9            \\
48\phantom{.}     & --5.4             & --14.5            & --9.1            \\
50\phantom{.}     & --5.2             & --14.3            & --9.1            \\
    \midrule
    \midrule
  \end{tabular}
  \caption{Gemessene Werte der Dämpfung des kurzen \CU-Kabels $A_0$ und
  des langen \CU-Kabels im Bezug zu \SI{1}{\milli\watt}
  sowie die berechnete Dämpfung A des \CU-Kabels.}
  \label{tab:Attenuation}
\end{table}

\begin{figure}[ht]
	\centering
	\includegraphics[scale=1.0]{bilder/alpha.pdf}
	\caption{Darstellung der berechneten Dämpfungen des langen \CU-Kabels.}
\label{fig:bilder/alpha}
\end{figure}

%%%%%%%%%%%%%%%%%%%%%%%%%%%%%%%%%%%%%%%%%%%%%%%%%%%%%%%%%%%%%%%%%%%%%%%%%%%
%                 Spannungsverlauf offen kurzgeschlossen                  %
%%%%%%%%%%%%%%%%%%%%%%%%%%%%%%%%%%%%%%%%%%%%%%%%%%%%%%%%%%%%%%%%%%%%%%%%%%%

\clearpage
\subsection{Spannungsverlauf bei offenen und kurzgeschlossenem Ende}
\label{sub:spannungsverlauf_bei_offenen_und_kurzgeschlossenem_ende}

% ==================================================
% 	Kabellänge aus Laufzeitmessung
% ==================================================

\subsubsection{Bestimmung der Kabellänge durch Laufzeitmessung}
\label{ssub:bestimmung_der_kabell_nge_durch_laufzeitmessung}

Hier wird die Länge der drei Kabel durch Betrachtung der Spannungsverläufe am
Anfang des Kabels bei offenen und kurzgeschlossenem Ende bestimmt. Hieran
lassen sich die Zeiten $t_1$, den Beginn des einlaufenden Impulses, und $t_2$,
den Beginn des reflektierten Impulses, ablesen. Die Länge der Kabel lassen sich
schließlich mit%
\begin{equation}
  l = \frac{v \Delta t}{2}, \qquad \Delta t = t_2 - t_1, \quad \text{und} \quad
  v = \frac{1}{\sqrt{LC}} = \frac{\text{c}}{\sqrt{\epsilon_r}}
  \label{eq:laenge}
\end{equation}
bestimmen, wobei $v$ die Ausbreitungsgeschwindigkeit,
$\text{c}$ die Lichtgeschwindigkeit und $\epsilon_r = 2.25$
die Dielektrizitätskonstante des Dielektrikums ist.
Die aufgenommenen Oszillosgraphenbilder sind in de
Abbildungen~\ref{fig:oszi_50k_offen} bis~\ref{fig:oszi_50l_kurz} dargestellt.
Im Anhang sind diese Bilder mit abgelesenen Zeiten zu finden.
Als Fehler der Zeiten wird die Ablesegenauigkeit in den Bildern
genommen. Die Werte befinden sich in Tabelle~\ref{tab:Zeiten}.
Die hiermit und mit der Gleichungen~\eqref{eq:laenge} berechneten Längen
befinden sich in Tabelle~\ref{tab:Laengen}.

\begin{table}[hb]
  \centering
  \begin{tabular}{lcccc}
    \midrule
    \midrule
    & \multicolumn{2}{c}{offenes Ende} &
    \multicolumn{2}{c}{kurzgeschlossenes Ende} \\
    \cmidrule(lr{0.75em}){2-5}
    % \cline{2-5}
    & $t_1 / \si{\nano\second}$ & $t_2 / \si{\nano\second}$    &
    $t_1 / \si{\nano\second}$ & $t_2 / \si{\nano\second}$ \\
    \midrule
    \CU, kurz         & 104 \pm 5         & 307 \pm 5         & 105 \pm 5         & 307 \pm 5        \\
\BU               & 105 \pm 5         & 308 \pm 5         & 104 \pm 5         & 306 \pm 5        \\
\CU, lang         & 530 \pm 50        & 1400 \pm 50       & 530 \pm 50        & 1400 \pm 50      \\
    \midrule
    \midrule
  \end{tabular}
  \caption{Darstellung der abgelesenen Zeiten $t_1$ und $t_2$ der
  jeweiligen Kabel.}
  \vspace{2em}
  \label{tab:Zeiten}
\end{table}

\begin{table}[h]
  \centering
  \begin{tabular}{lcc}
    \midrule
    \midrule
    & \multicolumn{2}{c}{$l / \si{\meter}$} \\
    \cmidrule(lr{0.75em}){2-3}
    & offenes Ende & kurzgeschlossenes Ende \\
    \midrule
    $\SI{50}{\ohm}$, kurz & 105.0 \pm 5.0     & 308.0 \pm 5.0     & 104.0 \pm 5.0     & 306.0 \pm 5.0    \\
$\SI{75}{\ohm}$, kurz & 104.0 \pm 5.0     & 307.0 \pm 5.0     & 105.0 \pm 5.0     & 307.0 \pm 5.0    \\
$\SI{50}{\ohm}$, lang & 530.0 \pm 50.0    & 1400 \pm 50       & 530.0 \pm 50.0    & 1400 \pm 50      \\
    \midrule
    \midrule
  \end{tabular}
  \caption{Darstellung der berechneten Längen $l$ der jeweiligen Kabel mit
    offenem und kurzgeschlossenem Ende.}
  \label{tab:Laengen}
\end{table}

\begin{figure}[ht]
  \centering
  \includegraphics[scale=0.5]{bilder/reflexion/F0000TEK.JPG}
  \caption{Mit dem Oszilloskop aufgenommenes Bild des kurzen \CU-Kabels mit
  offenem Ende.}
  \label{fig:oszi_50k_offen}
  \vspace{2em}
  \includegraphics[scale=0.5]{bilder/reflexion/F0001TEK.JPG}
  \caption{Mit dem Oszilloskop aufgenommenes Bild des kurzen \CU-Kabels mit
  kurzgeschlossenem Ende.}
  \label{fig:oszi_50k_kurz}
\end{figure}
\begin{figure}[ht]
  \centering
  \includegraphics[scale=0.5]{bilder/reflexion/F0002TEK.JPG}
  \caption{Mit dem Oszilloskop aufgenommenes Bild des kurzen \BU-Kabels mit
  offenem Ende.}
  \label{fig:oszi_75k_offen}
  \vspace{2em}
  \includegraphics[scale=0.5]{bilder/reflexion/F0003TEK.JPG}
  \caption{Mit dem Oszilloskop aufgenommenes Bild des kurzen \BU-Kabels mit
  kurzgeschlossenem Ende.}
  \label{fig:oszi_75k_kurz}
\end{figure}
\begin{figure}[ht]
  \centering
  \includegraphics[scale=0.5]{bilder/reflexion/F0004TEK.JPG}
  \caption{Mit dem Oszilloskop aufgenommenes Bild des langen \CU-Kabels mit
  offenem Ende.}
  \label{fig:oszi_50l_offen}
  \vspace{2em}
  \includegraphics[scale=0.5]{bilder/reflexion/F0005TEK.JPG}
  \caption{Mit dem Oszilloskop aufgenommenes Bild des langen \CU-Kabels mit
  kurzgeschlossenem Ende.}
  \label{fig:oszi_50l_kurz}
\end{figure}

% ==================================================
% 	Leitungskonstanten
% ==================================================

\clearpage
\subsubsection{Bestimmung der Leitungskonstanten}
\label{ssub:bestimmung_der_leitungskonstanten}

In diesem und den folgenden Unterkapiteln werden Daten aus den
Abbildungen~\ref{fig:oszi_50l_offen} und~\ref{fig:oszi_50l_kurz} über den
ganzen $t$-Bereich oder nur bestimmte Teilbereiche mit Hilfe des Programms
\texttt{Engauge} digitalisiert, um eine bessere Auswertung der Daten zu
ermöglichen. Des weiteren ist die Bestimmung der Leitungskonstante nur bei
dem langen \CU-Kabel möglich, da nur hier der kapazitive und induktive Verlauf
der Spannung zu sehen ist. Im Folgenden werden daher nun die Leitungskonstanten
des langen \CU-Kabels bestimmt.

\paragraph{Der Widerstandsbelag}
\label{par:der_widerstandsbelag}

Hier wird das Oszillosgraphenbild~\ref{fig:oszi_50l_kurz} digitalisiert.
Die erhaltenen Daten befinden sich in Tabelle~\ref{tab:R_Daten} und werden
in Abbildung~\ref{fig:R_Daten} graphisch dargestellt.
Mit Hilfe der Beziehung
\begin{equation}
  U_1 (1 + \Gamma) = U_0 \quad \Rightarrow \quad \Gamma = \frac{U_0}{U_1} - 1
  \label{eq:Gamma_R}
\end{equation}
kann der Reflektionsfaktor $\Gamma$ bestimmt werden. Einsetzen des Ergebnisses
in
\begin{equation}
  \Gamma = \frac{R-Z_0}{R+Z_0} \quad \Rightarrow \quad
  R = - \frac{Z_0(\Gamma + 1)}{\Gamma - 1}
  \label{eq:R}
\end{equation}
liefert schließlich den gewünschten Widerstandsbelag $R$.
Hierbei gilt für den Wellenwiderstand $Z_0 = \SI{50}{\ohm}$.
Es gilt also $U_0$ und $U_1$ zu bestimmen, wobei $U_0$ die Höhe des Pulses aus
Abbildung~\ref{fig:R_Daten} ist und $U_1$ die Höhe der Spannung nach Abfall des
Pulses. Um die Spannungen $U_1$ und $U_0$ bestimmen zu können, muss zudem eine
Offsetspannung $U_\text{off}$ ermittelt werden.
Die eben genannten Spannungen werden entsprechend
\begin{align}
  \begin{aligned}
    U_\text{off} &= \overline{U(t)}\,, \\
    U_1          &= \overline{U(t)}\,, \\
    U_0          &= \overline{U(t)}\,,
  \end{aligned}
  \qquad
  \begin{aligned}
    &t~< \SI{500}{\nano\second} \\
    \SI{750}{\nano\second}  <~&t~< \SI{1360}{\nano\second} \\
    \SI{2000}{\nano\second} <~&t~< \SI{2250}{\nano\second}
  \end{aligned}
\end{align}
bestimmt, wobei $U(t)$ und $t$ aus Tabelle~\ref{tab:R_Daten} bzw.
Abbildung~\ref{fig:R_Daten} entnommen werden.
Die so berechneten Werte lauten
\begin{align}
  U_\text{off} &= \SI[parse-numbers = false]{14.80 \pm 0.13}{\volt} \\
  U_0 &= \SI[parse-numbers = false]{13.7 \pm 1.0}{\volt} \\
  U_1 &= \SI[parse-numbers = false]{27.95 \pm 0.09}{\volt}~.
\end{align}
Damit berechnet sich der Widerstandsbelag nach Gleichung~\eqref{eq:Gamma_R} und
\eqref{eq:R} zu
\begin{equation}
  \underline{R = \phantom{0}0\phantom{.} & \phantom{0}30.1   & 38\phantom{.}     & 119.3             & \phantom{0}76\phantom{.} & 212.4             & 114\phantom{.}    & 309.3            \\
\phantom{0}1\phantom{.} & \phantom{0}32.4   & 39\phantom{.}     & 121.7             & \phantom{0}77\phantom{.} & 214.9             & 115\phantom{.}    & 311.9            \\
\phantom{0}2\phantom{.} & \phantom{0}34.7   & 40\phantom{.}     & 124.1             & \phantom{0}78\phantom{.} & 217.4             & 116\phantom{.}    & 314.5            \\
\phantom{0}3\phantom{.} & \phantom{0}37.0   & 41\phantom{.}     & 126.5             & \phantom{0}79\phantom{.} & 219.9             & 117\phantom{.}    & 317.1            \\
\phantom{0}4\phantom{.} & \phantom{0}39.3   & 42\phantom{.}     & 128.9             & \phantom{0}80\phantom{.} & 222.4             & 118\phantom{.}    & 319.7            \\
\phantom{0}5\phantom{.} & \phantom{0}41.6   & 43\phantom{.}     & 131.3             & \phantom{0}81\phantom{.} & 224.9             & 119\phantom{.}    & 322.3            \\
\phantom{0}6\phantom{.} & \phantom{0}44.0   & 44\phantom{.}     & 133.7             & \phantom{0}82\phantom{.} & 227.4             & 120\phantom{.}    & 324.9            \\
\phantom{0}7\phantom{.} & \phantom{0}46.3   & 45\phantom{.}     & 136.2             & \phantom{0}83\phantom{.} & 229.9             & 121\phantom{.}    & 327.6            \\
\phantom{0}8\phantom{.} & \phantom{0}48.6   & 46\phantom{.}     & 138.6             & \phantom{0}84\phantom{.} & 232.4             & 122\phantom{.}    & 330.2            \\
\phantom{0}9\phantom{.} & \phantom{0}50.9   & 47\phantom{.}     & 141.0             & \phantom{0}85\phantom{.} & 235.0             & 123\phantom{.}    & 332.8            \\
10\phantom{.}     & \phantom{0}53.2   & 48\phantom{.}     & 143.4             & \phantom{0}86\phantom{.} & 237.5             & 124\phantom{.}    & 335.4            \\
11\phantom{.}     & \phantom{0}55.5   & 49\phantom{.}     & 145.9             & \phantom{0}87\phantom{.} & 240.0             & 125\phantom{.}    & 338.1            \\
12\phantom{.}     & \phantom{0}57.9   & 50\phantom{.}     & 148.3             & \phantom{0}88\phantom{.} & 242.6             & 126\phantom{.}    & 340.7            \\
13\phantom{.}     & \phantom{0}60.2   & 51\phantom{.}     & 150.7             & \phantom{0}89\phantom{.} & 245.1             & 127\phantom{.}    & 343.3            \\
14\phantom{.}     & \phantom{0}62.5   & 52\phantom{.}     & 153.1             & \phantom{0}90\phantom{.} & 247.6             & 128\phantom{.}    & 346.0            \\
15\phantom{.}     & \phantom{0}64.9   & 53\phantom{.}     & 155.6             & \phantom{0}91\phantom{.} & 250.2             & 129\phantom{.}    & 348.6            \\
16\phantom{.}     & \phantom{0}67.2   & 54\phantom{.}     & 158.0             & \phantom{0}92\phantom{.} & 252.7             & 130\phantom{.}    & 351.3            \\
17\phantom{.}     & \phantom{0}69.5   & 55\phantom{.}     & 160.5             & \phantom{0}93\phantom{.} & 255.2             & 131\phantom{.}    & 353.9            \\
18\phantom{.}     & \phantom{0}71.9   & 56\phantom{.}     & 162.9             & \phantom{0}94\phantom{.} & 257.8             & 132\phantom{.}    & 356.6            \\
19\phantom{.}     & \phantom{0}74.2   & 57\phantom{.}     & 165.4             & \phantom{0}95\phantom{.} & 260.3             & 133\phantom{.}    & 359.2            \\
20\phantom{.}     & \phantom{0}76.6   & 58\phantom{.}     & 167.8             & \phantom{0}96\phantom{.} & 262.9             & 134\phantom{.}    & 361.9            \\
21\phantom{.}     & \phantom{0}78.9   & 59\phantom{.}     & 170.3             & \phantom{0}97\phantom{.} & 265.5             & 135\phantom{.}    & 364.5            \\
22\phantom{.}     & \phantom{0}81.3   & 60\phantom{.}     & 172.7             & \phantom{0}98\phantom{.} & 268.0             & 136\phantom{.}    & 367.2            \\
23\phantom{.}     & \phantom{0}83.6   & 61\phantom{.}     & 175.2             & \phantom{0}99\phantom{.} & 270.6             & 137\phantom{.}    & 369.8            \\
24\phantom{.}     & \phantom{0}86.0   & 62\phantom{.}     & 177.6             & 100\phantom{.}    & 273.1             & 138\phantom{.}    & 372.5            \\
25\phantom{.}     & \phantom{0}88.4   & 63\phantom{.}     & 180.1             & 101\phantom{.}    & 275.7             & 139\phantom{.}    & 375.2            \\
26\phantom{.}     & \phantom{0}90.7   & 64\phantom{.}     & 182.6             & 102\phantom{.}    & 278.3             & 140\phantom{.}    & 377.8            \\
27\phantom{.}     & \phantom{0}93.1   & 65\phantom{.}     & 185.0             & 103\phantom{.}    & 280.8             & 141\phantom{.}    & 380.5            \\
28\phantom{.}     & \phantom{0}95.5   & 66\phantom{.}     & 187.5             & 104\phantom{.}    & 283.4             & 142\phantom{.}    & 383.2            \\
29\phantom{.}     & \phantom{0}97.8   & 67\phantom{.}     & 190.0             & 105\phantom{.}    & 286.0             & 143\phantom{.}    & 385.9            \\
30\phantom{.}     & 100.2             & 68\phantom{.}     & 192.5             & 106\phantom{.}    & 288.6             & 144\phantom{.}    & 388.5            \\
31\phantom{.}     & 102.6             & 69\phantom{.}     & 194.9             & 107\phantom{.}    & 291.1             & 145\phantom{.}    & 391.2            \\
32\phantom{.}     & 105.0             & 70\phantom{.}     & 197.4             & 108\phantom{.}    & 293.7             & 146\phantom{.}    & 393.9            \\
33\phantom{.}     & 107.4             & 71\phantom{.}     & 199.9             & 109\phantom{.}    & 296.3             & 147\phantom{.}    & 396.6            \\
34\phantom{.}     & 109.7             & 72\phantom{.}     & 202.4             & 110\phantom{.}    & 298.9             & 148\phantom{.}    & 399.3            \\
35\phantom{.}     & 112.1             & 73\phantom{.}     & 204.9             & 111\phantom{.}    & 301.5             & 149\phantom{.}    & 402.0            \\
36\phantom{.}     & 114.5             & 74\phantom{.}     & 207.4             & 112\phantom{.}    & 304.1             & 150\phantom{.}    & 404.7            \\
37\phantom{.}     & 116.9             & 75\phantom{.}     & 209.9             & 113\phantom{.}    & 306.7             & 151\phantom{.}    & 407.4            \\}~.
\end{equation}

% Tabellen und Bilder
\begin{table}[hb]
  \centering
  \begin{tabular}{cc|cc|cc}
    \midrule
    \midrule
    $t / \si{\nano\second}$ & $U / \si{\milli\volt}$ &
    $t / \si{\nano\second}$ & $U / \si{\milli\volt}$ &
    $t / \si{\nano\second}$ & $U / \si{\milli\volt}$ \\
    \midrule
    \phantom{0}0\phantom{.} & \phantom{0}30.1   & 38\phantom{.}     & 119.3             & \phantom{0}76\phantom{.} & 212.4             & 114\phantom{.}    & 309.3            \\
\phantom{0}1\phantom{.} & \phantom{0}32.4   & 39\phantom{.}     & 121.7             & \phantom{0}77\phantom{.} & 214.9             & 115\phantom{.}    & 311.9            \\
\phantom{0}2\phantom{.} & \phantom{0}34.7   & 40\phantom{.}     & 124.1             & \phantom{0}78\phantom{.} & 217.4             & 116\phantom{.}    & 314.5            \\
\phantom{0}3\phantom{.} & \phantom{0}37.0   & 41\phantom{.}     & 126.5             & \phantom{0}79\phantom{.} & 219.9             & 117\phantom{.}    & 317.1            \\
\phantom{0}4\phantom{.} & \phantom{0}39.3   & 42\phantom{.}     & 128.9             & \phantom{0}80\phantom{.} & 222.4             & 118\phantom{.}    & 319.7            \\
\phantom{0}5\phantom{.} & \phantom{0}41.6   & 43\phantom{.}     & 131.3             & \phantom{0}81\phantom{.} & 224.9             & 119\phantom{.}    & 322.3            \\
\phantom{0}6\phantom{.} & \phantom{0}44.0   & 44\phantom{.}     & 133.7             & \phantom{0}82\phantom{.} & 227.4             & 120\phantom{.}    & 324.9            \\
\phantom{0}7\phantom{.} & \phantom{0}46.3   & 45\phantom{.}     & 136.2             & \phantom{0}83\phantom{.} & 229.9             & 121\phantom{.}    & 327.6            \\
\phantom{0}8\phantom{.} & \phantom{0}48.6   & 46\phantom{.}     & 138.6             & \phantom{0}84\phantom{.} & 232.4             & 122\phantom{.}    & 330.2            \\
\phantom{0}9\phantom{.} & \phantom{0}50.9   & 47\phantom{.}     & 141.0             & \phantom{0}85\phantom{.} & 235.0             & 123\phantom{.}    & 332.8            \\
10\phantom{.}     & \phantom{0}53.2   & 48\phantom{.}     & 143.4             & \phantom{0}86\phantom{.} & 237.5             & 124\phantom{.}    & 335.4            \\
11\phantom{.}     & \phantom{0}55.5   & 49\phantom{.}     & 145.9             & \phantom{0}87\phantom{.} & 240.0             & 125\phantom{.}    & 338.1            \\
12\phantom{.}     & \phantom{0}57.9   & 50\phantom{.}     & 148.3             & \phantom{0}88\phantom{.} & 242.6             & 126\phantom{.}    & 340.7            \\
13\phantom{.}     & \phantom{0}60.2   & 51\phantom{.}     & 150.7             & \phantom{0}89\phantom{.} & 245.1             & 127\phantom{.}    & 343.3            \\
14\phantom{.}     & \phantom{0}62.5   & 52\phantom{.}     & 153.1             & \phantom{0}90\phantom{.} & 247.6             & 128\phantom{.}    & 346.0            \\
15\phantom{.}     & \phantom{0}64.9   & 53\phantom{.}     & 155.6             & \phantom{0}91\phantom{.} & 250.2             & 129\phantom{.}    & 348.6            \\
16\phantom{.}     & \phantom{0}67.2   & 54\phantom{.}     & 158.0             & \phantom{0}92\phantom{.} & 252.7             & 130\phantom{.}    & 351.3            \\
17\phantom{.}     & \phantom{0}69.5   & 55\phantom{.}     & 160.5             & \phantom{0}93\phantom{.} & 255.2             & 131\phantom{.}    & 353.9            \\
18\phantom{.}     & \phantom{0}71.9   & 56\phantom{.}     & 162.9             & \phantom{0}94\phantom{.} & 257.8             & 132\phantom{.}    & 356.6            \\
19\phantom{.}     & \phantom{0}74.2   & 57\phantom{.}     & 165.4             & \phantom{0}95\phantom{.} & 260.3             & 133\phantom{.}    & 359.2            \\
20\phantom{.}     & \phantom{0}76.6   & 58\phantom{.}     & 167.8             & \phantom{0}96\phantom{.} & 262.9             & 134\phantom{.}    & 361.9            \\
21\phantom{.}     & \phantom{0}78.9   & 59\phantom{.}     & 170.3             & \phantom{0}97\phantom{.} & 265.5             & 135\phantom{.}    & 364.5            \\
22\phantom{.}     & \phantom{0}81.3   & 60\phantom{.}     & 172.7             & \phantom{0}98\phantom{.} & 268.0             & 136\phantom{.}    & 367.2            \\
23\phantom{.}     & \phantom{0}83.6   & 61\phantom{.}     & 175.2             & \phantom{0}99\phantom{.} & 270.6             & 137\phantom{.}    & 369.8            \\
24\phantom{.}     & \phantom{0}86.0   & 62\phantom{.}     & 177.6             & 100\phantom{.}    & 273.1             & 138\phantom{.}    & 372.5            \\
25\phantom{.}     & \phantom{0}88.4   & 63\phantom{.}     & 180.1             & 101\phantom{.}    & 275.7             & 139\phantom{.}    & 375.2            \\
26\phantom{.}     & \phantom{0}90.7   & 64\phantom{.}     & 182.6             & 102\phantom{.}    & 278.3             & 140\phantom{.}    & 377.8            \\
27\phantom{.}     & \phantom{0}93.1   & 65\phantom{.}     & 185.0             & 103\phantom{.}    & 280.8             & 141\phantom{.}    & 380.5            \\
28\phantom{.}     & \phantom{0}95.5   & 66\phantom{.}     & 187.5             & 104\phantom{.}    & 283.4             & 142\phantom{.}    & 383.2            \\
29\phantom{.}     & \phantom{0}97.8   & 67\phantom{.}     & 190.0             & 105\phantom{.}    & 286.0             & 143\phantom{.}    & 385.9            \\
30\phantom{.}     & 100.2             & 68\phantom{.}     & 192.5             & 106\phantom{.}    & 288.6             & 144\phantom{.}    & 388.5            \\
31\phantom{.}     & 102.6             & 69\phantom{.}     & 194.9             & 107\phantom{.}    & 291.1             & 145\phantom{.}    & 391.2            \\
32\phantom{.}     & 105.0             & 70\phantom{.}     & 197.4             & 108\phantom{.}    & 293.7             & 146\phantom{.}    & 393.9            \\
33\phantom{.}     & 107.4             & 71\phantom{.}     & 199.9             & 109\phantom{.}    & 296.3             & 147\phantom{.}    & 396.6            \\
34\phantom{.}     & 109.7             & 72\phantom{.}     & 202.4             & 110\phantom{.}    & 298.9             & 148\phantom{.}    & 399.3            \\
35\phantom{.}     & 112.1             & 73\phantom{.}     & 204.9             & 111\phantom{.}    & 301.5             & 149\phantom{.}    & 402.0            \\
36\phantom{.}     & 114.5             & 74\phantom{.}     & 207.4             & 112\phantom{.}    & 304.1             & 150\phantom{.}    & 404.7            \\
37\phantom{.}     & 116.9             & 75\phantom{.}     & 209.9             & 113\phantom{.}    & 306.7             & 151\phantom{.}    & 407.4            \\
    \midrule
    \midrule
  \end{tabular}
  \caption{Datstellung der digitalisierten Daten des
    Oszillosgraphenbild~\ref{fig:oszi_50l_kurz}.}
  \label{tab:R_Daten}
\end{table}
\begin{figure}[hb]
  \centering
  \includegraphics[scale=1.0]{bilder/R.pdf}
  \caption{Graphische Darstellung der digitalisierten Daten des
    Oszillosgraphenbild~\ref{fig:oszi_50l_kurz}.}
\label{fig:R_Daten}
\end{figure}

\clearpage
\paragraph{Der induktive Belag}
\label{par:der_induktive_belag}

Ein kurzgeschlossenes Koaxialkabel zeigt ein rein induktives Verhalten, daher
wird zur Bestimmung des induktiven Belags ebenfalls das
Oszillosgraphenbild~\ref{fig:oszi_50l_kurz} mit den entsprechend
digitalisierten Daten aus Tabelle~\ref{tab:R_Daten} verwandt.
Jedoch werden diese Daten auf den für den Induktivitätsbelag relevanten Teil
eingeschränkt, sodass nur die Werte $U(t)$ aus
\begin{equation}
  \SI{1430}{\nano\second} <~t~< \SI{2250}{\nano\second}
\end{equation}
betrachtet werden. Diese Werte sind in Abbildung~\ref{fig:induktivitaetsbelag}
graphisch dargestellt.
Die Abfallende Flanke des Pulses entspricht nun einem Ausschaltvorgang
einer Spule entsprechend
\begin{equation}
  U(t) = U_0 \exp(-\frac{t}{\tau}) + U_\text{off} \quad \text{mit} \quad
  \tau = \frac{L}{R+Z_0}~,
  \label{eq:L_abschalt}
\end{equation}
wobei $R$ den oben bestimmten Widerstandsbelag und $U_\text{off}$ die
Offsetspannung zu \SI{0}{\volt} darstellt. Durch Umstellen und Logarithmierung
der Gleichung \eqref{eq:L_abschalt} ergibt sich eine Geradengleichung der Form
\begin{equation}
  \ln(U - U_\text{off}) = -\frac{t}{\tau} + \ln U_0 \coloneqq
  m \cdot t + \ln U_0 \quad \text{mit} \quad m = - \frac{1}{\tau}
\end{equation}
womit schließlich eine lineare Regression durchgeführt und aus der
Steigung der Induktivitätsbelag bestimmt werden kann.
Die Offsetspannung wird entsprechend%
\begin{equation}
  U_\text{off} = \overline{U(t)}\,,
  \quad \SI{2100}{\nano\second} <~t~< \SI{2250}{\nano\second}
\end{equation}
zu%
\begin{equation}
  U_\text{off} = \SI{43.20}{\milli\volt}
\end{equation}
bestimmt.
In Abbildung~\ref{fig:induktivitaetsbelag_fit} sind nun die Daten
${\ln(U - U_\text{off})}$ sowie die dazugehörige lineare Regression eingetragen.
Die entsprechenden Daten befinden sich in Tabelle~\ref{tab:L_Daten}.
Der Fit ergibt dabei die Gleichung
\begin{equation}
  \sisetup{per-mode=reciprocal-positive-first}
  G_L(t) = \SI[parse-numbers = false]{-0.0053 \pm 0.0004}{\per\nano\second}\, \cdot \,t\, + \SI[parse-numbers = false]{11.6 \pm 0.7}{}~.
  \label{eq:L_fit}
\end{equation}
Aus der Steigung von~\ref{eq:L_fit} wird nun mit Gleichung~\eqref{eq:L_abschalt} der
Induktivitätsbelag zu
\begin{equation}
  \underline{L = \SI[parse-numbers = false]{226.0 \pm 4.2}{\milli\henry}}
  \label{eq:indukivitaetsbelag}
\end{equation}
bestimmt.

\begin{table}[htpb]
  \centering
  \begin{tabular}{cccc}
    \midrule
    \midrule
    $t / \si{\nano\second}$ & $U / \si{\milli\volt}$ &
    $U - U_\text{off} / \si{\milli\volt}$ & $\ln(U - U_\text{off})$ \\
    \midrule
    \SI[parse-numbers = false]{226.0 \pm 4.2}{\milli\henry}
    \midrule
    \midrule
  \end{tabular}
  \caption{Darstellung der für den Induktivitätsbelag relevanten Daten sowie
  die berechneten Werte für den Fit.}
  \label{tab:L_Daten}
\end{table}
\begin{figure}[htpb]
  \centering
  \includegraphics[scale=1.0]{bilder/L.pdf}
  \caption{Darstellung der für die zur Bestimmung des Induktivitätsbelags
    relevanten digitalisierten Daten von Abbildung~\ref{fig:oszi_50l_kurz}.}
  \label{fig:induktivitaetsbelag}
  \includegraphics[scale=1.0]{bilder/L_fit.pdf}
  \caption{Darstellung der für die zur Bestimmung des Induktivitätsbelags
    relevanten digitalisierten Daten von Abbildung~\ref{fig:oszi_50l_kurz}.}
  \label{fig:induktivitaetsbelag_fit}
\end{figure}

\clearpage
\paragraph{Der Kapazitätsbelag}
\label{par:der_kapazit_tsbelag}

Entgegen dem Induktivitätsbelag zeigt ein Koaxialkabel mit offenem Ende ein
rein kapazitives Verhalten. Somit wird zur Bestimmung des Kapazitätsbelags das
Oszillosgraphenbild~\ref{fig:oszi_50l_offen} verwandt. Die entsprechenden
digitalisierten Daten sind in Tabelle~\ref{tab:C_roh_Daten} und
Abbildung~\ref{fig:C_roh_Daten} dargestellt.
Die Einschränkung der relevanten Daten bezieht sich hier auf $U(t)$ aus
\begin{equation}
  \quad \SI{1370}{\nano\second} <~t~< \SI{1650}{\nano\second}~.
\end{equation}
Den zu betrachtenden Verlauf entspricht hier einer Aufladekurve eines
Kondensators gemäß
\begin{equation}
  U = U_0\qty(1 - \exp(-\frac{t}{\tau})) \quad \text{mit} \quad
  \tau = RC~,
  \label{eq:Aufladekurve_C}
\end{equation}
wobei $R$ wiederum der obiger Widerstandsbelag ist. Diese Gleichung kann wieder
linearisiert werden zu
\begin{equation}
  \ln(U_0 - U) = - \frac{t}{\tau} + \ln(U_0)~,
\end{equation}
woraus aus der Steigung der Kapazitätsbelag bestimmt werden kann.
Die Sättigungsspannung $U_0$ wird dabei entsprechend
\begin{equation}
  U_0 = \overline{U(t)}\,,
  \quad \SI{2000}{\nano\second} <~t
\end{equation}
zu
\begin{equation}
  U_0 = \SI{312.45}{\milli\volt}
\end{equation}
bestimmt. Die für den Fit relevanten Daten sind in Tabelle~\ref{tab:C_fit}
dargestellt. Der Fit befindet sich in Abbildung~\ref{fig:C_fit}.
Die aus dem Fit erhaltene Gleichung lautet
\begin{equation}
  \sisetup{per-mode=reciprocal-positive-first}
  G_C(t) = \SI[parse-numbers = false]{-0.0100 \pm 0.0006}{\per\nano\second}\, \cdot \,t\, + \SI[parse-numbers = false]{18.0 \pm 0.8}{}~.
  \label{eq:C_fit}
\end{equation}
Aus der Steigung kann nun der Kapazitätsbelag mit Hilfe von
Gleichung~\eqref{eq:Aufladekurve_C} zu
\begin{equation}
  \underline{C = \SI[parse-numbers = false]{2.01 \pm 0.11}{\nano\farad}}
\end{equation}
bestimmt werden.

% Tabellen und Plots
\begin{table}[htpb]
  \centering
  \begin{tabular}{cc|cc|cc}
    \midrule
    \midrule
    $t / \si{\nano\second}$ & $U / \si{\milli\volt}$ &
    $t / \si{\nano\second}$ & $U / \si{\milli\volt}$ &
    $t / \si{\nano\second}$ & $U / \si{\milli\volt}$ \\
    \midrule
    --40\phantom{.}   & \phantom{0}76.824 & \phantom{0}865\phantom{.} & 200.154           & 1560\phantom{.}   & 301.622          \\
\phantom{0}10\phantom{.} & \phantom{0}75.806 & \phantom{0}910\phantom{.} & 201.133           & 1600\phantom{.}   & 303.601          \\
\phantom{0}60\phantom{.} & \phantom{0}76.784 & \phantom{0}955\phantom{.} & 202.113           & 1645\phantom{.}   & 305.578          \\
105\phantom{.}    & \phantom{0}76.766 & \phantom{0}995\phantom{.} & 201.099           & 1690\phantom{.}   & 305.560          \\
150\phantom{.}    & \phantom{0}75.751 & 1030\phantom{.}   & 200.088           & 1735\phantom{.}   & 305.542          \\
190\phantom{.}    & \phantom{0}77.730 & 1070\phantom{.}   & 201.069           & 1775\phantom{.}   & 308.519          \\
235\phantom{.}    & \phantom{0}75.717 & 1115\phantom{.}   & 201.051           & 1815\phantom{.}   & 308.503          \\
285\phantom{.}    & \phantom{0}75.697 & 1160\phantom{.}   & 202.031           & 1860\phantom{.}   & 310.480          \\
330\phantom{.}    & \phantom{0}75.679 & 1210\phantom{.}   & 203.008           & 1910\phantom{.}   & 310.460          \\
375\phantom{.}    & \phantom{0}75.661 & 1255\phantom{.}   & 202.991           & 1955\phantom{.}   & 312.437          \\
415\phantom{.}    & \phantom{0}76.642 & 1295\phantom{.}   & 201.977           & 1995\phantom{.}   & 310.426          \\
455\phantom{.}    & \phantom{0}75.629 & 1340\phantom{.}   & 202.957           & 2030\phantom{.}   & 311.409          \\
500\phantom{.}    & \phantom{0}75.611 & 1365\phantom{.}   & 206.937           & 2070\phantom{.}   & 311.394          \\
515\phantom{.}    & \phantom{0}83.585 & 1375\phantom{.}   & 215.910           & 2120\phantom{.}   & 313.369          \\
520\phantom{.}    & \phantom{0}93.558 & 1380\phantom{.}   & 225.883           & 2160\phantom{.}   & 311.358          \\
525\phantom{.}    & 112.509           & 1385\phantom{.}   & 234.859           & 2200\phantom{.}   & 312.339          \\
530\phantom{.}    & 142.432           & 1390\phantom{.}   & 244.832           & 2240\phantom{.}   & 313.321          \\
535\phantom{.}    & 161.383           & 1395\phantom{.}   & 253.807           & 2280\phantom{.}   & 313.305          \\
540\phantom{.}    & 181.331           & 1405\phantom{.}   & 262.781           & 2325\phantom{.}   & 311.292          \\
565\phantom{.}    & 196.283           & 1420\phantom{.}   & 271.753           & 2370\phantom{.}   & 310.276          \\
600\phantom{.}    & 197.267           & 1435\phantom{.}   & 279.727           & 2405\phantom{.}   & 311.260          \\
640\phantom{.}    & 196.253           & 1465\phantom{.}   & 286.697           & 2450\phantom{.}   & 310.244          \\
680\phantom{.}    & 199.230           & 1490\phantom{.}   & 292.672           & -                 & -                \\
720\phantom{.}    & 199.214           & 1525\phantom{.}   & 296.648           & -                 & -                \\
    \midrule
    \midrule
  \end{tabular}
  \caption{Darstellung der digitalisierten Daten aus dem
    Oszillosgraphenbild~\ref{fig:oszi_50l_offen}.}
\label{tab:C_roh_Daten}
\end{table}

\begin{table}[htpb]
  \centering
  \begin{tabular}{cccc}
    \midrule
    \midrule
    $t / \si{\nano\second}$ & $U / \si{\milli\volt}$ &
    $U_0 - U / \si{\milli\volt}$ & $\ln(U_0 - U) / \si{\milli\volt}$ \\
    \midrule
    G_C(t) = \SI[parse-numbers = false]{-0.0100 \pm 0.0006}{\per\nano\second}\, \cdot \,t\, + \SI[parse-numbers = false]{18.0 \pm 0.8}{}
    \midrule
    \midrule
  \end{tabular}
  \caption{Darstellung der für den Fit relevanten Daten.}
\label{tab:C_fit}
\end{table}

\begin{figure}[htpb]
  \centering
  \includegraphics[scale=1.0]{bilder/C.pdf}
  \caption{Graphische Darstellung der digitalisierten Daten des langen
    \CU-Kabels mit offenem Ende.}
\label{fig:C_roh_Daten}
  \includegraphics[scale=1.0]{bilder/C_fit.pdf}
  \caption{Graphische Darstellung der digitalisierten Daten des langen
    \CU-Kabels mit offenem Ende.}
\label{fig:C_fit}
\end{figure}

% ==================================================
% 	Schmitt-Diagramm
% ==================================================

\clearpage
\subsubsection{Bestimmung der Kabellänge über ein Smith-Diagramm}
\label{ssub:bestimmung_der_kabell_nge_ber_ein_smith_diagramm}

Nun soll die Länge des langen \BU-Kabels (die Leitungskonstanten der anderen
Kabel konnten wie oben beschrieben nicht bestimmt werden) mit Hilfe eines
Smith-Diagramms bestimmt werden. Die vollständige Umdrehung der
Smith-Diagramms entspricht einer halben Wellenlänge $\lambda$, sodass die
Länge des Kabels durch
\begin{equation}
  l = \frac{\lambda}{2} \cdot \frac{\varphi}{2 \pi}
\end{equation}
angegeben werden kann, wobei $\varphi$ die Phasenverschiebung nach der Reflexion
am Kabelende beschreibt. Die Wellenlänge lässt sich zudem durch
\begin{equation}
  \lambda = \frac{c}{f \sqrt{\epsilon_r}}
\end{equation}
ausdrücken.
Das heißt, gesucht ist zunächst die Phasenverschiebung $\varphi$.
Dazu wird nun der kurzgeschlossene Fall betrachtet.
Die Impedanz des Kabels beträgt hier
\begin{equation}
  Z_L = R + \text{i}\, 2 \pi f L~,
\end{equation}
wobei $R$ und $L$ die oben bestimmten Widerstands- bzw. Induktivitätsbeläge
sind. Somit ergibt sich die Impedanz zu
\begin{equation}
  Z_L = \SI{(8.71 + 0.06i)}{\ohm}~.
\end{equation}
Die Phasenverschiebung $\varphi$ kann nun beschrieben werden als Winkel
zwischen den Reflexionsfaktoren
\begin{equation}
  \Gamma_{L,R} = \frac{Z - Z_0}{Z + Z_0}\,, \quad Z = R, Z_L~,
  % -0.70 + 0.00i
\end{equation}
wobei
\begin{equation}
  \Gamma_R = \SI[parse-numbers = false]{-0.222 \pm 0.009}{}~,
\end{equation}
der Reflektionsfaktor am Ende des Kabels,
aus Gleichung \eqref{eq:Gamma_R} gegeben ist.
Damit kann nun die Phase mit
\begin{equation}
  \varphi = \measuredangle(\Gamma_L, \Gamma_R)
\end{equation}
berechnet werden, womit sich schließlich für die Länge
\begin{equation}
  l_L = \underline{\SI{32.0}{\meter}}
\end{equation}
ergibt.
Im offenen Fall gilt
\begin{align}
  \Gamma_R &= 1 \\
  Z_L &= - \frac{i}{2\pi\ fC}~.
\end{align}
Hiermit wird die analog zum kurzgeschlossenen Fall die Länge zu
\begin{equation}
  L_C = \SI{20.1}{\meter}
\end{equation}
bestimmt.

% ==================================================
% 	Mehrfachreflexion
% ==================================================

\subsection{Mehrfachreflexion}
\label{sub:mehrfachreflexion}

In diesem Teil sollen Reflexionen betrachtet werden, die entstehen, wenn zwei
Kabel mit unterschiedlichem Widerstand hintereinandergeschaltet werden.
In diesem Fall das kurze \CU und das kurze \BU-Kabel.
Das aufgenommene Oszillosgraphenbild und die entsprechend digitalisierten Werte
sind in Abbildung~\ref{fig:mehrfachreflexion_roh}
bzw.~\ref{fig:mehrfachreflexion} dargestellt.
Nun werden wieder die Werte der Spannungsplateaus gemittelt, um die
Spannungsflanken und schließlich die Reflexionsfaktoren zu bestimmen.
Die hier verwandten Werte ergeben sich aus
\begin{align*}
  \begin{aligned}
    U_\text{off} &= \overline{U(t)}\,, \\
    U_0          &= \overline{U(t)}\,, \\
    U_1          &= \overline{U(t)}\,, \\
    U_2          &= \overline{U(t)}\,, \\
    U_3          &= \overline{U(t)}\,,
  \end{aligned}
  \qquad
  \begin{aligned}
    &t~< \SI{80}{\nano\second} \\
    \SI{110}{\nano\second} <~&t~< \SI{180}{\nano\second} \\
    \SI{210}{\nano\second}  <~&t~< \SI{285}{\nano\second} \\
    \SI{322}{\nano\second} <~&t~< \SI{380}{\nano\second} \\
    \SI{432}{\nano\second} <~&t
  \end{aligned}
\end{align*}
mit den Ergebnissen
\begin{align*}
  U_\text{off} &= \SI[parse-numbers = false]{202.0 \pm 0.8}{\milli\volt} \\
  U_0 &= \SI[parse-numbers = false]{202.0 \pm 0.8}{\milli\volt} \\
  U_1 &= \SI[parse-numbers = false]{224.5 \pm 1.0}{\milli\volt} \\
  U_2 &= \SI[parse-numbers = false]{330.4 \pm 1.7}{\milli\volt} \\
  U_3 &= \SI[parse-numbers = false]{330.4 \pm 1.7}{\milli\volt}~.
\end{align*}
Von diesen Spannungen wird die Offset-Spannung abgezogen und die folgenden
Spannungsdifferenzen berechnet.
\begin{align*}
  \Delta U_1 &= U_1 - U_0 \\
  \Delta U_2 &= U_2 - U_1 \\
  \Delta U_3 &= U_3 - U_2~.
\end{align*}
Hiermit lassen sich nun die Reflexionsfaktoren zu
\begin{align*}
  \Gamma_L &= \frac{\Delta U_1}{U_0} = \SI[parse-numbers = false]{0.183 \pm 0.012}{} \\
  \Gamma_E &= \frac{\Delta U_3}{\Delta U_2} +
  \frac{\Delta U_2}{U_0(1 - \Gamma_L)}
  = \SI[parse-numbers = false]{0.916 \pm 0.023}{} \\
  \Gamma_R &= \frac{\Delta U_3}{\Delta U_2 \Gamma_E}
  = \SI[parse-numbers = false]{-0.152 \pm 0.019}{}
\end{align*}
bestimmen.
Die Reflexionsfaktoren $\Gamma_L$ und $\Gamma_R$ sollten nach der Theorie
betragsmäßig ca. 0.2 sein. Somit ist anzunehmen, dass die verwandten Kabel von
ihrer Nennimpedanz abweichen.

\begin{figure}[htpb]
  \centering
  \includegraphics[scale=0.6]{bilder/mehrfachreflexion/F0000TEK.JPG}
  \caption{Aufgenommenes Oszillosgraphenbild bei hintereinandergeschalteten
    \CU- und \BU-Kabel.}
\label{fig:mehrfachreflexion_roh}
  \includegraphics[scale=1.0]{bilder/mehrfachreflexion/mf.pdf}
  \caption{Darstellung der digitalisierten Daten aus dem
    Oszilloskopenbild aus Abbildung~\ref{fig:mehrfachreflexion_roh}}
\label{fig:mehrfachreflexion}
\end{figure}

\subsection{Spannungsverlauf bei drei verschiedenen Abschlusswiderständen}
\label{sub:spannungsverlauf_bei_drei_verschiedenen_abschlusswiderst_nden}

In diesem Versuchsteil sollen die Spannungsverläufe mit drei unterschiedlichen,
noch unbekannten, Abschlusswiderständen untersucht werden.
Hierbei wird das kurze \CU-Kabel verwandt.

\paragraph{Abschluss 3}
\label{ssub:abschluss_3}

Der Spannungsverlauf ist in Abbildung~\ref{fig:abschluss_3} dargestellt.
Im Vergleich der im Anhang befindlichen Abbildung~\ref{fig:abschluesse} handelt
es sich hier um Abschluss, welcher aus einer Parallelschaltung von einem
Widerstand mit einem Kondensator besteht.

\paragraph{Abschluss 4}
\label{ssub:abschluss_4}

Der Spannungsverlauf des Signals mit Abschluss 4 ist in
Abbildung~\ref{fig:abschluss_4} dargestellt. Hierbei handelt es sich um eine
Reihenschaltung von einem Widerstand mit einer Spule.

\paragraph{Abschluss 1}
\label{par:abschluss_1}

Hier befindet sich der Spannungsverlauf in Abbildung~\ref{fig:abschluss_1}.
Der Vergleich mit den Abschlüssen in Abbildung~\ref{fig:abschluesse} liefert
hier einen Abschlusswiderstand, welcher aus einer Reihenschaltung eines
Widerstandes und einem Kondensator besteht.


\begin{figure}[htpb]
  \centering
  \fbox{\includegraphics[scale=0.45]{bilder/abschluss/F0003TEK.JPG}}
  \caption{Spannungsverlauf mit Abschlusswiderstand 3.}
\label{fig:abschluss_3}
\end{figure}

\begin{figure}[htpb]
  \centering
  \fbox{\includegraphics[scale=0.5]{bilder/abschluss/F0004TEK.JPG}}
  \caption{Spannungsverlauf mit Abschlusswiderstand 4.}
\label{fig:abschluss_4}
  \vspace{2em}
  \fbox{\includegraphics[scale=0.5]{bilder/abschluss/F0005TEK.JPG}}
  \caption{Spannungsverlauf mit Abschlusswiderstand 1.}
\label{fig:abschluss_1}
\end{figure}

