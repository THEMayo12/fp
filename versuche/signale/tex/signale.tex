% ==================================================
%	Dokumentklasse
% ==================================================

\documentclass[
	11pt,
	a4paper,
	fleqn,
	ngerman,
	parskip,
	toc=bibliography
]{scrartcl}

% input header
%        File: header.tex
%     Created: Sa Mär 08 11:00  2014 C
% Last Change: Sa Mär 08 11:00  2014 C
%

% ==================================================
%	Encoding
% ==================================================

\usepackage[utf8]{inputenc}
\usepackage[T1]{fontenc}
\usepackage{lmodern}
\usepackage{textcomp}

% ==================================================
%	Spacheinstellung
% ==================================================

\usepackage[ngerman]{babel}

% ==================================================
%	Referenzen
% ==================================================

\usepackage[ngerman]{varioref}
% Links im pdf
\usepackage[pdfborder={0 0 0}, hypertexnames=false]{hyperref}
% zusammen mit varioref, hyperref folgt damit
% z. B. in \vref{eq:1} -> in Gl. 1 auf Seite 4, cleveres referenzieren
\usepackage{cleveref}

% ==================================================
%	Grafiken, Abbildungen und Tabellen
% ==================================================

\usepackage{graphicx}
\usepackage{xcolor}
\usepackage[font=small, labelfont=bf, format=plain]{caption}
\usepackage{subcaption}
\usepackage{booktabs}
% for floating figures and floating tables
\usepackage[vflt]{floatflt}

\usepackage{rotating}

\usepackage{multicol}
\usepackage{multirow}

\usepackage{tikz}
\usetikzlibrary{shadows, calc, arrows, patterns}

\usepackage{tcolorbox}
\tcbuselibrary{skins, breakable, theorems}

% ==================================================
%	Seiten-Layout und -Definitionen
% ==================================================

% Maße für DIN A4
\usepackage[a4paper]{geometry}

\clubpenalty10000
\widowpenalty10000
\displaywidowpenalty=10000

% ==================================================
%	Float-Parameter
% ==================================================

% minimaler Anteil der Seite für den Text
\renewcommand{\textfraction}{0.05}
% maximaler Anteil der Seite für Floats am Anfang
\renewcommand{\topfraction}{0.95}
% maximaler Anteil der Seite für Floats am Enden
\renewcommand{\bottomfraction}{0.95}
% mininaler Anteil der Float-Seite für Text
\renewcommand{\floatpagefraction}{0.35}
% maximale Anzahl der Floats auf der Seite
\setcounter{totalnumber}{5}

% ==================================================
%	Fancy Header
% ==================================================

\usepackage{fancyhdr}
\pagestyle{fancy}

% Stil für gesamtes Dokument
\fancyhf{}

% Dicke der Linien ändern
\renewcommand{\headrulewidth}{1.0pt}
\renewcommand{\footrulewidth}{1.0pt}

% Abschnitte in Großschrift
\renewcommand{\sectionmark}[1]{\markright{\MakeUppercase{#1}}{}}
\renewcommand{\subsectionmark}[1]{}

% Die Positionen in Kopf- und Fußzeile füllen
\fancyhead[L]{
	TU Dortmund
}
\fancyhead[R]{\tit}
\fancyfoot[L]{\rightmark}
\fancyfoot[R]{Seite \thepage}

\addtolength{\headheight}{2\baselineskip}

% Neudefinition von plain (wird auf Kapitel/Verzeichnis-Seiten verwendet)
\fancypagestyle{plain}{
	\fancyhead[L]{}
	\fancyhead[R]{}
	\fancyfoot[L]{\rightmark}
	\fancyfoot[R]{Seite \thepage}
}

\fancypagestyle{firstpage}{
	\fancyhf{}
	\fancyhead[L]{}
	\renewcommand{\footrulewidth}{0.0pt}
}

% ==================================================
%	Bibliograhphie
% ==================================================

\newcommand{\anhang}{
	\clearpage
	\setcounter{page}{0}
	\pagenumbering{Roman}
	% Kapitelnummerierung in Großbuchstaben statt Zahlen
	\appendix
}

\newcommand{\referenzen}{
	% \renewcommand{\refname}{Quellenverzeichnis}
	\bibliographystyle{utphys}
	\bibliography{literatur}
}

% ========================================
%	Angaben für das Titelblatt
% ========================================

% einfaches Verändern des Titels/Versuchs in der Hauptdatei
\newcommand{\titel}[1]{\newcommand{\tit}{#1}}
\newcommand{\versuch}[1]{\newcommand{\ver}{Versuch #1}}

% ==================================================
%	Mathematik
% ==================================================

\usepackage{amsmath}
\usepackage{amsfonts}
\usepackage{amssymb}
\usepackage{amstext}
\usepackage{mathtools}
% für Einheitsmatrix '1'
\usepackage{bbm}

% Mathematik Font
\usepackage[sc]{mathpazo}
% Palatino needs more leading (space between lines)
\linespread{1.05}

% ==================================================
%	Physik
% ==================================================

\usepackage{upgreek}
\usepackage{physics}
\usepackage[nice]{units}
\usepackage[parse-numbers=false, per-mode=symbol]{siunitx}

% ==================================================
%	Sonstiges
% ==================================================

% rechnen mit LateX-Counter
\usepackage{calc}
% Euro-Zeichen
\usepackage{eurosym}
% ein paar LateX-Fehler beheben
\usepackage{fixltx2e}
% für bessere Darstellung von Text (Abstände)
\usepackage{microtype}
% Einfügen von Text um Layout zu testen
\usepackage{lipsum}


% ==================================================
%	Definitionen für das Dokument
% ==================================================


% ==================================================
%	Dokument beginnt
% ==================================================

\begin{document}

% ==================================================
%	Titelseite
% ==================================================

\titel{Signale auf Leitungen}
\versuch{V52}
\newcommand{\thedate}{12.01.2015}

\begin{titlepage}

% Alles zentriert
\center

% ==================================================
%	Oberer Teil
% ==================================================

\textsc{\LARGE TU Dortmund}\\[1.5cm]
\textsc{\Large Fakultät Physik}\\[0.5cm]
\textsc{\large \ver}\\[0.5cm]

\rule{\textwidth}{1.6pt}\vspace*{-\baselineskip}\vspace*{2pt}
\rule{\textwidth}{0.4pt}\\[\baselineskip]

{ \huge \bfseries \tit}\\[0.2cm]

\rule{\textwidth}{0.4pt}\vspace*{-\baselineskip}\vspace{3.2pt}
\rule{\textwidth}{1.6pt}\\[\baselineskip]

% ==================================================
%	Authoren
% ==================================================

\begin{minipage}{0.4\textwidth}
\begin{flushleft} \large
Mario \textsc{Dunsch}\\
\small \href{mario.dunsch@tu-dortmund.de}{mario.dunsch@tu-dortmund.de}
\end{flushleft}
\end{minipage}
~
\begin{minipage}{0.4\textwidth}
\begin{flushright} \large
Dominik \textsc{Kahl}\\
\small \href{dominik.kahl@tu-dortmund.de}{dominik.kahl@tu-dortmund.de}
\end{flushright}
\end{minipage}\\[1cm]

% ==================================================
%	Datum
% ==================================================

{\large \thedate}\\[3cm]

\vfill

\end{titlepage}


% ==================================================
%	Inhaltsverzeichnis
% ==================================================

\thispagestyle{plain}
\tableofcontents
\newpage

% ==================================================
%	Hauptteil
% ==================================================


% ==================================================
%	Einleitung
% ==================================================

\section{Einleitung}
In dem Versuch \textit{Temperaturabhängigkeit der Molwärme von Festkörpern} 
wird die Molwärme von Kupfer in Abhängigkeit der 
Temperatur gemessen und die materialspezifische Debye-Temperatur bestimmt. 

Zunächst werden drei Theoriemodelle zur Berechnung der Molwärme 
vorgestellt und die Debye-Temperatur eingeführt. Es folgt eine 
Beschreibung des Versuchsaufbaus und der Versuchsdurchführung. Danach werden 
aus den aufgenommenen Messwerten die gesuchten Größen ermittelt und 
abschließend werden diese mit Literaturwerten verglichen.

% ==================================================
%	Theorie
% ==================================================

\section{Theorie}
\subsection{Die verlustfreie und reale Leitung}
Um die Eigenschaften einer verlustfreien Leitung zu modellieren kann das in Abbildung 
\ref{fig:verlustfrei} dargestellte Ersatzschaltbild benutzt werden, wobei $L$ eine 
Spule und $C$ einen Kondensator darstellen. Um nun eine reale, verlustbehaftete 
Leitung zu simulieren werden die Widerstände $R$ und $G$ gemäß Abbildung 
\ref{fig:verlustbehaftet} in das Ersatzschaltbild integriert. Eine reale Leitung 
wird nun durch den Kapazitätsbelag $C$, den Induktivitätsbelag $L$, den ohmschen 
Belag $R$ und den Querleitfähigkeitsbelag $G$ charakterisiert. Die (komplexe) 
Spannung $U(t,z)$ zur Zeit $t$ am Ort $z$ kann dann durch

%fig

\begin{equation}
U(t,z)=U_0 \e^{-\gamma z}\e^{\i\omega t} \label{eq:Spannung}
\end{equation}
beschrieben werden, wobei $\gamma=\alpha + \beta \i =\sqrt{(R+\i \omega L)(G+\i 
\omega C)}$ die Ausbreitungskonsante mit dem Dämpfungsbelag $\alpha$ und dem 
Phasenbelag $\beta$ ist und $\omega$ die Kreisfrequenz der angelegten Spannung mit 
Amplitude $U_0$. Daraus 
lässt sich der Wellenwiderstand $Z_0$ ableiten als das Verhältnis von 
Spannungsamplitude zu Stromamplitude
\begin{equation}
Z_0:= \frac{U(\omega)}{I(\omega)} = \sqrt{\frac{R+\i \omega L}{G+\i \omega C}}
\label{eq:Wellenwiderstand} \quad ,
\end{equation}
hier für den Spezialfall, dass das Leitungskabel an jeder Stelle die gleichen 
Eigenschaften hat. Durch die Frequenzabhängigkeit von \eqref{eq:Spannung} und 
\eqref{eq:Wellenwiderstand} kommt es zur Dispersion und somit zur Verzerrung von 
Signalen, die auf die Leitung gegeben werden.
\subsection{Spannungsimpulse auf Leitungen}
Oft ist es nützlich, das Verhalten eines Spannungsimpulses auf einer Leitung zu 
untersuchen. Hierzu werden neben $Z_0$ noch die Quellenimpedanz $Z_\text{g}$ und 
die Lastimpedanz $Z_\text{L}$ benötigt. Ist $U_0$ die Spannung des eingehenden 
Impulses, und $U_\text{r}$ die des am anderen Ende reflektierten, so gilt für die 
Spannung $U_\text{L}$ an der Lastimpedanz
\begin{equation}
U_\text{L}=U_0+U_\text{r} \quad .
\end{equation}
Außerdem lässt sich der Reflexionsfaktor $\Gamma$ als
\begin{equation}
\Gamma := \frac{U_\text{r}}{U_0} =\frac{Z_\text{L}-Z_0}{Z_L + Z_0} =|\Gamma|e^{\i 
\varphi_\Gamma} \label{eq:Gamma}
\end{equation}
definieren. Die Signalspannung in Ortsdarstellung lässt sich dann mit einer 
Laplacetransformation $\mathfrak{L}$ aus der 
Impulsdarstellung gemäß
\begin{equation}
U_\text{r}(t)=\mathfrak{L}^{-1}(U_\text{r}(p))=\mathfrak{L}^{-1}(\Gamma(p)U_\text{h}
(p))
\end{equation}
bestimmen, wobei $\Gamma$ der Reflexionsfaktor in Impulsdarstellung ist, und 
$U_\text{r}(p)$ und $U_\text{h}(p)$ die Spannung des reflektierten bzw. hinlaufenden 
Spannungspulses ist. Diese Rechnung entspricht der Näherung, dass die Leitung nahezu 
verlustfrei ist.
\subsection{Der Skin-Effekt bei Koaxialkabeln}
In diesem Verusuch werden Koaxialkabel verwendet. Diese bestehen aus einem inneren 
Leiter als Kern. Dieser wird von einem Dielektrikum und dieses wiederum von einem 
äußeren Leiter umgeben. Durch die magnetischen Felder innerhalb des Kabels werden 
Wirbelströme induziert, welche den eigentlichen Wechselstrom an den äußeren Rand der 
Leiter drängt, sodass der Leitungsquerschnitt verkleinert und somit der Widerstand 
vergrößert wird. Aus diesem Grund wird der Widerstand $R$ eines Koaxialkabels bei 
Kreisfrequenzen ab $100 \text{ kHz}$ durch ein $\sqrt{\omega}$-Gesetzt angemessen 
beschrieben.
\subsection{Smith-Diagramme}
Smith-Diagramme sind ein Hilfsmittel, um zu einem gegebenen Lastwiderstand 
$Z_\text{L}$ und Wellenwiderstand $Z_0$ den Reflexionsfaktor $\Gamma$ numerisch 
zu ermitteln. Dazu wird \eqref{eq:Gamma} mit $z_\text{L}:=Z_\text{L}/Z_0$ 
umgeschrieben zu
\begin{equation}
\Gamma = \frac{z_\text{L}-1}{z_\text{L}+1} \quad .
\end{equation}
Interpretiert man nun $\Gamma(z_\text{L})$ als komplexe Funktion ist dies eine 
Möbiustransformation, sodass, da diese Abbildung konform ist, die Bilder der 
Koordinatenachsen im Zielraum ebenfalls ein Koordinatessystem bilden. In Abbildung 
\ref{fig:Smith} werden über ein (nicht eingezeichnetes) Kartesisches 
Koordinatensystem die Bilder der Koordinatenlinien eingezeichnet. Um nun hieraus 
einen Reflexionskoeffizienten zu bestimmen, muss lediglich $z_\text{L}$ in das 
transformierte Koordinatensystem eingetragen werden. Dieser Punkt entspricht, wenn 
er in Kartesischen Koordinaten abgelesen wird gerade $\Gamma(z_\text{L})$.

%fig

%\cite{FP}
% ==================================================
%	Aufbau
% ==================================================

\section{Aufbau}
\begin{figure}
\centering 
\includegraphics[scale=1]{figures/setup.pdf}
\caption{Schematische Abbildung des Versuchsaufbaus. \cite{FP}}
\label{fig:aufbau:setup}
\end{figure}
Der Versuchsaufbau ist in Abbildung \ref{fig:aufbau:setup} zu sehen. Der 
Rezipient befindet sich in einem Gehäuse, das mittels einer Vakuumpumpe 
evakuiert werden kann. Über ein Ventil kann das Gehäuse mit Helium befüllt 
werden. Rezipient und Gehäuse besitzen jeweils eine eigene Heizwicklung, sodass 
diese über getrennte Stromversorgungen einzeln geheizt werden können. Die 
Temperaturmessung geschieht mittels Widerstandsmessung eines Pt-100-Messwiderstandes, 
für den die Beziehung 
\begin{equation}
T = 0.00135 R^2 + 2.296 R -243.02
\end{equation}
zwischen dem gemessenen Widerstandswert $R$ in $\Omega$ und der Temperatur $T$ in 
$\si{\celsius}$ gilt. Das Gehäuse befindet sich in einem Dewargefäß.

% ==================================================
%	Durchführung
% ==================================================

\section{Aufbau und Durchführung}

Um die Kristallstruktur eines Probenmaterials zu bestimmen, wird die Probe mit
monochromatischen Röntgenlicht bestrahlt und die Beugungswinkel der
entstehenden Bragg-Reflexe ausgemessen.
In der Auswertung ist es schließlich die Aufgabe die, den Beugungswinkeln
entsprechenden Netzebenen, zu finden. Dabei ist es sinnvoll die Netzebenen zu
identifizieren, bei denen kein gebeugter Strahl erzeugt wird, da hierfür gerade
der Strukturfaktor verschwindet. Anhand dieser Reflexe, lässt sich in vielen
Fällen die Kristallstruktur bestimmen.

\begin{figure}[htpb]
  \centering
  \includegraphics[scale=0.5]{bilder/aufbau.png}
  \caption{Schematische Darstellung des Versuchsaufbaus.}
\label{fig:aufbau}
\end{figure}

Im Allgemeinen ist es schwierig Bragg-Reflexe zu erzeugen, weshalb in diesem
Versuch das Debye-Scherrer-Verfahren verwandt wird, in dem kein Einkristall
sondern eine fein pulverisierte Probe benutzt wird.
In dieser Probe sind die Mikrokristalle statistisch über den Raum verteilt,
sodass es sehr wahrscheinlich ist, bei jeder Einstrahlrichtung einen
Reflexstellung eines Kristalliten zu finden.
Der Nachweis der gebeugten Röntgenstrahlung geschieht hier mit Hilfe eines
Filmstreifens.
Der schematische Aufbau des Versuchs ist in Abbildung~\ref{fig:aufbau} gezeigt.
Das Röntgenlicht der Quelle tritt hier durch eine kleine Öffnung in ein
zylindrisches Gehäuse ein, in dessen Mitte sich die Probe, ein auf einem
Glasröhrchen aufgetragenes Probenmaterial, befindet.
Das an der Probe reflektierte Röntgenlicht trifft nun auf die Gehäusewand, an
welcher der Filmstreifen befestigt ist. Die dabei unter einem Winkel $\theta$
gebeugte Strahlung wird sich aufgrund der statistischen Verteilung der
Kristalliten auf einem Kegelwinkel mit Öffnungswinkel $2\theta$ befinden.
So werden nahezu kreisförmige Linien auf dem Filmstreifen erhalten.

Hierbei können nun noch zwei zu beachtenden systematischen Fehler auftreten.
\begin{enumerate}
  \item Die Probenachse fällt nicht mit der Achse des Filmzylinders zusammen.
    Dadurch werden die Ringe um einen systematischen Fehler vergrößert oder
    verkleinert. Die Korrektur $\Delta a_V$ zu Gitterkonstanten  $a$ beträgt
    hierbei
    \begin{equation}
      \frac{\Delta a_V}{a} = \frac{v}{\cos^2\theta}~,
      \label{eq:a_V}
    \end{equation}
    wobei $v$ die Verschiebung der Symmetrieachsen und $R$ der Radius des
    Filmzylinders ist.
  \item Die Probe absorbiert die einfallende Röntgenstrahlen nahezu
    vollständig. Dadurch findet die Beugung nur an einem schmalen Streifen der
    Probe statt, wodurch der Winkel $\theta$ systematisch zu groß gemessen
    wird. Die Korrektur lautet hierbei
    \begin{equation}
      \frac{\Delta a_A}{a} = \frac{\rho}{2R}
      \qty(1 - \frac{R}{F})\frac{\cos^2\theta}{\theta}~,
      \label{eq:a_A}
    \end{equation}
    wobei $\rho$ der Radius der Probe und $F$ der Abstand Fokus-Probe ist.
\end{enumerate}


\clearpage

% ==================================================
%	Auswertung
% ==================================================

\section{Auswertung}
Auf dem als Ergebnis der experimentellen Durchführung erhaltenen Fotostreifen sind 
jeweils einige Beugungsringe schwach zu erkennen. Zunächst wurden die Radien dieser 
Ringe gemessen und der Beugungswinkel $\vartheta$ bestimmt. Die Messwerte sind in den 
Tabellen \ref{tab:1} und \ref{tab:2} zu sehen.

\begin{table}[h]
\centering
\begin{tabular}{ccccc}
\toprule
\midrule
 $r$/mm & $\vartheta$ &$s_\text{exp}$& $s_\text{th}$& $a$/$\AA$ \\
\midrule
32.000            & 0.279             & \phantom{0}4.000  & \phantom{0}4.000  & 5.591            \\
39.500            & 0.345             & \phantom{0}6.012  & \phantom{0}6.000  & 5.585            \\
50.500            & 0.441             & \phantom{0}9.580  & \phantom{0}8.000  & 5.109            \\
53.000            & 0.462             & 10.482            & 10.000            & 5.461            \\
68.500            & 0.598             & 16.677            & 16.000            & 5.476            \\
79.000            & 0.689             & 21.302            & 20.000            & 5.417            \\
83.500            & 0.729             & 23.345            & 22.000            & 5.427            \\
\midrule
\bottomrule
\end{tabular}
\caption{Messwerte und Vergleichswerte zur ersten Probe. Dabei ist $r$ der Radius 
eines Ringes, $\vartheta$ der Beugungswinkel und $a$ die ermittelte Gitterkonstante.}
\label{tab:1}
\end{table}
\begin{table}[h]
\centering
\begin{tabular}{ccccc}
\toprule
\midrule
$r$/mm & $\vartheta$ &$s_\text{exp}$& $s_\text{th}$& $a$/$\AA$ \\
\midrule
32.000            & 0.279             & \phantom{0}4.000  & \phantom{0}4.000  & 5.591            \\
39.500            & 0.345             & \phantom{0}6.012  & \phantom{0}6.000  & 5.585            \\
50.500            & 0.441             & \phantom{0}9.580  & \phantom{0}8.000  & 5.109            \\
53.000            & 0.462             & 10.482            & 10.000            & 5.461            \\
68.500            & 0.598             & 16.677            & 16.000            & 5.476            \\
79.000            & 0.689             & 21.302            & 20.000            & 5.417            \\
83.500            & 0.729             & 23.345            & 22.000            & 5.427            \\
\midrule
\bottomrule
\end{tabular}
\caption{Messwerte und Vergleichswerte zur zweiten Probe. Dabei ist $r$ der Radius 
eines Ringes, $\vartheta$ der Beugungswinkel und $a$ die ermittelte Gitterkonstante.}
\label{tab:2}
\end{table}

\subsection{Bestimmung der Gitterstruktur und der Gitterkonstanten}

Die Werte wurden wie folgt berechnet. Zu einem Ringradius $r$ und der Fotospule vom 
Radius $R=57.4 \text{ cm}$ lässt sich der Beugungswinkel $\vartheta=r/2R$ ermitteln. 
Definiere
\begin{equation}
s_i^\text{exp}:=4 \frac{\sin^2(\vartheta_i)}{\sin^2(\vartheta_0)} \quad ,
\end{equation}
wobei $\vartheta_0$ der kleinste gemessene Wert ist. Diese Definition leitet sich 
aus
\begin{equation}
\lambda=2\sin(\vartheta) d \label{eq}
\end{equation}
ab, wobei $\lambda$ die Wellenlänge des verwendeten Röntgenlichtes ist, $\vartheta$ der Beugungswinkel und $d=a/\sqrt{h^2+k^2+l^2}$ der Netzebenenabstand der $(h,k,l)$-
Netzebenenschar zu einem kubischen Gitter mit Gitterkonstante $a$. Der Faktor $4$ rührt daher, dass angenommen wird, dass der erste Ring zur 
Beugungsebene (2,0,0) gehört. Dies kann nur damit begründet werden, dass durch diese
Wahl die Messwerte sinnvoll sind. 
Definiere weiter $s^\text{th}_i:=h_i^2+k_i^2+l_i^2$, wobei $(h_i,k_i,l_i)$ die 
Millerindizes zum $i$-ten Beugungsring sind.\\
Im Falle einer perfekten Messung sollte also $s_i^\text{th}=s_i^\text{exp}$ gelten.
\\
Anhand der $s_i^\text{exp}$ wurde nun abgeschätzt, zu welchen Netzebenenscharen die 
Beugungsringe gehören und mit den bekannten Gitterstrukturen verglichen. Eine 
theoretische Betrachtung der Strukturfaktoren eines bcc-Gitters ergibt zum Beispiel 
Reflexe bei $s^\text{th} \in \{ 2,4,8,10,16,20,22,... \}$, für ein Diamantgitter 
ergeben sich lediglich Reflexe bei $s^\text{th}\in \{ 4,11,... \}$. Da dies sehr gut 
zu den Beobachteten $s^\text{exp}$ passt, wird nun angenommen, dass es sich bei der 
ersten Probe um eine bcc- oder NaCl- und bei der zweiten Probe um eine 
Diamantstruktur handelt. Wobei die NaCl-Struktur der bcc-Struktur für kleine 
Millerindizes sehr ähnlich ist.
\\
Schließlich können die Gitterkonstanten über
\begin{equation}
a=\frac{\sqrt{s^\text{th}} \lambda}{2 \sin(\vartheta)}
\end{equation}
(vgl. \eqref{eq}) bestimmt werden, wobei $(s^\text{th},\vartheta)$ für ein 
zusammengehörendes Paar von Netzebene und Beugungswinkel steht, außerdem steht 
$\lambda$ wieder für die Wellenlänge der Röntgenstrahlung.

\subsection{Korrektur der Gitterkonstanten und Bestimmung der Probenmaterialien}
Um eine möglichst gute Schätzung für die Gitterkonstante $a$ zu bekommen, wird nun 
$a$ gegen $\cos^2(\vartheta)$ aufgetragen. Die Extrapolation auf 
$\cos^2(\vartheta)=0$ ergibt dann den besten Wert für die Gitterkonstante. In 
den Abbildungen \ref{fig:1} und \ref{fig:2} sind die Gitterkonstanten gegen 
$\cos^2(\vartheta)$ aufgetragen und es wurde eine lineare Regression durchgeführt.
\begin{figure}[h]
\centering
\includegraphics[scale=0.8]{bilder/fig1.pdf}
\caption{Extrapolation der Gitterkonstanten für die erste Probe.}
\label{fig:1}
\end{figure}
\begin{figure}[h]
\centering
\includegraphics[scale=0.8]{bilder/fig2.pdf}
\caption{Extrapolation der Gitterkonstanten für die zweite Probe.}
\label{fig:2}
\end{figure}
Es ergeben sich die Regressionsgeraden $(t:=\cos^2(\vartheta))$
\begin{equation}
G_1(t) = \SI[parse-numbers = false]{\left(2.3 \pm 5.0\right) \times 10^{-11}}{}\, \cdot \,t\, + \SI[parse-numbers = false]{\left(5.3 \pm 0.4\right) \times 10^{-10}}{\meter}
\end{equation}
sowie
\begin{equation}
G_2(t) = \SI[parse-numbers = false]{\left(-1.10880741251 \pm 0\right) \times 10^{-10}}{}\, \cdot \,t\, + \SI[parse-numbers = false]{\left(6.61540296504 \pm 0\right) \times 10^{-10}}{\meter}
\end{equation}

Als beste Werte für die Gitterkonstanten ergeben sich so
\begin{align*}
a_{\text{bcc}}&=(5.3 \pm 0.4)\, \AA \\
a_\text{Diamant}&=(6.61 \pm 0 )\, \AA \quad ,
\end{align*}
wobei der Fehler von $a_2$ wegen der zu geringen Datenmenge in der Rechnung zwar 
verschwindet, jedoch in der gleichen Größenordnung wie der Fehler von $a_1$ zu 
erwarten ist.
Eine NaCl-Struktur mit $a_\text{NaCl}=5.6 \, \AA$ weist NaCl auf, sodass wir 
annehmen, dass es sich der ersten Probe um diesen Stoff handelt.

\clearpage

% ==================================================
%	Diskussion
% ==================================================
\clearpage
\section{Diskussion}
In Tabelle \ref{tab:ergebnisse} sind noch einmal die Ergebnisse 
dieser Versuchsdurchführung und die entsprechenden Literaturwerte 
aufgelistet.

\begin{table}[h]
\centering
\begin{tabular}{crc}
\toprule \midrule
Physikalische Größe & Ermittelter Wert & Literaturwert \\
\midrule
$k_\text{B}$ 	& 41.8+/-0.8
$\times 10^{-23}\frac{\text{J}}{\text{K}}$ & $1.380 648 
52(79)\times 10^{-23}\frac{\text{J}}{\text{K}}$\cite{pdg}		\\
"  				& (0.2+/-1.4)e-23
$\times 10^{-23}\frac{\text{J}}{\text{K}}$ & "		\\ 
"  				& (1.32+/-0.04)e-20
$\times 10^{-23}\frac{\text{J}}{\text{K}}$ & "		\\
"  				& (1.0+/-1.3)e-22
$\times 10^{-23}\frac{\text{J}}{\text{K}}$ & "		\\
$\text{e}_0$		& (6.39+/-0.14)e-25
$\times 10^{-19}\text{C}$ & $1.602 176 6208(98)\times 
10^{-19}$C\cite{pdg} \\
$\alpha$			& 0.979+/-0.018 
&$\mathcal{O}(1)$\\
\midrule
\bottomrule
\end{tabular}
\caption{Zusammenstellung der Versuchsergebnisse.}
\label{tab:ergebnisse}
\end{table}


%cite{pdg} http://pdg.lbl.gov/2015/reviews/rpp2015-rev-phys-constants.pdf


Bei den ermittelten Werten für die Boltzmankonstante $k_\text{B}$ 
zeigen sich für die Messungen am einfachen Rauschspektrometer 
(erste zwei Werte) eine deutlich größere Abweichung vom Literaturwert 
im vergleich zu den Messungen am Rauschspektrometer nach 
Korrelationsprinzip. Aus den in der Theorie genannten Gründen 
war dieses Verhalten bereits zu erwarten. Insbesondere der Wert 
$k_\text{B}=(1.32+/-0.04)e-20$ am korrelierten Spektrometer 
mit kleinem Widerstand $R$ bis $1000\Omega$ enthält den Literaturwert 
in seinem Toleranzbereich. 

Der ermittelte Wert für die Elementarladung $\text{e}_0$ weicht 
geringfügig nach unten ab. Eine mögliche Begründung für diese 
Verschiebung ist, dass während der Messreihe die Kathode wieder 
an eine externe Stromquelle angeschlossen war und nicht allein 
vom Bleiakkumulator betrieben wurde, sodass Störsignale den 
Rauscheffekt beeinflusst haben könnten.

Der Exponent des Funkeleffekts, der als Repräsentant eines $1/f$ 
Rauschens untersucht wurde, liegt wie erwartet in der Größenordnung 
von $1$, sodass hier wirklich von einem $1/\nu^\alpha\approx 1/\nu$ 
Rauschen gesprochen werden kann. Wie ebenfalls erwartet trat der 
Funkeleffekt bei der Oxydkathode deutlich stärker auf als bei 
der Reinmetallkathode (vgl. dazu Abb. \ref{fig:kathode_rein} und 
\ref{fig:kathode_oxyd})  .

% ==================================================
%	Literaturverzeichnis
% ==================================================

\referenzen

% ==================================================
% 	Anhang
% ==================================================

\anhang

\section{Angang}
\label{sec:angang}

\begin{figure}[htpb]
  \centering
  \includegraphics[scale=0.6]{bilder/abschluss.png}
  \caption{Signalspannung $U(t)$ und Zeitkonstanten $T$ für verschiedene
  Abschlusswiderstände.}
  \label{fig:abschluesse}
\end{figure}

% \begin{figure}[htpb]
%   \centering
%   \includegraphics[scale=0.6]{bilder/smith.png}
%   \caption{Smith Diagramm.}
%   \label{fig:smith_diagramm}
% \end{figure}
\includepdf[pages=11]{../anleitung/Signale_E2.pdf}

% ==================================================
%	Dokument endet
% ==================================================

\end{document}
