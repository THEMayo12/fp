
% ==================================================
%	Auswertung
% ==================================================

\section{Auswertung}
\subsection{Ermittlung der Absorptionskeoffizienten}
Für die Berechnungen werden nur die Counts im Bereich von Kanal 257 bis Kanal 315 verwandt, da in
diesem Bereich der Gammastrahlungspeak liegt, der aus dem Teil der Strahlung stammt, welcher
das Medium direkt durchdringt und primär nicht aus anderen Streueffekten stammt. Um die
restlichen
Streubeiträge zu korrigieren wird eine Korrektur des Spektrums im relevanten Bereich vorgenommen.
Dazu wird eine Stufenfunktion von Kanal 220 bis 257 und von 315 bis 1023 (letzter Kanal) mit dem
Sprung beim Kanal mit den maximalen Counts an die Spektren gefittet und durch Faltung mit einer
Gaußfunktion geglättet. Die resultierende Untergrundabschätzung kann dann durch
\begin{equation}
\text{erfc}_{a,b,c,d}(x):=\frac{a-c}{2} \left(1-\text{erf}(d(x-b))\right)+c
\end{equation}
parametrisiert werden. Dabei sind $a$ und $c$ die Grenzwerte am linken und rechten Rand
(entsprechend der Stufenfunktion), $b$ der Kanal mit maximalen Counts und $d$ ein Maß für die
Breite der Verteilung, $\text{erf}$ ist die Gaußsche Fehlerfunktion. Die Abbildungen aller
Spektren und die gefitteten Untergrundfunktionen sind im Anhang zu sehen. In den Tabellen
\ref{tab:0}, \ref{tab:1}, \ref{tab:2} und \ref{tab:3} sind die Messzeiten mit den zugehörigen
Counts im relevanten Bereich zu sehen. Die Messung ohne Würfel ergibt $T_\text{init}=45.14$ s mit
$C_\text{init}=10236$ Counts.
\begin{table}
\centering
\begin{tabular}{ccc}
\toprule \midrule
Projektion &$T/$s & $C$ \\
\midrule
2.00              & 21.96             & \phantom{0}4779.30\\
8.00              & 25.42             & \phantom{0}5423.27\\
9.00              & 58.70             & 12539.47         \\
\midrule
\bottomrule
\end{tabular}
\caption{Korrigierte Messwertpaare von Messzeit $T$ und Counts $C$ für die vermessenen
Projektionen zu
Würfel 0.} \label{tab:0}

\begin{tabular}{ccc}
\toprule \midrule
Projektion &$T/$s & $C$ \\
\midrule
32.000            & 0.279             & \phantom{0}4.000  & \phantom{0}4.000  & 5.591            \\
39.500            & 0.345             & \phantom{0}6.012  & \phantom{0}6.000  & 5.585            \\
50.500            & 0.441             & \phantom{0}9.580  & \phantom{0}8.000  & 5.109            \\
53.000            & 0.462             & 10.482            & 10.000            & 5.461            \\
68.500            & 0.598             & 16.677            & 16.000            & 5.476            \\
79.000            & 0.689             & 21.302            & 20.000            & 5.417            \\
83.500            & 0.729             & 23.345            & 22.000            & 5.427            \\
\midrule
\bottomrule
\end{tabular}
\caption{Korrigierte Messwertpaare von Messzeit $T$ und Counts $C$ für die vermessenen
Projektionen zu
Würfel 1. Für diesen Würfel wurde die Projektion 2 nicht gemessen.} \label{tab:1}
\end{table}

\begin{table}
\centering
\begin{tabular}{ccc}
\toprule
\midrule
Projektion &$T/$s & $C$ \\
\midrule
\phantom{0}1\phantom{.} & \phantom{0}94.160 & 1198.704         \\
\phantom{0}2\phantom{.} & 195.640           & 1410.985         \\
\phantom{0}3\phantom{.} & 147.620           & 1170.847         \\
\phantom{0}4\phantom{.} & 115.560           & 1215.972         \\
\phantom{0}5\phantom{.} & 176.720           & 1316.292         \\
\phantom{0}6\phantom{.} & 144.620           & 1221.793         \\
\phantom{0}7\phantom{.} & \phantom{0}87.180 & 1250.064         \\
\phantom{0}8\phantom{.} & 428.820           & 1522.417         \\
\phantom{0}9\phantom{.} & 595.780           & 4119.903         \\
10\phantom{.}     & 152.200           & 1352.362         \\
11\phantom{.}     & 415.340           & 1498.667         \\
12\phantom{.}     & 175.040           & 2065.847         \\
\midrule
\bottomrule
\end{tabular}
\caption{Korrigierte Messwertpaare von Messzeit $T$ und Counts $C$ für die vermessenen
Projektionen zu
Würfel 2.} \label{tab:2}

\begin{tabular}{ccc}
\toprule
\midrule
Projektion &$T/$s & $C$ \\
\midrule
\phantom{0}1\phantom{.} & \phantom{0}508.100 & 10538.597        \\
\phantom{0}2\phantom{.} & \phantom{0}543.500 & 10872.075        \\
\phantom{0}3\phantom{.} & \phantom{0}585.580 & 12629.847        \\
\phantom{0}4\phantom{.} & \phantom{0}317.480 & 14410.246        \\
\phantom{0}5\phantom{.} & \phantom{0}353.200 & 12534.861        \\
\phantom{0}6\phantom{.} & 1047.180          & 10416.850        \\
\phantom{0}7\phantom{.} & \phantom{0}210.800 & 11614.966        \\
\phantom{0}8\phantom{.} & \phantom{0}628.400 & 10524.622        \\
\phantom{0}9\phantom{.} & \phantom{0}716.740 & 10686.688        \\
10\phantom{.}     & \phantom{0}637.780 & 10227.250        \\
11\phantom{.}     & \phantom{0}910.660 & 10327.936        \\
12\phantom{.}     & \phantom{0}224.540 & 10732.770        \\
\midrule
\bottomrule
\end{tabular}
\caption{Korrigierte Messwertpaare von Messzeit $T$ und Counts $C$ für die vermessenen
Projektionen zu
Würfel 3.} \label{tab:3}
\end{table}
\clearpage
Im nächsten Schritt werden die Intensitäten $I_{i,j}:=C_{i,j}/T_{i,j}$ der $j$-ten Projektion des
$i$-ten
Würfels aus den Messzeiten $T_{i,j}$ und Counts $C_{i,j}$ der $j$-ten Projektion des
$i$-ten Würfels berechnet.\\
Da in die Abschwächung der Intensität bei Durchtritt durch Materie exponentiell zu dem Produkt
aus Absorptionskoeffizient und Materialdicke ist,\footnote{Hierin bedeutet $i$ den Würfel und
$j$ die Projektion; $\vec{\mu}_i$ sind die Ansorptionskoeffizienten zum $i$-ten Würfel. Die
Gleichung ist für $i=1,2,3$ gültig.}
\begin{equation}
I_{i,j} \propto -\exp\left( (d A \vec{\mu}_i)_{j} + F_{\text{Mantel},j} \right)=
-\exp\left( (d A \vec{\mu}_i)_j \right) \exp\left( F_{\text{Mantel},j} \right) \quad ,
\end{equation}
kann der Effekt $F_{\text{Mantel},j}$ des Aluminiummantels einfach
korrigiert werden, indem an jede Intensität der 1,2,3,4,5 oder 6 Projektion der Faktor
$I_\text{init}/I_{0,2}$, an die 7,9,10 und 12 Projektionen der Faktor $I_\text{init}/I_{0,9}$ und
an die 8 und 11 Projektion der Faktor $I_\text{init}/I_{0,8}$ multipliziert wird. Die
resultierenden Intensitäten $J_1$, $J_2$ und $J_3$ sind in Tabelle \ref{tab:korr} zu sehen.
\begin{table}[h]
\centering
\begin{tabular}{cccc}
\toprule
\midrule
Projektion &	$J_1$ & $J_2$ & $J_3$ \\
\midrule
\phantom{0}1\phantom{.} & 37.854            & 13.514            & 22.017           \\
\phantom{0}2\phantom{.} & \phantom{0}-  & \phantom{0}7.515  & 20.843           \\
\phantom{0}3\phantom{.} & 38.403            & \phantom{0}8.264  & 22.472           \\
\phantom{0}4\phantom{.} & 37.999            & 10.964            & 47.292           \\
\phantom{0}5\phantom{.} & 37.734            & \phantom{0}7.761  & 36.977           \\
\phantom{0}6\phantom{.} & 38.981            & \phantom{0}8.803  & 10.365           \\
\phantom{0}7\phantom{.} & 45.676            & 14.940            & 57.410           \\
\phantom{0}8\phantom{.} & 22.481            & \phantom{0}3.188  & 17.779           \\
\phantom{0}9\phantom{.} & 44.270            & \phantom{0}7.350  & 15.848           \\
10\phantom{.}     & 50.458            & \phantom{0}9.432  & 17.022           \\
11\phantom{.}     & 23.317            & \phantom{0}3.075  & 12.039           \\
12\phantom{.}     & 47.914            & 12.544            & 50.804           \\
\midrule
\bottomrule
\end{tabular}
\caption{Die um die Effekte des Aluminiummantels korrigierten Intensitäten.} \label{tab:korr}
\end{table}
\clearpage

Für den so korrigierten Intensitätsvektor $\vec{J}_i:=(J_{i,1},J_{i,2},\ldots,J_{i,12})^\text{T}$
des $i$-ten Würfels gilt
\begin{equation}
J_{i,j}=I_\text{init}\exp\left(- (d A \vec{\mu}_i)_j \right) \quad ,
\end{equation}
und nach Anwenden des ln ergibt sich
\begin{equation}
A \vec{\mu}_i = \hat{J}_i \label{eq:linear}
\end{equation}
mit $\hat{J}_i:=(\ln(I_\text{init}/J_{i,1}),\ldots,\ln(I_\text{init}/J_{i,12}))^\text{T}$
  \footnote{
Da bei Würfel 1 die Projektion 2 fehlt, fällt für $i=1$ die zweite Zeile der Matrix $A$ weg.}.
Die Gleichung \eqref{eq:linear} kann in die Normalen-Gleichung
\begin{equation}
\vec{\mu}_i = \left( A^\text{T}A \right)^{-1} A^\text{T} \hat{J}_i \label{eq:linear2}
\end{equation}
überführt werden. Die daraus resultierenden Werte für die Absorptionskoeffizienten sind in
Tabelle \ref{tab:Koeff} zu sehen.
\begin{table}[h]
\centering
\begin{tabular}{cccc}
\toprule
\midrule
Elementarwürfel & $\vec{\mu}_1$/cm & $\vec{\mu}_2$/cm & $\vec{\mu}_3$/cm \\
\midrule
1.00              & 0.60              & 1.15              & 0.67             \\
2.00              & 0.60              & 1.01              & 0.61             \\
3.00              & 0.60              & 1.09              & 0.87             \\
4.00              & 0.52              & 0.95              & 0.42             \\
5.00              & 0.46              & 1.04              & 0.49             \\
6.00              & 0.51              & 1.30              & 1.26             \\
7.00              & 0.60              & 0.84              & 0.51             \\
8.00              & 0.63              & 1.16              & 0.68             \\
9.00              & 0.58              & 0.77              & 0.98             \\
\midrule
\bottomrule
\end{tabular}
\caption{Die Absorptionskoeffizienten-Vektoren $\vec{\mu}$ der Würfel 1, 2 und 3.}
\label{tab:Koeff}
\end{table}
\clearpage
\subsection{Fehlerbetrachtung}
Es bleibt nun noch eine Fehlerabschätzung für die gefundenen Absorptionskoeffizienten zu geben.
Da die Messunsicherheit der Counts einer Poisson-Verteilung unterliegt, wird zu $C$ gemessene
Counts der Fehler $\Delta C=\sqrt{C}$ erwartet.
Wird die Intensität $I_{i,j}$ der $j$-ten
Projektion des $i$-ten Würfels durch die fehlerbehaftete Messgrößen $C_{i,j}$ ausgedrückt,
\begin{equation}
I_{i,j}=\frac{C_{i,j}}{T_{i,j}}
\end{equation}
ergibt sich mit der Gaußschen Fehlerfortpflanzung\footnote{$i\in \{0,1,2,3,\text{init}\}$, für
$i=\text{init}$ fällt der Index $j$ weg.}
\begin{equation}
\Delta I_{i,j}=\frac{\Delta C_{i,j}}{T_{i,j}} \quad.
\end{equation}
Die Korrektur des Aluminiummantels\footnote{$i\in \{1,2,3\}$, zur besseren Übersichtlichkeit
steht $J_{i,j}$ für $\vec{J}_{i,j}$}
\begin{equation}
J_{i,j}=\frac{I_\text{init}}{I_{0,j}} I_{i,j}
\end{equation}
bekommt einen Fehler von
\begin{equation}
\Delta J_{i,j}= \sqrt{\left(\frac{\Delta I_\text{init}}{I_{0,j}} I_{i,j} \right)^2 +
				\left(- \frac{I_\text{init}}{I_{0,j}^2} I_{i,j} \Delta I_{0,j} \right)^2 +
				\left(\frac{I_\text{init}}{I_{0,j}}\Delta I_{i,j}\right)^2 } \quad.
\end{equation}
Schließlich werden daraus die Werte
\begin{equation}
\hat{J}_{i,j}=\ln\left( \frac{I_\text{init}}{J_{i,j}}  \right)
\end{equation}
berechnet, denen wiederum gemäß der Gaußschen Fehlerfortpflanzung der Fehler
\begin{equation}
\Delta \hat{J}_{i,j}=\sqrt{
\left(\frac{J_{i,j}}{I_\text{init}} \Delta I_\text{init}\right)^2 +
\left(\frac{J_{i,j}}{I_\text{init}} \frac{-I_\text{init}}{J^2_{i,j}} \Delta J_{i,j}\right)^2
}
\end{equation}
zugeordnet wird. \\
Für die Fehlerabschätzung der Normalengleichung \ref{eq:linear2} werden nun die Matrizen
\begin{equation}
W_i:=\text{diag} (1/ (\Delta\hat{J}_{i,j})^2)
\end{equation}
definiert, wobei angenommen wird, dass die Messungen nicht korreliert sind \cite{blobel1998statistische} .
Die Fehler der Absorptionskoeffizienten lassen sich dann als
\begin{equation}
\left(\Delta \vec{\mu}_i\right)_j=\left( (A^\text{T}W_i A)^{-1} \right)_{jj}
\end{equation}
bestimmen.

\begin{table}
\centering
\begin{tabular}{c|cc|cc|cc}
\toprule \midrule
Elementarwürfel & $\vec{\mu}_1$/cm & $\Delta \vec{\mu}_1$/cm& $\vec{\mu}_2$/cm & $\Delta \vec{\mu}_2$/cm& $\vec{\mu}_3$/cm & $\Delta \vec{\mu}_3$/cm \\
\midrule
1\phantom{.}      & 0.5942            & 0.0605            & 1.1967            & 0.0145            & 0.6496            & 0.0336           \\
2\phantom{.}      & 0.6120            & 0.0486            & 0.9960            & 0.0053            & 0.6414            & 0.0184           \\
3\phantom{.}      & 0.5941            & 0.0635            & 1.1141            & 0.0141            & 0.8823            & 0.0177           \\
4\phantom{.}      & 0.5196            & 0.0626            & 0.9329            & 0.0055            & 0.5637            & 0.0302           \\
5\phantom{.}      & 0.4483            & 0.0436            & 1.1588            & 0.0092            & 0.4442            & 0.0173           \\
6\phantom{.}      & 0.5156            & 0.0634            & 1.2828            & 0.0052            & 1.2139            & 0.0158           \\
7\phantom{.}      & 0.6005            & 0.0620            & 0.8580            & 0.0143            & 0.5076            & 0.0344           \\
8\phantom{.}      & 0.6382            & 0.0479            & 1.1471            & 0.0053            & 0.6883            & 0.0176           \\
9\phantom{.}      & 0.5753            & 0.0602            & 0.8094            & 0.0147            & 1.0031            & 0.0175           \\
\midrule
\bottomrule
\end{tabular}
\caption{Zusammenstellung der ermittelten Absorptionskoeffizienten mit Fehlern.}
\label{Ergebnis}
\end{table}
\clearpage
\subsection{Klassifizierung der Würfel}
Da bekannt ist, dass Würfel 1 aus Messing und Würfel 2 aus Blei besteht und außerdem, dass
die Elementarwürfel von Würfel 3 ebenfalls aus Messing oder Blei bestehen, sollen nun
die Werte $\mu_\text{Messing}$ und $\mu_\text{Blei}$ gefunden werden, mit denen es möglich ist,
die Zusammensetzung von Würfel 3 zu bestimmen.
Dazu werden die Mittelwerte aus $\vec{\mu}_1$ und $\vec{\mu}_2$ bestimmt. Es folgt
\begin{align}
\mu_\text{Messing}&=(0.57 \pm 0.06) \text{ cm}^{-1}\\
\mu_\text{Blei}&=(1.05 \pm 0.15) \text{ cm}^{-1} \quad .
\end{align}
Vergleicht man diese Werte mit $\vec{\mu}_3$, so lässt sich die in Tabelle \ref{Zusammensetzung}
zu sehende Zusammensetzung für Würfel 3 vermuten.
\begin{table}[h]
\centering
\begin{tabular}{cccccccccc}
\toprule \midrule
Elementarwürfel 	& 1&2&3&4&5&6&7&8&9 \\
\midrule
	Material&CuZn &CuZn &Pb	&CuZn &CuZn &Pb &CuZn &CuZn & Pb\\
\midrule
\bottomrule
\end{tabular}
\caption{Vermutung über die Zusammensetzung des Würfels 3, ausgehend von den bisher ermittelten
Werten $\mu_\text{Blei}$, $\mu_\text{Messing}$ und $\vec{\mu}_3$.}
\label{Zusammensetzung}
\end{table}
\clearpage
