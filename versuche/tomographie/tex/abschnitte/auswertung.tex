
% ==================================================
%	Auswertung
% ==================================================

\section{Auswertung}
Für die Berechnungen werden nur die Counts im Bereich von Kanal 257 bis Kanal 315 verwand, da in 
diesem Bereich der Gammastrahlungspeak liegt, der aus dem Teil der Strahlung stammt, welcher 
das Medium direkt durchdringt und primär nicht aus anderen Streueffekten stammt. Um die restlichen 
Streubeiträge zu korrigieren wird eine Korrektur des Spektrums im relevanten Bereich vorgenommen. 
Dazu wird eine Stufenfunktion von Kanal 220 bis 257 und von 315 bis 1023 (letzter Kanal) mit dem 
Sprung beim Kanal mit den maximalen Counts an die Spektren gefittet und durch Faltung mit einer 
Gaußfunktion geglättet. Die resultierende Untergrundabschätzung kann dann durch
\begin{equation}
\text{erfc}_{a,b,c,d}(x):=\frac{a-c}{2} \left(1-\text{erf}(d(x-b))\right)+c
\end{equation}
parametrisiert werden. Dabei sind $a$ und $c$ die Grenzwerte am linken und rechten Rand 
(entsprechend der Stufenfunktion), $b$ der Kanal mit maximalen Counts und $d$ ein Maß für die 
Breite der Verteilung, $\text{erf}$ ist die Gaußsche Fehlerfunktion. Die Abbildungen aller 
Spektren und die gefitteten Untergrundfunktionen sind im Anhang zu sehen. In den Tabellen 
\ref{tab:0}, \ref{tab:1}, \ref{tab:2} und \ref{tab:3} sind die Messzeiten mit den zugehörigen 
Counts im relevanten Bereich zu sehen. Die Messung ohne Würfel ergibt $T_\text{init}=45.14$ s mit 
$C_\text{init}=10236$ Counts.
\begin{table}
\centering
\begin{tabular}{ccc}
\toprule \midrule
Projektion &$T/$s & $C$ \\
\midrule
2.00              & 21.96             & \phantom{0}4779.30\\
8.00              & 25.42             & \phantom{0}5423.27\\
9.00              & 58.70             & 12539.47         \\
\midrule
\bottomrule
\end{tabular}
\caption{Korrigierte Messwertpaare von Messzeit $T$ und Counts $C$ für die vermessenen 
Projektionen zu 
Würfel 0.} \label{tab:0}

\begin{tabular}{ccc}
\toprule \midrule
Projektion &$T/$s & $C$ \\
\midrule
32.000            & 0.279             & \phantom{0}4.000  & \phantom{0}4.000  & 5.591            \\
39.500            & 0.345             & \phantom{0}6.012  & \phantom{0}6.000  & 5.585            \\
50.500            & 0.441             & \phantom{0}9.580  & \phantom{0}8.000  & 5.109            \\
53.000            & 0.462             & 10.482            & 10.000            & 5.461            \\
68.500            & 0.598             & 16.677            & 16.000            & 5.476            \\
79.000            & 0.689             & 21.302            & 20.000            & 5.417            \\
83.500            & 0.729             & 23.345            & 22.000            & 5.427            \\
\midrule
\bottomrule
\end{tabular}
\caption{Korrigierte Messwertpaare von Messzeit $T$ und Counts $C$ für die vermessenen 
Projektionen zu 
Würfel 1. Für diesen Würfel wurde die Projektion 2 nicht gemessen.} \label{tab:1}
\end{table}

\begin{table}
\centering
\begin{tabular}{ccc}
\toprule
\midrule
Projektion &$T/$s & $C$ \\
\midrule
\phantom{0}1\phantom{.} & \phantom{0}94.160 & 1198.704         \\
\phantom{0}2\phantom{.} & 195.640           & 1410.985         \\
\phantom{0}3\phantom{.} & 147.620           & 1170.847         \\
\phantom{0}4\phantom{.} & 115.560           & 1215.972         \\
\phantom{0}5\phantom{.} & 176.720           & 1316.292         \\
\phantom{0}6\phantom{.} & 144.620           & 1221.793         \\
\phantom{0}7\phantom{.} & \phantom{0}87.180 & 1250.064         \\
\phantom{0}8\phantom{.} & 428.820           & 1522.417         \\
\phantom{0}9\phantom{.} & 595.780           & 4119.903         \\
10\phantom{.}     & 152.200           & 1352.362         \\
11\phantom{.}     & 415.340           & 1498.667         \\
12\phantom{.}     & 175.040           & 2065.847         \\
\midrule
\bottomrule
\end{tabular}
\caption{Korrigierte Messwertpaare von Messzeit $T$ und Counts $C$ für die vermessenen 
Projektionen zu 
Würfel 2.} \label{tab:2}

\begin{tabular}{ccc}
\toprule
\midrule
Projektion &$T/$s & $C$ \\
\midrule
\phantom{0}1.00   & \phantom{0}508.10 & 10538.60         \\
\phantom{0}2.00   & \phantom{0}543.50 & 10872.07         \\
\phantom{0}3.00   & \phantom{0}585.58 & 12629.85         \\
\phantom{0}4.00   & \phantom{0}317.48 & 14410.25         \\
\phantom{0}5.00   & \phantom{0}353.20 & 12534.86         \\
\phantom{0}6.00   & 1047.18           & 10416.85         \\
\phantom{0}7.00   & \phantom{0}210.80 & 11614.97         \\
\phantom{0}8.00   & \phantom{0}628.40 & 10524.62         \\
\phantom{0}9.00   & \phantom{0}716.74 & 10686.69         \\
10.00             & \phantom{0}637.78 & 10227.25         \\
11.00             & \phantom{0}910.66 & 10327.94         \\
12.00             & \phantom{0}224.54 & 10732.77         \\
\midrule
\bottomrule
\end{tabular}
\caption{Korrigierte Messwertpaare von Messzeit $T$ und Counts $C$ für die vermessenen 
Projektionen zu 
Würfel 3.} \label{tab:3}
\end{table}

Im nächsten Schritt werden die Intensitäten $I_{j,i}:=T_{j,i}/C_{j,i}$ der $i$-ten Projektion des 
$j$-ten 
Würfels aus den Messzeiten $T_{j,i}$ und Counts $C_{j,i}$ der $i$-ten Projektion des 
$j$-ten Würfels berechnet.\\
Da in die Abschwächung der Intensität bei Durchtritt durch Materie exponentiell zu dem Produkt 
aus Absorptionskoeffizient und Materialdicke ist,\footnote{Diese Gleichung ist für $I$ und 
$\vec{\mu}$ für die Würfel 1,2 und 3 gültig.}
\begin{equation}
I_(j) \propto \exp\left( (d A \vec{\mu})_j + F_\text{Mantel,j} \right)=
\exp\left( (d A \vec{\mu})_j \right) \exp\left( F_\text{Mantel,j} \right) \quad ,
\end{equation} 
kann der Effekt $F_\text{Mantel,j}$ des Aluminiummantels einfach 
korrigiert werden, indem an jede Intensität der 1,2,3,4,5 oder 6 Projektion der Faktor 
$I_\text{init}/I_{0,2}$, an die 7,9,10 und 12 Projektionen der Faktor $I_\text{init}/I_{0,9}$ und 
an die 8 und 11 Projektion der Faktor $I_\text{init}/I_{0,8}$ multipliziert wird. Die 
resultierenden Intensitäten $J_1$, $J_2$ und $J_3$ sind in Tabelle \ref{tab:korr} zu sehen.
\begin{table}
\centering
\begin{tabular}{cccc}
\toprule
\midrule
Projektion &	$J_1$ & $J_2$ & $J_3$ \\
\midrule
\phantom{0}1\phantom{.} & 37.854            & 13.514            & 22.017           \\
\phantom{0}2\phantom{.} & \phantom{0}0.000  & \phantom{0}7.515  & 20.843           \\
\phantom{0}3\phantom{.} & 38.403            & \phantom{0}8.264  & 22.472           \\
\phantom{0}4\phantom{.} & 37.999            & 10.964            & 47.292           \\
\phantom{0}5\phantom{.} & 37.734            & \phantom{0}7.761  & 36.977           \\
\phantom{0}6\phantom{.} & 38.981            & \phantom{0}8.803  & 10.365           \\
\phantom{0}7\phantom{.} & 45.676            & 14.940            & 57.410           \\
\phantom{0}8\phantom{.} & 22.481            & \phantom{0}3.769  & 17.779           \\
\phantom{0}9\phantom{.} & 44.270            & \phantom{0}7.350  & 15.848           \\
10\phantom{.}     & 50.458            & \phantom{0}9.432  & 17.022           \\
11\phantom{.}     & 23.317            & \phantom{0}3.830  & 12.039           \\
12\phantom{.}     & 47.914            & 12.544            & 50.804           \\
\midrule
\bottomrule
\end{tabular}
\caption{Die um die Effekte des Aluminiummantels korrigierten Intensitäten.} \label{tab:korr}
\end{table}