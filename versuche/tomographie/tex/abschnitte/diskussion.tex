
% ==================================================
%	Diskussion
% ==================================================

\section{Diskussion}
Wie in den Tabellen \ref{tab:Koeff} und \ref{Zusammensetzung} zu sehen ist, ist es mit dem 
durchgeführten Tomographieverfahren 
möglich den Absorptionskoeffizienten eines bekannten Materials zu bestimmen und die 
Zusammensetzung eines Würfels über die Absorptionskoeffizienten zu ermitteln. In der Literatur 
lassen sich die Werte $\mu_\text{Blei}=1.415\text{ cm}^{-1}$ und $\mu_\text{Messing}=0.645\text{ 
cm}^{-1}$ finden \cite{Internet}. Die in dieser Versuchsdurchführung ermittelten Werte enthalten 
die Literaturwerte zwar nicht in ihren erwarteten Fehlerbereichen, sie weichen aber nur 
geringfügig davon ab, sodass auch mit den Literaturwerten die gleiche Zusammensetzung für Würfel 
3 vermutet würde. Als mögliche Fehlerquelle lässt sich insbesondere nennen, dass es in der 
experimentellen Durchführung schwierig war die Projektionen 7,9,10 und 12 genau einzustellen ohne 
den Strahlengang zu sehen. Außerdem wurde in der Auswertung der Messergebnisse angenommen, dass 
der Strahl keine Breite hat und bei den diagonalen Projektionen keinen der umgebenden 
Elementarwürfel streift. Diese Annahme ist berechtigt, solange der Strahldurchmesser viel 
kleiner als die Kantenlänge der Elementarwürfel ist, was bei diesem Versuch schwierig zu 
bewerten ist, da die Elementarwürfel eine Kantenlänge von nur $1$ cm haben.\\
Insgesamt lässt sich sagen, dass die hier durchgeführte Tomographiemethode trotz der genannten 
Fehlerquellen eine zuverlässige Methode ist Absorptionskoeffizienten zu bestimmen und unbekannte 
Zusammensetzungen zu untersuchen.