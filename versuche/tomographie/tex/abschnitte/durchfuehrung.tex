
% ==================================================
%	Durchführung
% ==================================================

\section{Durchführung}
In dieser Versuchsdurchführung wurde nur die mittlere horizontale Ebene der Würfel mit Kantenlänge
$3$ cm vermessen.

\begin{figure}[h]
\centering
\includegraphics[scale=0.18]{../skript/domi.jpg}
\caption{Schematische Darstellung der mittleren Würfelebene sowie des umgebenden
Aluminiummantels.}
\label{fig:Wuerfel}
\end{figure}

Vor Beginn der Messung wird die Ebene wie in Abbildung \ref{fig:Wuerfel} in neun Elementarwürfel
der Kantenlänge $d=1$ cm eingeteilt und die Projektionsrichtungen $1,2,\ldots,12$ werden
entsprechend gewählt. Später wird für die $i$-te Projektionen der Ausdruck $\sum_{j=1}^9
d_{ij}\mu_j$  relevant, wobei $d_{ij}$ der Laufweg der $i$-ten Projektion durch den $j$-ten
Elementarwürfel ist. Diese Terme können einfach durch die Schreibweise $d A\vec{\mu}$
zusammengefasst werden, wobei $\vec{\mu}:=(\mu_1,\ldots,\mu_9)^\text{T}$ und
\begin{equation}
A=
\begin{pmatrix}
0	&0	&0	&0	&0	&0	&1	&1	&1	\\
0	&0	&0	&1	&1	&1	&0	&0	&0	\\
1	&1	&1	&0	&0	&0	&0	&0	&0	\\
1	&0	&0	&1	&0	&0	&1	&0	&0	\\
0	&1	&0	&0	&1	&0	&0	&1	&0	\\
0	&0	&1	&0	&0	&1	&0	&0	&1	\\
0	&\sqrt{2}	&0	&\sqrt{2}	&0	&0	&0	&0	&0	\\
0	&0	&\sqrt{2}	&0	&\sqrt{2}	&0	&\sqrt{2}	&0	&0	\\
0	&0	&0	&0	&0	&\sqrt{2}	&0	&\sqrt{2}	&0	\\
0	&\sqrt{2}	&0	&0	&0	&\sqrt{2}	&0	&0	&0	\\
\sqrt{2}	&0	&0	&0	&\sqrt{2}	&0	&0	&0	&\sqrt{2}	\\
0	&0	&0	&\sqrt{2}	&0	&0	&0	&\sqrt{2}	&0
\end{pmatrix} \quad . \label{eq:Matrix}
\end{equation}
Das Cäsium-Präparat wird eingesetzt und der zu vermessene Würfel in den Strahlengang gebracht.
Die Messung geschieht mit Hilfe der Software "`winTMCA32"', welche die Daten des
Multichannelanalyzer zu einem Channel-Counts Spektrum zusammenfasst. Die Spektren und die
dazugehörigen um die Detektor-Totzeit korrigierten Messzeiten werden
gespeichert.\\
Folgende Messungen werden durchgeführt:
\begin{enumerate}
\item Eine Messung ohne Würfel.
\item Von dem Würfel 0 werden die Projektionen 2,8 und 9 gemessen.
\item Von den Würfeln 1,2 und 3 werden alle Projektionen vermessen.
\end{enumerate}
Der relevante Peak im Spektrum, welcher in der Auswertung benutzt wird, liegt zwischen den
Kanälen 257 und 315.
Nur dieser Peak wird betrachtet, weil wir davon ausgehen, dass die
$\gamma$-Strahlung in diesem Bereich im wesentlichen das Medium direkt
durchdrungen hat und nicht aus unerwünschten Streueffekten stammt.
Die restlichen Streueffekte werden in der späteren Auswertung abgeschätzt und
korrigiert.
Da eine relative Unsicherheit von maximal $0.03$ (bzw. $0.01$ bei Würfel 3)
erwünscht ist, muss darauf geachtet werden, dass im relevanten Bereich mindestens 1200 (bzw.
10000) Counts gemessen werden.
