
% ==================================================
%	Auswertung
% ==================================================

\section{Auswertung}

In diesem Kapitel sollen die Messgrößen Schichtdicke, Dispersion und Rauigkeit
der auf einem Silizium-Wafer befindlichen Polystyrolprobe aus der Messung des
Diffraktometers bestimmt werden.
Hierzu werden die Messdaten zunächst aufbereitet und anschließend wird ein Fit
mit dem Parratt-Algorithmus durchgeführt.

\subsection{Datenaufbereitung}
\label{sub:datenaufbereitung}

Die Datenaufbereitung gliedert sich in drei Schritte.

\paragraph{Diffuser Scan}
\label{par:diffuser_scan}

Zusätzlich zu dem Reflektivitätsscan wird ein diffuser Scan, welcher den Anteil
der gestreuten Intensität bestimmt, durchgeführt.
Hierbei ist der Detektor um \SI{0.1}{\degree} gegenüber dem Einfallswinkel
verschoben. Der so erhaltene Scan wird von dem Reflektivitätsscan abgezogen.
In Abbildung~\ref{fig:data} sind der Reflektivitätsscan und der diffuse Scan
dargestellt.

\begin{figure}[htpb]
  \centering
  \includegraphics[scale=1.0]{bilder/data.pdf}
  \caption{Gemessener Reflektivitätsscan zusammen mit dem diffusen Scan.}
\label{fig:data}
  \includegraphics[scale=1.0]{bilder/geometriefaktor.pdf}
  \caption{Darstellung der Messdaten mit abgezogenen diffusen Scan einmal mit
    und ohne Berücksichtigung des Geometriefaktors.}
\label{fig:geometriefaktor}
\end{figure}

\paragraph{Geometriefaktor}
\label{par:geometriefaktor}

Unterhalb eines Grenzwinkels $\alpha_\text{g}$ deckt der Röntgenstrahl eine
Fläche ab die Größer ist als die Fläche der Probe. Dies verringert die
Intensität und muss mit einem Geometriefaktor $G$ berücksichtigt werden.
Der Grenzwinkel wird dabei mit
\begin{equation}
  \alpha_\text{g} = \arcsin(\frac{d_0}{D})
\end{equation}
berechnet, wobei $d_0 = \SI{0.1}{\mm}$ die Strahlhöhe und
$D = \SI{30}{\mm}$ die Ausdehnung der Probe ist. Damit ergibt sich
ein Grenzwinkel von
\begin{equation}
  \alpha_\text{g} = \SI{0.190986}{\degree}~.
\end{equation}
Die Messwerte unterhalb des Grenzwinkels werden schließlich durch den
Geometriefaktor
\begin{equation}
  G(\alpha) = \frac{D \sin\alpha}{d_0}
\end{equation}
dividiert. In Abbildung~\ref{fig:geometriefaktor} ist die Auswirkung der
Berücksichtigung des Geometriefaktors dargestellt.
Der Geometriefaktor führt dazu, dass im Winkelbereich der Totalreflexion
eine nahezu flaches Plateau entsteht.

% \begin{figure}[htpb]
%   \centering
%   \includegraphics[scale=1.0]{bilder/geometriefaktor.pdf}
%   \caption{Darstellung der Messdaten mit abgezogenen diffusen Scan einmal mit
%     und ohne Berücksichtigung des Geometriefaktors.}
% \label{fig:geometriefaktor}
% \end{figure}

\paragraph{Normierung}
\label{par:normierung}

Um die Messdaten schließlich mit dem Parratt-Algorithmus zu vergleichen, müssen
diese noch normiert werden.
Im folgenden werden nur noch die Messwerte oberhalb von \SI{0.08}{\degree}
verwandt. Um die Lage des 1-Niveaus zu bestimmen, werden die Daten im
Intervall $\SI{0.082}{\degree} < \alpha < \SI{0.168}{\degree}$
(dem Bereich des Plateaus) betrachtet und
gemittelt. Die Daten werden nun durch den so erhaltenen Wert dividiert.

\subsection{Fit der Parameter}
\label{sub:fit_der_parameter}

Für den Fit wird ein \texttt{Python}-Skript erstellt und mit Hilfe des Moduls
\texttt{scipy.optimize} der Parratt-Algorithmus an die Daten gefittet.
Hierbei werden die Parameter der Dispersion ($n_1$, $n_2$, $n_3$),
der Rauigkeit ($\sigma_1$, $\sigma_2$) und die Schichtdicke $z_2$ variiert.
Die Startwerte werden zu
\begin{align*}
  n_1      &= n_2 = n_3 = 1 \\
  \sigma_1 &= 3 \times 10^{-10} \\
  \sigma_2 &= 6 \times 10^{-10} \\
  z_2      &= 209 \times 10^{-10}
\end{align*}
gewählt.
Für den Fit werden die Werte im Intervall
$\SI{1.0}{\degree} < \alpha < \SI{2.0}{\degree}$ verwandt.
Das Ergebnis des Fits ist in Abbildung~\ref{fig:fit} dargestellt.
Die Parameter ergeben sich zu
\begin{align*}
  n_1      &= 1 - \SI[parse-numbers = false]{\left(8.4 \pm 0.5\right) \times 10^{-6}}{} \\
  n_2      &= 1 - \SI[parse-numbers = false]{\left(4.3 \pm 0.4\right) \times 10^{-6}}{} \\
  n_3      &= 1 - \SI[parse-numbers = false]{\left(-2.90 \pm 0.18\right) \times 10^{-5}}{} \\
  \sigma_1 &= \SI[parse-numbers = false]{\left(0.0 \pm 1.7\right) \times 10^{-8}}{} \\
  \sigma_2 &= \SI[parse-numbers = false]{\left(-3.81 \pm 0.30\right) \times 10^{-10}}{} \\
  z_2      &= \SI[parse-numbers = false]{-217.7 \pm 0.5}{\angstrom}
\end{align*}

\begin{figure}[htpb]
  \centering
  \includegraphics[scale=1.0]{bilder/fit.pdf}
  \caption{Darstellung der aufbereiteten Daten mit der daran gefitteten
  Theoriekurve.}
\label{fig:fit}
\end{figure}
