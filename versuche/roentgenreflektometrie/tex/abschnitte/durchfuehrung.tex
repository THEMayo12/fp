
% ==================================================
%	Durchführung
% ==================================================

\section{Durchführung}
Bei der Durchführung des Versuches müssen zunächst Strahl und Probe einjustiert 
werden. Dies geschieht wie folgt.
\begin{enumerate}
\item Die Probe wird in das Labordiffraktometer gebracht.
\item Die Probe wird entlang der $z$-Achse aus dem Strahlengang gefahren.
\item Der Detektor wird zur Röntgenröhre ausgerichtet. (Detektor Scan)
\item Die $z$-Koordinate der Probe wird so gewählt, dass die dann gemessene 
Intensität gerade der halben Maximalintensität entspricht. (z-Scan)
\item Der Strahlengang wird parallel zur Probenoberfläche eingestellt. (Rocking 
Scan)
\item Die $z$-Koordinate der Probe wird erneut so gewählt, dass die dann 
gemessene Intensität gerade der halben Maximalintensität entspricht. (z-Scan)
\item Die Detektorröhre wird so eingestellt, dass sie bei der Totalreflexion bei 
ca. $15^\circ$ genau im reflektierten Strahl ist. (Rocking Scan)
\item Die $z$-Koordinate der Probe wird erneut so gewählt, dass die dann 
gemessene Intensität gerade der halben Maximalintensität entspricht. (z-Scan)
\item Für große Einfallswinkel muss nun noch eine Feinjustierung des Detektors 
durchgeführt werden. Dazu wird um $2\vartheta=1^\circ$ ein Rocking Scan 
durchgeführt, also einem Winkelbereich von ca. 0.45 bis 0.55, und der 
Detektorwinkel so auf 0.5 geeicht.
\end{enumerate}
Nach der Justierung des Labordiffraktometers kann nun die Vermessung der Probe 
begonnen werden. Als Scanbereich (Winkel) wird das Intervall $0^\circ$ bis 
$2.5^\circ$ mit einer Schrittweite von $0.0025^\circ$ gewählt. Pro Winkel 
wird die Intensität über einen Zeitraum von $10 \, \text{s}$ gemessen. Um die 
Fehler durch Streustrahlen zu vermindern, wird außerdem ein "´Diffuser Scan"' 
durchgeführt. Dabei wird die eben beschriebene Messung, jedoch mit einem um 
$0.1^\circ$ verschobenen Detektor, wiederholt.