
% ==================================================
%	Diskussion
% ==================================================

\section{Diskussion}

Bei einem Vergleich der in dieser Versuchsdurchführung ermittelten Werten für
die Brechungsindizes und Rauigkeiten der Materialien mit den in der Versuchsanleitung
angegeben Werten zeigen sich teilweise deutliche Abweichungen. So während der
ermittelte Brechungsindex $n_2$ noch nah am angegeben ($n_{2\text{lit}}=1-(3.5\pm0.4)\times
10^{-6}$) liegt, weicht der Brechungsindex $n_3$ des Substrat sogar qualitativ vom
Literaturwert ab, da $n_3>1$. Da die Rauigkeiten speziell von der verwendeten Probe
abhängen, ist hier ein genauer Vergleich mit Literaturwerten nicht sinnvoll, die
ermittelten Werte liegen jedoch in einer sinnvollen Größenordnung. Eine mögliche
Ursache für die teilweise starken Abweichungen von den erwarteten Werten könnte der
"`Diffuse Scan"' sein. Bei diesem wurde der Detektor vermutlich nicht weit genug
aus dem Strahl gefahren, sodass hier nicht nur Streuintensitäten sondern auch
noch ein Teil des gebeugten Strahl mit gemessen wurde.
