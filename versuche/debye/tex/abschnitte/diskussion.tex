
% ==================================================
%	Diskussion
% ==================================================

\section{Diskussion}
Wie in der Auswertung zu sehen ist, ist die Debye-Scherrer-Methode ein Verfahren, mit 
dem die Gitterkonstante und Gitterstruktur einer kubisch-kristallinen Probe durch 
Vergleich von Strukturfaktoren einfach bestimmt werden kann. In dieser 
Versuchsdurchführung waren die Beugungsringe auf den Fotostreifen nur sehr schlecht 
bis gar nicht zu erkennen, sodass insbesondere bei der zweiten Probe das Ergebnis der 
Auswertung dadurch wahrscheinlich stark von den wahren Werten abweicht. Da bei der 
ersten Probe noch vergleichsweise viele Beugungsringe erkennbar sind, ist hier vor 
allem die Bestimmung der Kristallstruktur zuverlässiger, während die wenigen Ringe 
bei der zweiten Probe entweder auf die Diamant-Struktur zurückzuführen ist, oder 
einfach auf darauf, dass die Ringe zwar vorhanden, aber nicht erkennbar sind.