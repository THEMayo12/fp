
% ==================================================
%	Theorie
% ==================================================

\section{Theorie}

\subsection{Beschreibung der Kristallstrukturen}
\label{sub:beschreibung_der_kristallstrukturen}

Ein Kristall ist eine räumlich periodische von Atomen oder Atomgruppen.
Er kann daher als Punktgitter gesehen werden, wobei jeder Punkt einem Atom oder
einer Atomgruppe, genannt Basis, zugeordnet wird. Es genügen nun drei
Basisvektoren, um das Gitter aufzuspannen. Die besondere Eigenschaft eines
Gitters ist nun, dass die Verschiebung des Gitters um einen Vektor
$\vb*{t} = n_1 \vb*{a} + n_2 \vb*{b} + n_3 \vb*{c}$ mit $n_i \in \mathbb{Z}$
das Gitter selbst wieder ergibt.
Zur Einordnung verschiedener Kristallarten wird der Einkristall bzw.\ die
Elementarzelle verwandt.
Sie ist die kleinste Volumeneinheit, die einen Kristall eindeutig festlegt.
Besitzt die Elementarzelle nur ein Atom, so wird sie primitiv genannt.
Nun existieren 14 verschiedene Gittertypen, welche entsprechend der Längen der
Gittervektoren und deren Winkel zueinander klassifiziert werden.
In diesem Versuch werden ausschließlich kubische Gitter betrachtet. Hierbei
sind alle Längen der Gittervektoren gleich lang und die Winkel betragen
\SI{90}{\degree}.
Die so in Frage kommenden Gittertypen sind
\begin{description}
  \item[Das kubische Gitter (sc):] Die Elementarzelle besitzt nur ein Basis in
    einer Ecke des Würfels.
  \item[Das kubisch-raumzentrierte Gitter (bcc):] Die Elementarzelle besitzt
    zusätzlich zum sc-Gitter eine Basis in der Mitte des Würfels. Die
    Koordinaten der Basen lauten hier
    \begin{equation}
      (0, 0, 0) \,, \quad \qty(\frac12, \frac12, \frac12)
    \end{equation}
  \item[Das kubisch-flächenzentrierte Gitter (fcc):] Die Elementarzelle besitzt
    zusätzlich zum sc-Gitter noch eine Basis in der Mitte der Würdelseiten.
    Die Koordinaten lauten hier
    \begin{equation}
      (0,0,0)\,,\quad \qty(\frac12, \frac12, 0)\,,\quad
      \qty(\frac12, 0, \frac12)\,,\quad \qty(0, \frac12, \frac12)
    \end{equation}
\end{description}

In der Natur gibt es zusätzlich Gitterstrukturen, welche aus den oben genannten
zusammengesetz sind. Die Struktur der Elementarzellen mit den entsprechenden
Koordinaten der Basen sollen im Folgenden genannt werden.
\begin{description}
  \item[Die Diamantstruktur:] Sie besteht aus zwei fcc-Gittern, welche um ein
    Viertel der Raumdiagonalen verschoben sind.
    \begin{equation}
      (0, 0, 0)\,, \qty(\frac12, \frac12, 0)\,, \qty(\frac12, 0, \frac12)\,,
      \qty(\frac14, \frac14, \frac14)\,, \qty(\frac34, \frac34, \frac14)\,,
      \qty(\frac34, \frac14, \frac34)\,, \qty(\frac14, \frac34, \frac34)
    \end{equation}
  \item[Die Zinkblende-Struktur:] Sie entspricht der Diamantstruktur worin die
    beiden fcc-Gitter mit unterschiedlichen Basen besetzt sind.
  \item[Die Steinsalz-Struktur:] Sie besteht aus zwei fcc-Gittern, welche um
    die Hälfte der Raumdiagonalen versetzt sind. Zudem bestehen die fcc-Gitter
    aus zwei verschiedenen Basisarten $A$ und $B$.
    \begin{align*}
      A&:\quad (0, 0, 0)\,, \qty(\frac12, 0, \frac12)\,,
      \qty(0, \frac12, \frac12) \\
      B&:\quad \qty(\frac12, \frac12, \frac12)\,, \qty(1, 1, \frac12)\,,
      \qty(1, \frac12, 1)\,, \qty(\frac12, 1, 1)
    \end{align*}
  \item[Die Cäsiumchlorid-Struktur:] Sie ist aus zwei sc-Gittern mit
    verschiedenen Basisarten $A$ und $B$ aufgebaut,
    welche um eine halbe Raumdiagonale verschoben ist.
    \begin{align*}
      A&:\quad (0, 0, 0) \\
      B&:\quad \qty(\frac12, \frac12, \frac12)
    \end{align*}
  \item[Die Fluorit-Struktur:] Sie tritt bei Verbindungen des Typs $AB_2$ auf
    und besteht aus drei fcc-Gittern, welche gegeneinander um 1/4 bzw. 3/4 der
    Würdelidagonalen verschoben sind.
    \begin{align*}
      A&: (0, 0, 0)\,, \qty(\frac12, \frac12, 0)\,, \qty(\frac12, 0, \frac12)
      \qty(0, \frac12, \frac12) \\
      B&:\left\{
      \begin{aligned}
        &\qty(\frac14, \frac14, \frac14)\,, \qty(\frac34, \frac34, \frac14)\,,
        \qty(\frac34, \frac14, \frac34)\,, \qty(\frac14, \frac34, \frac34) \\
        &\qty(\frac34, \frac34, \frac34)\,, \qty(\frac14, \frac14, \frac34)\,,
        \qty(\frac14, \frac34, \frac14)\,, \qty(\frac34, \frac14, \frac14)
      \end{aligned}
      \right.
    \end{align*}
\end{description}

\subsection{Netzebenenabstand}
\label{sub:netzebenenabstand}

Eine Netzebene im Kristall ist eine Ebene, in welcher Schwerpunkte von Atomen
liegen. Die Gesamtheit der Netzebenen, die zu einer vorgegebenen Netzebene
parallel liegen, werden Netzebenenschar bezeichnet. Die Lage der
Netzebenenschar in den von den Gittervektoren aufgespannten Basis wird mit den
Miller'schen Indizes beschrieben.
Zur Bestimmung der Miller'schen Indizes betrachte man die Schnittpunkte einer
Netzebene mit den Achsen der von den Gittervektoren aufgespannten Basis.
Die reziproken Werte, so multipliziert, dass Brüche verschwinden, ergeben die
Miller'schen Indizes. Negative Zahlen werden hierbei mit einem Balken über der
Zahl angegeben und wird eine Achse nicht geschnitten, so lautet deren Index 0.
Der Abstand der Netzebenen mit den Miller'schen Indizes $(h, k, l)$
lautet für das kubisch Gitter mit der Gitterkonstanten $a$
\begin{equation}
  d = \frac{a}{\sqrt{h^2 + k^2 + l^2}}~.
  \label{eq:gitterabstand}
\end{equation}

\subsection{Beugung von Röntgenstrahlen an Kristallen}
\label{sub:beugung_von_r_ntgenstrahlen_an_kristallen}

Die Wechselwirkung von Röntgenstrahlen einem Kristall kann als klassischer
Streuprozess interpretiert werden. Das heißt, die geladenen Teilchen werden
unter dem Einfluss des elektrischen Feldes zum schwingen angeregt und
emittieren ebenfalls aufgrund der Beschleunigung wieder elektromagnetische
Strahlung. Entscheidend ist nun, dass die Streuzentren periodisch angeordnet
sind und somit in bestimmten Raumrichtungen Maxima und Minima der gestreuten
Intensität auftritt. Aufgrund der Zusammenhänge zwischen der Extrema und der
Lage der Streuzentren, können nun Rückschlüsse auf die Kristallstruktur gezogen
werden. Da ein zum schwingen angeregtes Teilchen kann als Hertz'scher Dipol
gesehen werden. Dieser besitzt eine $1/m^2$-Abhängigkeit, sodass die schweren
Atomkerne an der Beugung praktisch nicht beteiligt sind und im folgenden nur
die Elektronen eine Rolle spielen.

\subsubsection{Der Atomfaktor}
\label{ssub:der_atomfaktor}

Bei der Streuung an einem Atom, muss berücksichtigt werden, dass das Elektron
nicht punktförmig auftritt, sondern mit einer Ladungsverteilung $\rho(\vb*{r})$
um der Kern "`verschmiert"' ist, weshalb die Intensität der Röntgenstrahlung
von der Intensität der Beugung an einer Punktladung abweichen wird.
Diese Abweichung wird durch
\begin{equation}
  f^2 \coloneqq \frac{I_a}{I_e}
\end{equation}
definiert, wobei $I_a$ die Intensität von der Streuung an einem Atom und
$I_e$ die Intensität von der Streuung an einem Elektron ist.
Der Formfaktor kann letztendlich durch eine Fourier-Transformation der
Ladungsverteilung entsprechend
\begin{equation}
  f = \int\limits_\text{Hülle} \dd[3]{\vb*{r}}\ \rho(\vb*{r})
  \ \exp\big[-2\uppi\mathrm{i}\ (\vb*{k}-\vb*{k}_0)\big]
\end{equation}
bestimmt werden, wobei $\vb*{k}_0$ der Wellenzahlvektor der einfallenden Welle
und $\vb*{k}$ der Wellenzahlvektor der gestreuten Wellen ist.

\subsubsection{Bragg-Bedingung}
\label{ssub:bragg_bedingung}

Die Streuung der Röntgenstrahlen findet nicht nur an einer Netzebene statt,
sondern an der gesamten Netzebenenschar. Somit müssen auch die
Interferenzeffekte an allen Ebenen der Netzebenenschar berücksichtigt werden.
Die Bedingung wann an zwei benachbarten Ebenen konstruktive Interferenz
auftritt liefert die Bragg-Bedingung
\begin{equation}
  n \lambda = 2d\ \sin\theta~,
\end{equation}
wobei $\lambda$ die Wellenlänge der Röntgenstrahlung, $d$ der
Netzebenenabstand, $theta$ der Winkel zwischen dem einfallenden Wellenvektor
und der Netzebene und $n \in \mathbb{R}$ ist.
Diese Bedingung ist äquivalent zu
\begin{equation}
  \vb*{k} - \vb*{k}_0 = \vb*{g}~,
\end{equation}
wobei
\begin{equation}
  \vb*{g} = h\vb{A} + k\vb*{B} + l\vb*{C}
\end{equation}
der reziproke Gittervektor mit der reziproken Basis
$\{\vb*{A}, \vb*{B}, \vb*{C}\}$ und den Miller'schen Indizes $(h, k, l)$ ist.
Die Basis des reziproken Gittervektors ist mit den Gittervektoren durch
die Bedingung
\begin{equation}
  \vb*{a}_i\vb*{A}_j = 2\uppi\ \delta_{ij}
\end{equation}
miteinander verknüpft.

\subsubsection{Der Strukturfaktor}
\label{ssub:der_strukturfaktor}

Der Strukturfaktor erweitert den Atomfaktor insofern, dass nicht nur die
Ladungsverteilung eines Atoms, sondern zusätzlich auch die
Ladungsverteilung in der Elementarzelle betrachtet wird.
Da die Ladungsverteilung in der Elementarzelle diskret ist, reicht hier
allerdings eine Summe, womit für den Strukturfaktor
\begin{equation}
  S = \sum_i f_i\ \exp\big[-2\uppi\mathrm{i}\ (x_ih + y_ik + z_il)\big]
  \label{eq:strukturfaktor}
\end{equation}
folgt, wobei $f_i$ die Atomfaktoren des $i$-ten Atoms und $(x_i, y_i, z_i)$ die
Position des $i$-ten Atoms in der Basis der Gittervektoren sind.
