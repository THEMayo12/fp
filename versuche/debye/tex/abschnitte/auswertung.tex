
% ==================================================
%	Auswertung
% ==================================================

\section{Auswertung}
Auf dem als Ergebnis der experimentellen Durchführung erhaltenen Fotostreifen sind 
jeweils einige Beugungsringe schwach zu erkennen. Zunächst wurden die Radien dieser 
Ringe gemessen und der Beugungswinkel $\vartheta$ bestimmt. Die Messwerte sind in den 
Tabellen \ref{tab:1} und \ref{tab:2} zu sehen.

\begin{table}[h]
\centering
\begin{tabular}{ccccc}
\toprule
\midrule
 $r$/mm & $\vartheta$ &$s_\text{exp}$& $s_\text{th}$& $a$/$\AA$ \\
\midrule
32.000            & 0.279             & \phantom{0}4.000  & \phantom{0}4.000  & 5.591            \\
39.500            & 0.345             & \phantom{0}6.012  & \phantom{0}6.000  & 5.585            \\
50.500            & 0.441             & \phantom{0}9.580  & \phantom{0}8.000  & 5.109            \\
53.000            & 0.462             & 10.482            & 10.000            & 5.461            \\
68.500            & 0.598             & 16.677            & 16.000            & 5.476            \\
79.000            & 0.689             & 21.302            & 20.000            & 5.417            \\
83.500            & 0.729             & 23.345            & 22.000            & 5.427            \\
\midrule
\bottomrule
\end{tabular}
\caption{Messwerte und Vergleichswerte zur ersten Probe. Dabei ist $r$ der Radius 
eines Ringes, $\vartheta$ der Beugungswinkel und $a$ die ermittelte Gitterkonstante.}
\label{tab:1}
\end{table}
\begin{table}[h]
\centering
\begin{tabular}{ccccc}
\toprule
\midrule
$r$/mm & $\vartheta$ &$s_\text{exp}$& $s_\text{th}$& $a$/$\AA$ \\
\midrule
32.000            & 0.279             & \phantom{0}4.000  & \phantom{0}4.000  & 5.591            \\
39.500            & 0.345             & \phantom{0}6.012  & \phantom{0}6.000  & 5.585            \\
50.500            & 0.441             & \phantom{0}9.580  & \phantom{0}8.000  & 5.109            \\
53.000            & 0.462             & 10.482            & 10.000            & 5.461            \\
68.500            & 0.598             & 16.677            & 16.000            & 5.476            \\
79.000            & 0.689             & 21.302            & 20.000            & 5.417            \\
83.500            & 0.729             & 23.345            & 22.000            & 5.427            \\
\midrule
\bottomrule
\end{tabular}
\caption{Messwerte und Vergleichswerte zur zweiten Probe. Dabei ist $r$ der Radius 
eines Ringes, $\vartheta$ der Beugungswinkel und $a$ die ermittelte Gitterkonstante.}
\label{tab:2}
\end{table}

\subsection{Bestimmung der Gitterstruktur und der Gitterkonstanten}

Die Werte wurden wie folgt berechnet. Zu einem Ringradius $r$ und der Fotospule vom 
Radius $R=57.4 \text{ cm}$ lässt sich der Beugungswinkel $\vartheta=r/2R$ ermitteln. 
Definiere
\begin{equation}
s_i^\text{exp}:=4 \frac{\sin^2(\vartheta_i)}{\sin^2(\vartheta_0)} \quad ,
\end{equation}
wobei $\vartheta_0$ der kleinste gemessene Wert ist. Diese Definition leitet sich 
aus
\begin{equation}
\lambda=2\sin(\vartheta) d \label{eq}
\end{equation}
ab, wobei $\lambda$ die Wellenlänge des verwendeten Röntgenlichtes ist, $\vartheta$ der Beugungswinkel und $d=a/\sqrt{h^2+k^2+l^2}$ der Netzebenenabstand der $(h,k,l)$-
Netzebenenschar zu einem kubischen Gitter mit Gitterkonstante $a$. Der Faktor $4$ rührt daher, dass angenommen wird, dass der erste Ring zur 
Beugungsebene (2,0,0) gehört. Dies kann nur damit begründet werden, dass durch diese
Wahl die Messwerte sinnvoll sind. 
Definiere weiter $s^\text{th}_i:=h_i^2+k_i^2+l_i^2$, wobei $(h_i,k_i,l_i)$ die 
Millerindizes zum $i$-ten Beugungsring sind.\\
Im Falle einer perfekten Messung sollte also $s_i^\text{th}=s_i^\text{exp}$ gelten.
\\
Anhand der $s_i^\text{exp}$ wurde nun abgeschätzt, zu welchen Netzebenenscharen die 
Beugungsringe gehören und mit den bekannten Gitterstrukturen verglichen. Eine 
theoretische Betrachtung der Strukturfaktoren eines bcc-Gitters ergibt zum Beispiel 
Reflexe bei $s^\text{th} \in \{ 2,4,8,10,16,20,22,... \}$, für ein Diamantgitter 
ergeben sich lediglich Reflexe bei $s^\text{th}\in \{ 4,11,... \}$. Da dies sehr gut 
zu den Beobachteten $s^\text{exp}$ passt, wird nun angenommen, dass es sich bei der 
ersten Probe um eine bcc- oder NaCl- und bei der zweiten Probe um eine 
Diamantstruktur handelt. Wobei die NaCl-Struktur der bcc-Struktur für kleine 
Millerindizes sehr ähnlich ist.
\\
Schließlich können die Gitterkonstanten über
\begin{equation}
a=\frac{\sqrt{s^\text{th}} \lambda}{2 \sin(\vartheta)}
\end{equation}
(vgl. \eqref{eq}) bestimmt werden, wobei $(s^\text{th},\vartheta)$ für ein 
zusammengehörendes Paar von Netzebene und Beugungswinkel steht, außerdem steht 
$\lambda$ wieder für die Wellenlänge der Röntgenstrahlung.

\subsection{Korrektur der Gitterkonstanten und Bestimmung der Probenmaterialien}
Um eine möglichst gute Schätzung für die Gitterkonstante $a$ zu bekommen, wird nun 
$a$ gegen $\cos^2(\vartheta)$ aufgetragen. Die Extrapolation auf 
$\cos^2(\vartheta)=0$ ergibt dann den besten Wert für die Gitterkonstante. In 
den Abbildungen \ref{fig:1} und \ref{fig:2} sind die Gitterkonstanten gegen 
$\cos^2(\vartheta)$ aufgetragen und es wurde eine lineare Regression durchgeführt.
\begin{figure}[h]
\centering
\includegraphics[scale=0.8]{bilder/fig1.pdf}
\caption{Extrapolation der Gitterkonstanten für die erste Probe.}
\label{fig:1}
\end{figure}
\begin{figure}[h]
\centering
\includegraphics[scale=0.8]{bilder/fig2.pdf}
\caption{Extrapolation der Gitterkonstanten für die zweite Probe.}
\label{fig:2}
\end{figure}
Es ergeben sich die Regressionsgeraden $(t:=\cos^2(\vartheta))$
\begin{equation}
G_1(t) = \SI[parse-numbers = false]{\left(2.3 \pm 5.0\right) \times 10^{-11}}{}\, \cdot \,t\, + \SI[parse-numbers = false]{\left(5.3 \pm 0.4\right) \times 10^{-10}}{\meter}
\end{equation}
sowie
\begin{equation}
G_2(t) = \SI[parse-numbers = false]{\left(-1.10880741251 \pm 0\right) \times 10^{-10}}{}\, \cdot \,t\, + \SI[parse-numbers = false]{\left(6.61540296504 \pm 0\right) \times 10^{-10}}{\meter}
\end{equation}

Als beste Werte für die Gitterkonstanten ergeben sich so
\begin{align*}
a_{\text{bcc}}&=(5.3 \pm 0.4)\, \AA \\
a_\text{Diamant}&=(6.61 \pm 0 )\, \AA \quad ,
\end{align*}
wobei der Fehler von $a_2$ wegen der zu geringen Datenmenge in der Rechnung zwar 
verschwindet, jedoch in der gleichen Größenordnung wie der Fehler von $a_1$ zu 
erwarten ist.
Eine NaCl-Struktur mit $a_\text{NaCl}=5.6 \, \AA$ weist NaCl auf, sodass wir 
annehmen, dass es sich der ersten Probe um diesen Stoff handelt.
