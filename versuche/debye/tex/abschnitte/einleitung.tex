
% ==================================================
%	Einleitung
% ==================================================

\section{Einleitung}

In diesem Versuch sollen die Kristallstrukturen von zwei Proben mit Hilfe
des \textsc{Debye-Scherrer}-Verfahrens bestimmt werden. Der größte Teil der
festen Materie besitzt eine Kristallstruktur. Die meisten von ihnen liegen
zudem in einem polykristallinen Zustand vor, d.h sie bestehen aus vielen
kleineren Teilkristallen, den Kristalliten, welche eine periodische
Anordnung von Atomen besitzt. Die Kristalliten sind in dem Festkörper
statistisch in alle Raumrichtungen verteilt und damit isotrop.
Die makroskopischen Eigenschaften des Festkörpers beruhen somit auf der
Anisotropie der physikalischen Eigenschaften der Kristalliten oder den
Einkristallen.
