
% ==================================================
%	Diskussion
% ==================================================

\section{Diskussion}

Wie in Abbildung~\ref{fig:C_V} zu sehen ist, schwanken die Werte sehr stark.
Der Grund hierfür kann sein, dass die Zeitdifferenzen zwischen den Messungen
zum Teil stark schwanken. So schwanken die Werte von \SI{193}{\second} bis
\SI{386}{\second} mit Ausnahme der ersten Messung.  Ebenfalls konnte die
Temperaturdifferenz von \SI{7}{\kelvin} bis \SI{11}{\kelvin} nicht immer genau
eingehalten werden.  Hinzu kommt, dass es nicht gelungen ist die Temperatur des
Gehäuses synkron mit der Temperatur der Probe steigen zu lassen, sodass hier
ein Wärmeaustausch stattfinden konnte, wie in Abbildung~\ref{fig:temp} zu
sehen ist.  Eine weitere Fehler ist die
Wärmestrahlung, welche nie ganz ausgeschlossen werden kann.  Das Endresultat
der Debye-Temperatur von
\begin{equation}
  \theta_D = \SI[parse-numbers = false]{295 \pm 12}{\kelvin}
\end{equation}
weicht so ebenfalls stark von dem Literaturwert
$\theta_D = \SI{343}{\kelvin}$ ab~\cite{DEB}.
