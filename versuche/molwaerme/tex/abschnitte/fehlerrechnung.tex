%
% ========================================
%	Fehlerrechung
% ========================================
\subsection{Fehlerrechnung}
%
Grundsätzlich ist jedes Messergebnis mit einem Fehler behaftet. Das gemessene
Ergebnis weicht vom idealen, wahren Ergebnis ab. In diesem Versuch ist dabei
eine ausführliche Fehlerrechnung notwendig.  Die dafür benötigten Formel werden
nun kurz dargestellt.
\begin{description}
  \item[arithmetisches Mittel:] Bei Messwerten mit statistischen Fehlern
    entspricht das arithmetische Mittel dem empirischen Erwartungswert.
    Es ist definiert als
  \begin{equation}
    \langle x \rangle = \frac{1}{n} \sum_{i = 1}^n x_i
    \label{eq:Mittelwert}\\
    % \big(x_i : \text{$i$-te Messwert}\,,
    % \quad n : \text{Anzahl der Messwerte} \big)~. \notag
  \end{equation}
  mit dem $i$-ten Messwert $x_i$ und der Anzahl der Messwerte $n$.
  %
  \item[mittlerer Fehler des Mittelwertes:] Je größer die Zahl der Messungen
    ist, desto vertrauenswürdiger ist der arithmetische Mittelwert.
    Dies wird mit dem mittleren Fehler des Mittelwertes $\sigma_m$ ausgedrückt.
    Er ist definiert als
  \begin{equation}
    \sigma_m = \frac{\sigma}{\sqrt{n}} =
    \sqrt{\sum_{i = 1}^n \frac{(x_i - \langle x \rangle)^2}{n(n-1)}}~.
    \label{eq:absloluter_Fehler}
  \end{equation}
  Zudem entspricht $\sigma_m$ dem absoluten Fehler $\Delta x$ einer
  fehlerbehafteten Größe $x$.
  %
  \item[Gauß'sche Fehlerfortpflanzung:] Besteht eine Messgröße $y$ aus mehreren
    fehlerbehafteten Größen $x_i$, so muss eine Fehlerfortpflanzung
    durchgeführt werden. Sie hat die Form
  \begin{equation}
    \Delta y =
    \sqrt{\sum_{i = 1}^n \left(\pdv{y}{x_i} \Delta x_i  \right)^2}~,
    \label{Gauss}
  \end{equation}
  % \begin{equation*}
  % \left(\,
  %     \begin{aligned}
  %     y &:
  %     \text{zusammengesetzte Größe mit mehreren fehlerbehafteten Größen}~, \\
  %     \Delta y &: \text{absoluter Fehler von $y$}~, \\
  %     \Delta x_i &: \text{absolute Fehler der einzelnen Größen}
  %     \end{aligned}
  % \,\right)~.
  % \end{equation*}
  wobei $y$ die zusammengesetzte Größe mit mehreren fehlerbehafteten Größen
  ist, $\Delta y$ der absolute Fehler von $y$ und $\Delta x_i$ der absolute
  Fehler der einzelnen Größen.
  %
\end{description}
%
Die Angabe einer Messgröße besitzt damit die Form
\begin{equation}
  y = \langle y \rangle \pm \Delta y~. \notag
\end{equation}
%
%
\clearpage
