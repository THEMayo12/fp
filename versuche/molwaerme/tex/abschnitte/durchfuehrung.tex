
% ==================================================
%	Durchführung
% ==================================================

\section{Durchführung}
Bevor die Messung beginnt, muss die Probe auf ca. $\SI{80}{\kelvin}$ 
abgekühlt werden. Dazu wird das Gehäuse evakuiert und mit Helium gefüllt, um eine 
Wärmeleitung zu ermöglichen und gleichzeitig 
Kondensationen am Rezipienten zu vermeiden. Das Dewargefäß wird mit 
flüssigem Stickstoff gefüllt. Nach ca. $\SI{1}{\hour}$ ist der Rezipient auf 
$\SI{82.93}{\kelvin}$ abgekühlt. Das Gehäuse wird erneut evakuiert. Bei 
Heizstrom $I$ und -spannung $U$ von ca. $I=\SI{170}{\milli \ampere}$ bzw. $U=\SI{18}{V}$ wird die Probe geheizt.
Aufgenommen wird die Zeit $t$ seit Beginn der Messung und der Widerstand $R$ 
am Rezipienten sowie die genauen Werte für den Heizstrom $I$ und die 
Heizspannung $U$ am Rezipienten. Um Fehler durch die Wärmestrahlung des 
Rezipienten zu minimieren muss darauf geachtet werden, dass das Gehäuse 
die gleiche Temperatur wie der Rezipient hat.

Die zugeführte Energie $E$ im Zeitintervall $\Delta t$ ist dann
\begin{equation}
E = IU \Delta t  \quad ,
\end{equation}
die Molwärme bei der Temperatur $T$ ist 
\begin{equation}
C_p (T) = \frac{E}{\Delta T} \frac{M}{m} \quad .
\end{equation}
Da die Messung bei konstantem Druck $p$ geschieht, muss mit der Korrekturformel 
\eqref{eq:theorie:korrekturformel} der Wert für $C_V$ bestimmt werden.

