
% ==================================================
%	Durchführung
% ==================================================

\section{Durchführung}
Bevor die Messung beginnt, muss die Probe auf ca. $\SI{80}{\kelvin}$
abgekühlt werden. Dazu wird der Rezipient evakuiert und mit Helium gefüllt. 
Das Dewargefäß wird mit
flüssigem Stickstoff gefüllt. Nach ca. $\SI{1}{\hour}$ sind der Zylinder und 
die Probe auf 
$\SI{82.93}{\kelvin}$ abgekühlt. Der Rezipient wird erneut evakuiert. Bei
Heizstrom $I$ und -spannung $U$ von ca. $I=\SI{170}{\milli \ampere}$ bzw. $U=\SI{18}{V}$ wird die Probe geheizt.
Aufgenommen wird die Zeit $t$ seit Beginn der Messung und der Widerstand $R$
an der Probe sowie die genauen Werte für den Heizstrom $I$ und die
Heizspannung $U$ an der Proben-Heizwicklung. Um Fehler durch die Wärmestrahlung der Probe zu minimieren muss darauf geachtet werden, dass der Zylinder 
die gleiche Temperatur wie die Probe hat.

Die zugeführte Energie $E$ im Zeitintervall $\Delta t$ ist dann
\begin{equation}
E = IU \Delta t  \quad ,
\end{equation}
die Molwärme bei der Temperatur $T$ ist
\begin{equation}
C_p (T) = \frac{E}{\Delta T} \frac{M}{m} \quad .
\label{eq:C_p}
\end{equation}
Da die Messung bei konstantem Druck $p$ geschieht, muss mit der Korrekturformel
\eqref{eq:theorie:korrekturformel} der Wert für $C_V$ bestimmt werden.

