%http://www.lenntech.de/pse/elemente/cu.htm als Molmasse


% ==================================================
%	Theorie
% ==================================================

\section{Theorie}
Wird einem Mol eines Stoffes die Energiemenge $Q$ zugeführt, so wird dadurch eine
Temperaturänderung $\Delta T$ hervorgerufen. Mithilfe der \textit{Molwärme} $C$ lässt sich dies als 
\begin{equation}
Q = C \Delta T
\end{equation}
schreiben. Die Molwärme ist im Allgemeinen eine stoffspezifische und 
temperaturabhängige Größe.

Ist die \textit{innere Energie} $U$ bekannt, so werden die 
\textit{Molwärme bei konstantem Druck} $C_p$ und die 
\textit{Molwärme bei konstantem Volumen} $C_V$ als
\begin{equation}
C_p = \left. \frac{\partial}{\partial T} \right|_p U
\quad \text{bzw.} \quad 
C_V = \left. \frac{\partial}{\partial T} \right|_V U
\end{equation}
definiert. Sie hängen über die Korrekturformel 
\begin{equation}
C_p - C_V = 9 \alpha^2 \kappa V_0 T \label{eq:theorie:korrekturformel}
\end{equation}
zusammen. Dabei ist $\alpha$ der \textit{lineare Ausdehnungskoeffizient}, 
$\kappa$ das \textit{Kompressionsmodul}, $V_0$ das Molvolumen und $T$ die 
Temperatur.

\paragraph{Klassisches Modell}
	Der klassische harmonische Oszillator besitzt eine \textit{mittlere 
	kinetische 
	Energie 
	pro Freiheitsgrad} $\langle u \rangle_\text{kin}$ von
	\begin{equation}
	\langle u \rangle_\text{kin} = \frac{1}{2} \text{k}_\text{B} T \quad ,
	\end{equation}
	wobei $T$ die Temperatur und $\text{k}_\text{B}$ die Boltzmann-Konstante 
	darstellen. Die innere Energie eines Mols wird dann für Atome mit 
	drei Freiheitsgraden und mit der 
	allgemeinen Gaskonstante $\text{R}$ zu 
	\begin{equation}
	U = 3 \text{R} T
	\end{equation}
	berechnet. Es folgt 
	\begin{equation}
	C_V = 3 \text{R} \quad . \label{eq:theorie:C_klassisch}
	\end{equation}
	In der klassischen Theorie ergibt sich demnach ein konstanter Wert, 
	der nicht von der Temperatur oder dem Material abhängt.

\paragraph{Einstein-Modell}
	Im Einstein-Modell wird die Energie-Quantelung der harmonischen 
	Oszillatoren berücksichtigt unter der Annahme, dass jedes Atom eine 
	harmonische Oszillation mit der Kreisfrequenz $\omega$ vollführt. 
	Der quantenmechanische harmonische 
	Oszillator mit Kreisfrequenz $\omega$ besitzt eine 
	\textit{mittlere Energie pro Freiheitsgrad} 
	$\langle u \rangle$ von 
	\begin{equation}
	\langle u \rangle = n \hbar \omega \quad .
	\end{equation}
	Dabei ist $\hbar$ das reduzierte Plancksche Wirkungsquantum und 
	$n$ kann die Werte $0,1,2,\ldots$ annehmen. Gemäß der 
	Boltzmann-Verteilung ist die Wahrscheinlichkeit $W$, dafür, dass ein 
	Oszillator die eine bestimmte Energie $n\hbar \omega$ besitzt, 
	\begin{equation}
	W(n) = \exp\left(-\frac{n \hbar \omega}{\text{k}_\text{B}T}\right)
	\quad .
	\end{equation}
	Unter Berücksichtigung der Boltzmann-Verteilung ergibt sich die innere 
	Energie 
	\begin{equation}
	U=3 \frac{\text{R}}{\text{k}_\text{B}} \frac{\hbar \omega}{
	\exp\left(\frac{\hbar \omega}{\text{k}_\text{B} T}\right)-1} 
	\end{equation}
	und somit 
	\begin{equation}
	C_V = 3 \text{R} \frac{\hbar^2 \omega^2}{\text{k}_\text{B}^2} 
	\frac{1}{T^2} \frac{\exp\left(\frac{\hbar \omega}{\text{k}_\text{B} 
	T}\right)}{\left(\exp\left(\frac{\hbar \omega}{\text{k}_\text{B} 
	T}\right)-1\right)^2} \quad .
	\label{eq:theorie:C_Einstein}
	\end{equation}
	
\paragraph{Debye-Modell}
	Im Debye-Modell werden nun alle Eigenschwingungen des Körpers 
	berücksichtigt. Dazu wir die \textit{spektrale Verteilung} $Z(\omega)$ 
	eingeführt. Mit dieser Modifikation folgt für die innere Energie 
	\begin{equation}
	U= \int\limits_0^{\omega_\text{max}} Z(\omega)
	\frac{\hbar \omega}{
	\exp\left(\frac{\hbar \omega}{\text{k}_\text{B} T}\right)-1} 
	\,\,\, \text{d}\omega \quad .
	\end{equation}
	Nun werden die Annahmen gemacht, dass die \textit{Phasengeschwindigkeit 
	einer Transver\\salwelle} $v_\text{tr}$ und 
	die \textit{Phasengeschwindigkeit 
	einer Longitudinalwelle} $v_\text{l}$ Frequenz- und 
	Richtungsunabhängig sind. Dann ist im Würfel der Kantenlänge $L$
	\begin{equation}
	Z(\omega) \,\,\, \text{d}\omega = 
	\frac{L^3}{2 \uppi^2} \omega^2 \left( 
	\frac{1}{v_\text{l}^3} + \frac{2}{v_\text{tr}^3} \right)
	\,\,\, \text{d} \omega \quad .
	\end{equation}
	Ein Kristall aus $N$ Atomen besitzt genau $3N$ Eigenschwingungen, sodass 
	die Bedingung 
	\begin{equation}
	\int\limits_0^{\omega_\text{D}} Z(\omega) 
	\,\,\, \text{d} \omega
	= 3 N
	\end{equation}
	erfüllt sein muss. Die obere Grenze $\omega_\text{D}$ ist die 
	\textit{Debye-Frequenz}. Für die genannten Annahmen gilt 
	\begin{equation}
	\omega_\text{D}^3 = \frac{18 \uppi^2 N}{L^3} \left(
	\frac{1}{v_\text{l}^3} + \frac{2}{v_\text{tr}^3}
	\right)^{-1} \quad . \label{eq:theorie:debye_frequenz}
	\end{equation}
	Die \textit{Debye-Temperatur} $\theta_\text{D}$ wird über 
	\begin{equation}
	\text{k}_\text{B} \theta_\text{D} = \hbar \omega_\text{D} 
	\label{eq:theorie:debye_temperatur}
	\end{equation}
	definiert. Für die Molwärme folgt schließlich 
	\begin{equation}
	C_V = 9 \text{R} \left(\frac{T}{\theta_\text{D}}\right)^3
	 \int\limits_0^{\frac{\theta_\text{D}}{T}} 
	 \frac{x^4\exp(x)}{\left(\exp(x)-1\right)^2} \,\,\, \text{d}x
	 \quad .
	 \label{eq:theorie:C_Debye}
	\end{equation}
	Hier ist insbesondere anzumerken, dass die Molwärme als Funktion 
	von $(\theta_\text{D}/T)$ universell ist, die einzige materialspezifische 
	Größe ist die Debye-Temperatur $\theta_\text{D}$.
	
Ein Vergleich der drei Ergebnisse \eqref{eq:theorie:C_klassisch}, 
\eqref{eq:theorie:C_Einstein} und \eqref{eq:theorie:C_Debye} zeigt, dass 
das Hochtemperaturverhalten durch den Wert $C_V = 3\text{R}$ bestimmt ist. 
Dieses Verhalten ist auch als das Dulong-Petit'sche Gesetz bekannt. 

Das Tieftemperaturverhalten kann im Einstein-Modell  \eqref{eq:theorie:C_Einstein} 
mit 
\begin{equation}
\exp(\nicefrac{\hbar \omega}{\text{k}_\text{B}T})-1 \approx \exp(\nicefrac{\hbar \omega}{\text{k}_\text{B}T}) 
\end{equation}
für kleine $T$ zu 
\begin{equation}
C_V \simeq 3 \text{R} \frac{\hbar^2 \omega^2}{\text{k}_\text{B}^2} 
\frac{1}{T^2} \exp\left(-\frac{\hbar \omega}{\text{k}_\text{B} T} \right) 
\propto T^{-2}
\exp\left(-\frac{\hbar \omega}{\text{k}_\text{B} T} \right) 
\end{equation}
abgeschätzt werden und zeigt damit ein exponentielles Verhalten. Im Debye-Modell \eqref{eq:theorie:C_Debye} folgt für 
niedrige Temperaturen hingegen die Abschätzung 
\begin{equation}
C_V \simeq 9 \text{R} \left( \frac{T}{\theta_\text{D}}\right)^3 
\int\limits_0^\infty \frac{x^4\exp(x)}{(\exp(x)-1)^2} \,\,\, \text{d}x 
\propto T^3 \quad .
\end{equation}
Dieses Verhalten ist als das $T^3$-Gesetz bekannt und beschreibt 
das Verhalten der Molwärme  bei niedrigen Temperaturen deutlich präziser 
als das Einstein-Modell. Für extrem 
niedrige Temperaturen muss das $T^3$ Gesetz noch einmal modifiziert werden, 
dies ist für diesen Versuch aber vernachlässigbar.

\paragraph{Theoretische Berechnung von $\omega_\text{D}$ und $\theta_\text{D}$}
Es sollen nun die Debye-Frequenz und die Debye-Temperatur berechnet werden. 
Dazu werden zunächst einige Materialgrößen zusammengetragen. Die Kupferprobe 
hat in diesem Versuch eine Masse von $m=\SI{342}{\gram}$. Die Molare Masse 
von Kupfer ist $M=\SI{63.55}{\gram \per \mol}$ \cite{Molmasse}. Die Dichte 
von Kupfer bei $T=\SI{20}{\celsius}$ ist $\rho = \SI{8.95}{\gram \per \cm^3}$ 
\cite{Molmasse}. 
Die Phasengeschwindigkeit beträgt $v_\text{l}=\SI{4.7\times 10^{3}}{\meter \per \second}$ bzw. $v_\text{tr}=\SI{2.26\times 10^{3}}{\meter\per\second}$.

Damit ist die Anzahl der Oszillatoren $N=m/M=3.241\times 10^{24}$. Das 
Volumen der Probe ist $V=L^3 \approx m/\rho =\SI{3.821\times 10^{-5}}{\m^3}$. 

Aus \eqref{eq:theorie:debye_frequenz} folgt damit
\begin{equation}
\omega_\text{D} = \SI{4.351 \times 10^{13} }{\second^{-1}} 
\end{equation}
und mit \eqref{eq:theorie:debye_temperatur}
\begin{equation}
\theta_\text{D} = \SI{332.3}{\kelvin} = \SI{59.19}{\celsius} \quad . 
\end{equation}