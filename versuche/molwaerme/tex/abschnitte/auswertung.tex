
% ==================================================
%	Auswertung
% ==================================================

\section{Auswertung}

%
% ========================================
%	Fehlerrechung
% ========================================
\subsection{Fehlerrechnung}
%
Grundsätzlich ist jedes Messergebnis mit einem Fehler behaftet. Das gemessene
Ergebnis weicht vom idealen, wahren Ergebnis ab. In diesem Versuch ist dabei
eine ausführliche Fehlerrechnung notwendig.  Die dafür benötigten Formel werden
nun kurz dargestellt.
\begin{description}
  \item[arithmetisches Mittel:] Bei Messwerten mit statistischen Fehlern
    entspricht das arithmetische Mittel dem empirischen Erwartungswert.
    Es ist definiert als
  \begin{equation}
    \langle x \rangle = \frac{1}{n} \sum_{i = 1}^n x_i
    \label{eq:Mittelwert}\\
    % \big(x_i : \text{$i$-te Messwert}\,,
    % \quad n : \text{Anzahl der Messwerte} \big)~. \notag
  \end{equation}
  mit dem $i$-ten Messwert $x_i$ und der Anzahl der Messwerte $n$.
  %
  \item[mittlerer Fehler des Mittelwertes:] Je größer die Zahl der Messungen
    ist, desto vertrauenswürdiger ist der arithmetische Mittelwert.
    Dies wird mit dem mittleren Fehler des Mittelwertes $\sigma_m$ ausgedrückt.
    Er ist definiert als
  \begin{equation}
    \sigma_m = \frac{\sigma}{\sqrt{n}} =
    \sqrt{\sum_{i = 1}^n \frac{(x_i - \langle x \rangle)^2}{n(n-1)}}~.
    \label{eq:absloluter_Fehler}
  \end{equation}
  Zudem entspricht $\sigma_m$ dem absoluten Fehler $\Delta x$ einer
  fehlerbehafteten Größe $x$.
  %
  \item[Gauß'sche Fehlerfortpflanzung:] Besteht eine Messgröße $y$ aus mehreren
    fehlerbehafteten Größen $x_i$, so muss eine Fehlerfortpflanzung
    durchgeführt werden. Sie hat die Form
  \begin{equation}
    \Delta y =
    \sqrt{\sum_{i = 1}^n \left(\pdv{y}{x_i} \Delta x_i  \right)^2}~,
    \label{Gauss}
  \end{equation}
  % \begin{equation*}
  % \left(\,
  %     \begin{aligned}
  %     y &:
  %     \text{zusammengesetzte Größe mit mehreren fehlerbehafteten Größen}~, \\
  %     \Delta y &: \text{absoluter Fehler von $y$}~, \\
  %     \Delta x_i &: \text{absolute Fehler der einzelnen Größen}
  %     \end{aligned}
  % \,\right)~.
  % \end{equation*}
  wobei $y$ die zusammengesetzte Größe mit mehreren fehlerbehafteten Größen
  ist, $\Delta y$ der absolute Fehler von $y$ und $\Delta x_i$ der absolute
  Fehler der einzelnen Größen.
  %
\end{description}
%
Die Angabe einer Messgröße besitzt damit die Form
\begin{equation}
  y = \langle y \rangle \pm \Delta y~. \notag
\end{equation}
%
%
\clearpage


\subsection{Bestimmung der Molwärme}
\label{sub:bestimmung_der_molwärme}

Die in diesem Versuch aufgenommenen Messwerte sind in
Tabelle~\ref{tab:messwerte} dargestellt. Um $C_p$ zu bestimmen wird zunächst
die zugeführte Energie berechnet. Dazu wird die erste Spalte in
Tabelle~\ref{tab:messwerte} in Sekunden umgerechnet und die Zeitdifferenzen
berechnet.  Die Zeitdifferenzen und die zugehörigen Spannungen und Ströme,
sowie die daraus berechneten Energien sind in Tabelle~\ref{tab:energie}
dargestellt. Als Fehler werden dabei die letzte Stelle Stelle der Anzeige der
Digitalmessgeräte genommen. Dies sind für die Spannung \SI{0.01}{\volt} und für
den Strom \SI{0.1}{\mA}.

Zur Bestimmung von $C_p$ werden nun aus den aufgenommenen Widerständen
die Temperaturen in Grad Celsius mit Hilfe
von~\eqref{eq:T_berechnung} ermittelt. Für den Fehler des Widerstandes wird
ebenfalls die letzte Stelle des Anzeigegerätes verwandt mit \SI{0.1}{\ohm}.
Desweiteren werden die Temperaturdifferenzen bestimmt.
Die Zwischenergebnisse sowie die mit Hilfe von~\eqref{eq:C_p}
berechneten $C_p$ bei den jeweiligen
Temperaturen sind in Tabelle~\ref{tab:C_p} angegeben.

Die Molwärme $C_V$ wird nun mit Gleichung~\ref{eq:theorie:korrekturformel}
bestimmt.
Hierfür werden die Größen
\begin{align}
  \kappa &= \SI{1.378 \times 10^{-11}}{\newton\per\cubic\meter} \\
  V_0 &= \frac{M}{\varrho} = \SI{7.124 \times 10^{-6}}{\cubic\meter\per\mol}
\end{align}
mit
\begin{align}
  \varrho &= \SI{8.92}{\gram\per\cubic\meter} \\
  M &= \SI{63.55}{\gram\per\mol}
\end{align}
verwandt.
Die Werte für den Längenausdehnungskoeffizienten $\alpha$ werden aus der Tabelle
in Abbildung~\ref{fig:alpha} gemäß
\begin{equation}
  \alpha(T) = \frac{\alpha_i - \alpha_{i-1}}{T_i - T_{i-1}} \qty(T - T_{i-1})
  + \alpha_{i-1}
\end{equation}
linear interpoliert.
Die interpolierten $\alpha$ und das berechnete $C_V$ sind in
Tabelle~\ref{tab:C_V} dargestellt.
In Abbildung~\ref{fig:C_V} sind die Molwärmen $C_V$ in Abhängigkeit der
Temperaturen in Kelvin dargestellt.
Hier ist um \SI{210}{\kelvin} ein Einbruch des Verlaufes zu erkennen.
Aus diesem Grund ist in Abbildung~\ref{fig:temp} die Gehäuse- und
Probentemperatur graphisch dargestellt.
In Tabelle~\ref{tab:temp} sind die entsprechenden Werte angegeben.
Darin ist zu erkennen, dass der Einbruch im Verlauf von $C_V$ mit dem
Auseinanderlaufen der Temperaturen zusammenhängt.
Somit hat es in diesem Bereich ein Wärmeabfuhr von der Probe gegeben, sodass
die Energie zu gering berechnet wird und damit auch $C_V$ zu gering ausfällt.

\begin{table}[htpb]
  \centering
  \begin{tabular}{ccccc}
    \midrule
    \midrule
    Zeit & $R~/~\si{\ohm}$ & $I~/~\si{\milli\ampere}$ &
    $U~/~\si{\volt}$ & $R_\text{Geh}~/~\si{\ohm}$\\
    \midrule
    00:00:00          & \phantom{0}22.7   & \phantom{00}0.0   & \phantom{0}0.00   & \phantom{0}22.6  \\
00:08:14          & \phantom{0}25.2   & \phantom{0}99.1   & 10.38             & \phantom{0}26.3  \\
00:14:15          & \phantom{0}29.1   & 145.9             & 15.29             & \phantom{0}28.3  \\
00:19:12          & \phantom{0}32.9   & 172.9             & 18.16             & \phantom{0}31.6  \\
00:24:15          & \phantom{0}36.6   & 173.4             & 18.23             & \phantom{0}36.0  \\
00:28:58          & \phantom{0}40.6   & 173.6             & 18.30             & \phantom{0}40.9  \\
00:32:11          & \phantom{0}42.9   & 173.9             & 18.31             & \phantom{0}44.1  \\
00:38:37          & \phantom{0}47.8   & 174.2             & 18.37             & \phantom{0}48.9  \\
00:43:22          & \phantom{0}51.1   & 174.2             & 18.40             & \phantom{0}51.4  \\
00:49:13          & \phantom{0}55.2   & 174.5             & 18.44             & \phantom{0}54.2  \\
00:54:45          & \phantom{0}58.9   & 174.6             & 18.45             & \phantom{0}58.2  \\
01:00:30          & \phantom{0}62.7   & 174.9             & 18.47             & \phantom{0}69.5  \\
01:05:09          & \phantom{0}66.2   & 174.8             & 18.48             & \phantom{0}81.3  \\
01:10:14          & \phantom{0}70.2   & 175.0             & 18.50             & \phantom{0}82.2  \\
01:15:15          & \phantom{0}73.8   & 175.0             & 18.50             & \phantom{0}80.7  \\
01:20:09          & \phantom{0}77.3   & 175.0             & 18.50             & \phantom{0}79.3  \\
01:24:59          & \phantom{0}80.2   & 175.0             & 18.50             & \phantom{0}80.6  \\
01:31:35          & \phantom{0}84.0   & 175.2             & 18.50             & \phantom{0}83.8  \\
01:38:36          & \phantom{0}88.0   & 175.3             & 18.50             & \phantom{0}88.0  \\
01:44:38          & \phantom{0}91.5   & 175.3             & 19.50             & \phantom{0}91.5  \\
01:50:40          & \phantom{0}94.8   & 175.3             & 18.51             & \phantom{0}95.0  \\
01:56:25          & \phantom{0}98.2   & 175.5             & 18.51             & \phantom{0}98.9  \\
02:02:33          & 101.9             & 175.4             & 18.51             & 101.5            \\
02:08:50          & 105.4             & 175.3             & 18.50             & 106.1            \\
02:15:10          & 108.9             & 175.6             & 18.50             & 106.2            \\
02:18:50          & 111.0             & 175.5             & 18.50             & 111.5            \\
    \midrule
    \midrule
  \end{tabular}
  \caption{Im Versuch aufgenommene Messwerte für den Widerstand $R$ des
    Pt-100 der Probe und $R_\text{Geh}$ des Gehäuses sowie den Strom $I$
    und der Spannung $U$.}
  \label{tab:messwerte}
\end{table}

\begin{table}[htpb]
  \centering
  \begin{tabular}{rrr||rr}
    \midrule
    \midrule
    Zeit &
    $U~/~\si{\volt}$ &
    $I~/~\si{\mA}$ &
    $\Delta t~/~\si{\second}$ &
    $E~/~\si{\joule}$ \\
    \midrule
    00:00:00          & \SI[parse-numbers = false]{0.000 \pm 0.010}{} & \SI[parse-numbers = false]{0.00 \pm 0.10}{} & \phantom{00}0\phantom{.} & \SI[parse-numbers = false]{0.0 \pm 0}{}\\
00:08:14          & \SI[parse-numbers = false]{10.380 \pm 0.010}{} & \SI[parse-numbers = false]{99.10 \pm 0.10}{} & 494\phantom{.}    & \SI[parse-numbers = false]{508.2 \pm 0.7}{}\\
00:14:15          & \SI[parse-numbers = false]{15.290 \pm 0.010}{} & \SI[parse-numbers = false]{145.90 \pm 0.10}{} & 361\phantom{.}    & \SI[parse-numbers = false]{805.3 \pm 0.8}{}\\
00:19:12          & \SI[parse-numbers = false]{18.160 \pm 0.010}{} & \SI[parse-numbers = false]{172.90 \pm 0.10}{} & 297\phantom{.}    & \SI[parse-numbers = false]{932.5 \pm 0.7}{}\\
00:24:15          & \SI[parse-numbers = false]{18.230 \pm 0.010}{} & \SI[parse-numbers = false]{173.40 \pm 0.10}{} & 303\phantom{.}    & \SI[parse-numbers = false]{957.8 \pm 0.8}{}\\
00:28:58          & \SI[parse-numbers = false]{18.300 \pm 0.010}{} & \SI[parse-numbers = false]{173.60 \pm 0.10}{} & 283\phantom{.}    & \SI[parse-numbers = false]{899.1 \pm 0.7}{}\\
00:32:11          & \SI[parse-numbers = false]{18.310 \pm 0.010}{} & \SI[parse-numbers = false]{173.90 \pm 0.10}{} & 193\phantom{.}    & \SI[parse-numbers = false]{614.5 \pm 0.5}{}\\
00:38:37          & \SI[parse-numbers = false]{18.370 \pm 0.010}{} & \SI[parse-numbers = false]{174.20 \pm 0.10}{} & 386\phantom{.}    & \SI[parse-numbers = false]{1235.2 \pm 1.0}{}\\
00:43:22          & \SI[parse-numbers = false]{18.400 \pm 0.010}{} & \SI[parse-numbers = false]{174.20 \pm 0.10}{} & 285\phantom{.}    & \SI[parse-numbers = false]{913.5 \pm 0.7}{}\\
00:49:13          & \SI[parse-numbers = false]{18.440 \pm 0.010}{} & \SI[parse-numbers = false]{174.50 \pm 0.10}{} & 351\phantom{.}    & \SI[parse-numbers = false]{1129.4 \pm 0.9}{}\\
00:54:45          & \SI[parse-numbers = false]{18.450 \pm 0.010}{} & \SI[parse-numbers = false]{174.60 \pm 0.10}{} & 332\phantom{.}    & \SI[parse-numbers = false]{1069.5 \pm 0.8}{}\\
01:00:30          & \SI[parse-numbers = false]{18.470 \pm 0.010}{} & \SI[parse-numbers = false]{174.90 \pm 0.10}{} & 345\phantom{.}    & \SI[parse-numbers = false]{1114.5 \pm 0.9}{}\\
01:05:09          & \SI[parse-numbers = false]{18.480 \pm 0.010}{} & \SI[parse-numbers = false]{174.80 \pm 0.10}{} & 279\phantom{.}    & \SI[parse-numbers = false]{901.3 \pm 0.7}{}\\
01:10:14          & \SI[parse-numbers = false]{18.500 \pm 0.010}{} & \SI[parse-numbers = false]{175.00 \pm 0.10}{} & 305\phantom{.}    & \SI[parse-numbers = false]{987.4 \pm 0.8}{}\\
01:15:15          & \SI[parse-numbers = false]{18.500 \pm 0.010}{} & \SI[parse-numbers = false]{175.00 \pm 0.10}{} & 301\phantom{.}    & \SI[parse-numbers = false]{974.5 \pm 0.8}{}\\
01:20:09          & \SI[parse-numbers = false]{18.500 \pm 0.010}{} & \SI[parse-numbers = false]{175.00 \pm 0.10}{} & 294\phantom{.}    & \SI[parse-numbers = false]{951.8 \pm 0.7}{}\\
01:24:59          & \SI[parse-numbers = false]{18.500 \pm 0.010}{} & \SI[parse-numbers = false]{175.00 \pm 0.10}{} & 290\phantom{.}    & \SI[parse-numbers = false]{938.9 \pm 0.7}{}\\
01:31:35          & \SI[parse-numbers = false]{18.500 \pm 0.010}{} & \SI[parse-numbers = false]{175.20 \pm 0.10}{} & 396\phantom{.}    & \SI[parse-numbers = false]{1283.5 \pm 1.0}{}\\
01:38:36          & \SI[parse-numbers = false]{18.500 \pm 0.010}{} & \SI[parse-numbers = false]{175.30 \pm 0.10}{} & 421\phantom{.}    & \SI[parse-numbers = false]{1365.3 \pm 1.1}{}\\
01:44:38          & \SI[parse-numbers = false]{19.500 \pm 0.010}{} & \SI[parse-numbers = false]{175.30 \pm 0.10}{} & 362\phantom{.}    & \SI[parse-numbers = false]{1237.4 \pm 0.9}{}\\
01:50:40          & \SI[parse-numbers = false]{18.510 \pm 0.010}{} & \SI[parse-numbers = false]{175.30 \pm 0.10}{} & 362\phantom{.}    & \SI[parse-numbers = false]{1174.6 \pm 0.9}{}\\
01:56:25          & \SI[parse-numbers = false]{18.510 \pm 0.010}{} & \SI[parse-numbers = false]{175.50 \pm 0.10}{} & 345\phantom{.}    & \SI[parse-numbers = false]{1120.7 \pm 0.9}{}\\
02:02:33          & \SI[parse-numbers = false]{18.510 \pm 0.010}{} & \SI[parse-numbers = false]{175.40 \pm 0.10}{} & 368\phantom{.}    & \SI[parse-numbers = false]{1194.8 \pm 0.9}{}\\
02:08:50          & \SI[parse-numbers = false]{18.500 \pm 0.010}{} & \SI[parse-numbers = false]{175.30 \pm 0.10}{} & 377\phantom{.}    & \SI[parse-numbers = false]{1222.6 \pm 1.0}{}\\
02:15:10          & \SI[parse-numbers = false]{18.500 \pm 0.010}{} & \SI[parse-numbers = false]{175.60 \pm 0.10}{} & 380\phantom{.}    & \SI[parse-numbers = false]{1234.5 \pm 1.0}{}\\
02:18:50          & \SI[parse-numbers = false]{18.500 \pm 0.010}{} & \SI[parse-numbers = false]{175.50 \pm 0.10}{} & 220\phantom{.}    & \SI[parse-numbers = false]{714.3 \pm 0.6}{}\\
    \midrule
    \midrule
  \end{tabular}
  \caption{Werte für die Berechnung der zugeführten Energie.}
\label{tab:energie}
\end{table}

\begin{table}[htpb]
  \centering
  \begin{tabular}{rr||rrr}
    \midrule
    \midrule
    Zeit & $R~/~\si{\ohm}$ & $T~/~\si{\celsius}$ &%& $T~/~\si{\kelvin}$ &
    $\Delta T~/~\si{\kelvin}$ &
    $C_p~/~\si{\joule\per\mol\per\kelvin}$ \\
    % $\alpha~/~\SI{10^-6}{\per\kelvin}$ &
    % $C_V~/~\si{\joule\per\mol\per\kelvin}$ \\
    \midrule
    00:00:00          & \SI[parse-numbers = false]{22.70 \pm 0.10}{} & \SI[parse-numbers = false]{-190.21 \pm 0.24}{} & \SI[parse-numbers = false]{0.0 \pm 0}{} & \SI[parse-numbers = false]{0.0 \pm 0}{}\\
00:08:14          & \SI[parse-numbers = false]{25.20 \pm 0.10}{} & \SI[parse-numbers = false]{-184.31 \pm 0.24}{} & \SI[parse-numbers = false]{5.90 \pm 0.33}{} & \SI[parse-numbers = false]{16.0 \pm 0.9}{}\\
00:14:15          & \SI[parse-numbers = false]{29.10 \pm 0.10}{} & \SI[parse-numbers = false]{-175.07 \pm 0.24}{} & \SI[parse-numbers = false]{9.24 \pm 0.33}{} & \SI[parse-numbers = false]{16.2 \pm 0.6}{}\\
00:19:12          & \SI[parse-numbers = false]{32.90 \pm 0.10}{} & \SI[parse-numbers = false]{-166.03 \pm 0.24}{} & \SI[parse-numbers = false]{9.04 \pm 0.34}{} & \SI[parse-numbers = false]{19.2 \pm 0.7}{}\\
00:24:15          & \SI[parse-numbers = false]{36.60 \pm 0.10}{} & \SI[parse-numbers = false]{-157.19 \pm 0.24}{} & \SI[parse-numbers = false]{8.84 \pm 0.34}{} & \SI[parse-numbers = false]{20.1 \pm 0.8}{}\\
00:28:58          & \SI[parse-numbers = false]{40.60 \pm 0.10}{} & \SI[parse-numbers = false]{-147.59 \pm 0.24}{} & \SI[parse-numbers = false]{9.60 \pm 0.34}{} & \SI[parse-numbers = false]{17.4 \pm 0.6}{}\\
00:32:11          & \SI[parse-numbers = false]{42.90 \pm 0.10}{} & \SI[parse-numbers = false]{-142.06 \pm 0.24}{} & \SI[parse-numbers = false]{5.54 \pm 0.34}{} & \SI[parse-numbers = false]{20.6 \pm 1.3}{}\\
00:38:37          & \SI[parse-numbers = false]{47.80 \pm 0.10}{} & \SI[parse-numbers = false]{-130.21 \pm 0.24}{} & \SI[parse-numbers = false]{11.85 \pm 0.34}{} & \SI[parse-numbers = false]{19.4 \pm 0.6}{}\\
00:43:22          & \SI[parse-numbers = false]{51.10 \pm 0.10}{} & \SI[parse-numbers = false]{-122.20 \pm 0.24}{} & \SI[parse-numbers = false]{8.01 \pm 0.34}{} & \SI[parse-numbers = false]{21.2 \pm 0.9}{}\\
00:49:13          & \SI[parse-numbers = false]{55.20 \pm 0.10}{} & \SI[parse-numbers = false]{-112.20 \pm 0.24}{} & \SI[parse-numbers = false]{10.00 \pm 0.34}{} & \SI[parse-numbers = false]{21.0 \pm 0.7}{}\\
00:54:45          & \SI[parse-numbers = false]{58.90 \pm 0.10}{} & \SI[parse-numbers = false]{-103.14 \pm 0.25}{} & \SI[parse-numbers = false]{9.06 \pm 0.35}{} & \SI[parse-numbers = false]{21.9 \pm 0.8}{}\\
01:00:30          & \SI[parse-numbers = false]{62.70 \pm 0.10}{} & \SI[parse-numbers = false]{-93.79 \pm 0.25}{} & \SI[parse-numbers = false]{9.34 \pm 0.35}{} & \SI[parse-numbers = false]{22.2 \pm 0.8}{}\\
01:05:09          & \SI[parse-numbers = false]{66.20 \pm 0.10}{} & \SI[parse-numbers = false]{-85.15 \pm 0.25}{} & \SI[parse-numbers = false]{8.64 \pm 0.35}{} & \SI[parse-numbers = false]{19.4 \pm 0.8}{}\\
01:10:14          & \SI[parse-numbers = false]{70.20 \pm 0.10}{} & \SI[parse-numbers = false]{-75.24 \pm 0.25}{} & \SI[parse-numbers = false]{9.92 \pm 0.35}{} & \SI[parse-numbers = false]{18.5 \pm 0.7}{}\\
01:15:15          & \SI[parse-numbers = false]{73.80 \pm 0.10}{} & \SI[parse-numbers = false]{-66.28 \pm 0.25}{} & \SI[parse-numbers = false]{8.96 \pm 0.35}{} & \SI[parse-numbers = false]{20.2 \pm 0.8}{}\\
01:20:09          & \SI[parse-numbers = false]{77.30 \pm 0.10}{} & \SI[parse-numbers = false]{-57.53 \pm 0.25}{} & \SI[parse-numbers = false]{8.74 \pm 0.35}{} & \SI[parse-numbers = false]{20.2 \pm 0.8}{}\\
01:24:59          & \SI[parse-numbers = false]{80.20 \pm 0.10}{} & \SI[parse-numbers = false]{-50.26 \pm 0.25}{} & \SI[parse-numbers = false]{7.27 \pm 0.35}{} & \SI[parse-numbers = false]{24.0 \pm 1.2}{}\\
01:31:35          & \SI[parse-numbers = false]{84.00 \pm 0.10}{} & \SI[parse-numbers = false]{-40.70 \pm 0.25}{} & \SI[parse-numbers = false]{9.6 \pm 0.4}{} & \SI[parse-numbers = false]{24.9 \pm 0.9}{}\\
01:38:36          & \SI[parse-numbers = false]{88.00 \pm 0.10}{} & \SI[parse-numbers = false]{-30.60 \pm 0.25}{} & \SI[parse-numbers = false]{10.1 \pm 0.4}{} & \SI[parse-numbers = false]{25.1 \pm 0.9}{}\\
01:44:38          & \SI[parse-numbers = false]{91.50 \pm 0.10}{} & \SI[parse-numbers = false]{-21.72 \pm 0.25}{} & \SI[parse-numbers = false]{8.9 \pm 0.4}{} & \SI[parse-numbers = false]{25.9 \pm 1.0}{}\\
01:50:40          & \SI[parse-numbers = false]{94.80 \pm 0.10}{} & \SI[parse-numbers = false]{-13.32 \pm 0.26}{} & \SI[parse-numbers = false]{8.4 \pm 0.4}{} & \SI[parse-numbers = false]{26.0 \pm 1.1}{}\\
01:56:25          & \SI[parse-numbers = false]{98.20 \pm 0.10}{} & \SI[parse-numbers = false]{-4.63 \pm 0.26}{} & \SI[parse-numbers = false]{8.7 \pm 0.4}{} & \SI[parse-numbers = false]{24.0 \pm 1.0}{}\\
02:02:33          & \SI[parse-numbers = false]{101.90 \pm 0.10}{} & \SI[parse-numbers = false]{4.86 \pm 0.26}{} & \SI[parse-numbers = false]{9.5 \pm 0.4}{} & \SI[parse-numbers = false]{23.4 \pm 0.9}{}\\
02:08:50          & \SI[parse-numbers = false]{105.40 \pm 0.10}{} & \SI[parse-numbers = false]{13.86 \pm 0.26}{} & \SI[parse-numbers = false]{9.0 \pm 0.4}{} & \SI[parse-numbers = false]{25.2 \pm 1.0}{}\\
02:15:10          & \SI[parse-numbers = false]{108.90 \pm 0.10}{} & \SI[parse-numbers = false]{22.91 \pm 0.26}{} & \SI[parse-numbers = false]{9.0 \pm 0.4}{} & \SI[parse-numbers = false]{25.4 \pm 1.0}{}\\
02:18:50          & \SI[parse-numbers = false]{111.00 \pm 0.10}{} & \SI[parse-numbers = false]{28.35 \pm 0.26}{} & \SI[parse-numbers = false]{5.4 \pm 0.4}{} & \SI[parse-numbers = false]{24.4 \pm 1.6}{}\\
    \midrule
    \midrule
  \end{tabular}
  \caption{Werte zur Berechnung von $C_p$.}
\label{tab:C_p}
\end{table}

\begin{table}[htpb]
  \centering
  \begin{tabular}{crr||cr}
    \midrule
    \midrule
    Zeit & $T~/~\si{\kelvin}$ &
    $C_p~/~\si{\joule\per\mol\per\kelvin}$ &
    $\alpha~/~\SI{10^{-5}}{\per\kelvin}$ &
    $C_V~/~\si{\joule\per\mol\per\kelvin}$ \\
    \midrule
    00:00:00          & \SI[parse-numbers = false]{82.94 \pm 0.24}{} & \SI[parse-numbers = false]{0.0 \pm 0}{} & 0.887             & \SI[parse-numbers = false]{-0.05762 \pm 0.00016}{}\\
00:08:14          & \SI[parse-numbers = false]{88.84 \pm 0.24}{} & \SI[parse-numbers = false]{16.0 \pm 0.9}{} & 0.961             & \SI[parse-numbers = false]{15.9 \pm 0.9}{}\\
00:14:15          & \SI[parse-numbers = false]{98.08 \pm 0.24}{} & \SI[parse-numbers = false]{16.2 \pm 0.6}{} & 1.052             & \SI[parse-numbers = false]{16.1 \pm 0.6}{}\\
00:19:12          & \SI[parse-numbers = false]{107.12 \pm 0.24}{} & \SI[parse-numbers = false]{19.2 \pm 0.7}{} & 1.127             & \SI[parse-numbers = false]{19.0 \pm 0.7}{}\\
00:24:15          & \SI[parse-numbers = false]{115.96 \pm 0.24}{} & \SI[parse-numbers = false]{20.1 \pm 0.8}{} & 1.186             & \SI[parse-numbers = false]{20.0 \pm 0.8}{}\\
00:28:58          & \SI[parse-numbers = false]{125.56 \pm 0.24}{} & \SI[parse-numbers = false]{17.4 \pm 0.6}{} & 1.241             & \SI[parse-numbers = false]{17.2 \pm 0.6}{}\\
00:32:11          & \SI[parse-numbers = false]{131.09 \pm 0.24}{} & \SI[parse-numbers = false]{20.6 \pm 1.3}{} & 1.270             & \SI[parse-numbers = false]{20.4 \pm 1.3}{}\\
00:38:37          & \SI[parse-numbers = false]{142.94 \pm 0.24}{} & \SI[parse-numbers = false]{19.4 \pm 0.6}{} & 1.328             & \SI[parse-numbers = false]{19.2 \pm 0.6}{}\\
00:43:22          & \SI[parse-numbers = false]{150.95 \pm 0.24}{} & \SI[parse-numbers = false]{21.2 \pm 0.9}{} & 1.363             & \SI[parse-numbers = false]{20.9 \pm 0.9}{}\\
00:49:13          & \SI[parse-numbers = false]{160.95 \pm 0.24}{} & \SI[parse-numbers = false]{21.0 \pm 0.7}{} & 1.393             & \SI[parse-numbers = false]{20.7 \pm 0.7}{}\\
00:54:45          & \SI[parse-numbers = false]{170.01 \pm 0.25}{} & \SI[parse-numbers = false]{21.9 \pm 0.8}{} & 1.425             & \SI[parse-numbers = false]{21.6 \pm 0.8}{}\\
01:00:30          & \SI[parse-numbers = false]{179.36 \pm 0.25}{} & \SI[parse-numbers = false]{22.2 \pm 0.8}{} & 1.448             & \SI[parse-numbers = false]{21.8 \pm 0.8}{}\\
01:05:09          & \SI[parse-numbers = false]{188.00 \pm 0.25}{} & \SI[parse-numbers = false]{19.4 \pm 0.8}{} & 1.470             & \SI[parse-numbers = false]{19.0 \pm 0.8}{}\\
01:10:14          & \SI[parse-numbers = false]{197.91 \pm 0.25}{} & \SI[parse-numbers = false]{18.5 \pm 0.7}{} & 1.491             & \SI[parse-numbers = false]{18.1 \pm 0.7}{}\\
01:15:15          & \SI[parse-numbers = false]{206.87 \pm 0.25}{} & \SI[parse-numbers = false]{20.2 \pm 0.8}{} & 1.512             & \SI[parse-numbers = false]{19.8 \pm 0.8}{}\\
01:20:09          & \SI[parse-numbers = false]{215.62 \pm 0.25}{} & \SI[parse-numbers = false]{20.2 \pm 0.8}{} & 1.531             & \SI[parse-numbers = false]{19.8 \pm 0.8}{}\\
01:24:59          & \SI[parse-numbers = false]{222.89 \pm 0.25}{} & \SI[parse-numbers = false]{24.0 \pm 1.2}{} & 1.546             & \SI[parse-numbers = false]{23.5 \pm 1.2}{}\\
01:31:35          & \SI[parse-numbers = false]{232.45 \pm 0.25}{} & \SI[parse-numbers = false]{24.9 \pm 0.9}{} & 1.564             & \SI[parse-numbers = false]{24.4 \pm 0.9}{}\\
01:38:36          & \SI[parse-numbers = false]{242.55 \pm 0.25}{} & \SI[parse-numbers = false]{25.1 \pm 0.9}{} & 1.579             & \SI[parse-numbers = false]{24.6 \pm 0.9}{}\\
01:44:38          & \SI[parse-numbers = false]{251.43 \pm 0.25}{} & \SI[parse-numbers = false]{25.9 \pm 1.0}{} & 1.593             & \SI[parse-numbers = false]{25.3 \pm 1.0}{}\\
01:50:40          & \SI[parse-numbers = false]{259.83 \pm 0.26}{} & \SI[parse-numbers = false]{26.0 \pm 1.1}{} & 1.610             & \SI[parse-numbers = false]{25.4 \pm 1.1}{}\\
01:56:25          & \SI[parse-numbers = false]{268.52 \pm 0.26}{} & \SI[parse-numbers = false]{24.0 \pm 1.0}{} & 1.623             & \SI[parse-numbers = false]{23.4 \pm 1.0}{}\\
02:02:33          & \SI[parse-numbers = false]{278.01 \pm 0.26}{} & \SI[parse-numbers = false]{23.4 \pm 0.9}{} & 1.633             & \SI[parse-numbers = false]{22.7 \pm 0.9}{}\\
02:08:50          & \SI[parse-numbers = false]{287.01 \pm 0.26}{} & \SI[parse-numbers = false]{25.2 \pm 1.0}{} & 1.646             & \SI[parse-numbers = false]{24.5 \pm 1.0}{}\\
02:15:10          & \SI[parse-numbers = false]{296.06 \pm 0.26}{} & \SI[parse-numbers = false]{25.4 \pm 1.0}{} & 1.659             & \SI[parse-numbers = false]{24.7 \pm 1.0}{}\\
02:18:50          & \SI[parse-numbers = false]{301.50 \pm 0.26}{} & \SI[parse-numbers = false]{24.4 \pm 1.6}{} & 1.665             & \SI[parse-numbers = false]{23.7 \pm 1.6}{}\\
    \midrule
    \midrule
  \end{tabular}
  \caption{Werte zur Berechnung von $C_V$.}
\label{tab:C_V}
\end{table}

\begin{figure}[h]
  \centering
  \includegraphics[scale=0.3]{bilder/alpha.png}
  \caption{Ausdehnungskoeffizienten $\alpha$~\cite{FP}.}
\label{fig:alpha}
\end{figure}

\begin{figure}[h]
  \centering
  \includegraphics[scale=1.0]{bilder/cv.pdf}
  \caption{Darstellung von $C_V$ in Abhängigkeit der Temperatur $T$.}
\label{fig:C_V}
\end{figure}

\begin{table}[htpb]
  \centering
  \begin{tabular}{c||ccc}
    \midrule
    \midrule
    Zeit & $T~/~\si{\kelvin}$ & $T_\text{Geh}~/~\si{\kelvin}$ &
    $C_V~/~\si{\joule\per\mol\per\kelvin}$ \\
    \midrule
    00:00:00          & \phantom{0}82.9   & \phantom{0}82.7   & --0.1            \\
00:08:14          & \phantom{0}88.8   & \phantom{0}91.4   & 15.9             \\
00:14:15          & \phantom{0}98.1   & \phantom{0}96.2   & 16.1             \\
00:19:12          & 107.1             & 104.0             & 19.0             \\
00:24:15          & 116.0             & 114.5             & 20.0             \\
00:28:58          & 125.6             & 126.3             & 17.2             \\
00:32:11          & 131.1             & 134.0             & 20.4             \\
00:38:37          & 142.9             & 145.6             & 19.2             \\
00:43:22          & 151.0             & 151.7             & 20.9             \\
00:49:13          & 161.0             & 158.5             & 20.7             \\
00:54:45          & 170.0             & 168.3             & 21.6             \\
01:00:30          & 179.4             & 196.2             & 21.8             \\
01:05:09          & 188.0             & 225.7             & 19.0             \\
01:10:14          & 197.9             & 227.9             & 18.1             \\
01:15:15          & 206.9             & 224.1             & 19.8             \\
01:20:09          & 215.6             & 220.6             & 19.8             \\
01:24:59          & 222.9             & 223.9             & 23.5             \\
01:31:35          & 232.4             & 231.9             & 24.4             \\
01:38:36          & 242.6             & 242.6             & 24.6             \\
01:44:38          & 251.4             & 251.4             & 25.3             \\
01:50:40          & 259.8             & 260.3             & 25.4             \\
01:56:25          & 268.5             & 270.3             & 23.4             \\
02:02:33          & 278.0             & 277.0             & 22.7             \\
02:08:50          & 287.0             & 288.8             & 24.5             \\
02:15:10          & 296.1             & 289.1             & 24.7             \\
02:18:50          & 301.5             & 302.8             & 23.7             \\
    \midrule
    \midrule
  \end{tabular}
  \caption{Werte der Temperatur der Probe $T$ und des Gehäuses $T_\text{Geh}$
  sowie der Molwärmen $C_V$ bie den jeweiligen Messzeiten.}
\label{tab:temp}
\end{table}

\begin{figure}[htpb]
  \centering
  \includegraphics[scale=1.0]{bilder/temp.pdf}
  \caption{Darstellung der Proben- und Gehäusetemperatur in Abhängigkeit der
  Zeit. Zum Vergleich is zudem ist die Molwärme $C_V$ aufgetragen.}
\label{fig:temp}
\end{figure}

\clearpage
\subsection{Berechnung der Debye-Temperatur}
\label{sub:berechnung_der_debye_temperatur}

Die Debye-Temperatur $\theta_D$ wird nun anhand der Wertepaare $(C_V, T)$ aus
Tabelle~\ref{tab:C_V} per Vergleich mit
\begin{equation}
  C_V = f\qty(\frac{\theta_D}{T})
\end{equation}
aus den Werten der Abbildung~\ref{fig:debye} bestimmt.
Hierbei werden nur Werte zwischen \SI{80}{\kelvin} und \SI{170}{\kelvin}
betrachtet.
Die abgelesenen Werte für $\theta_D/T$ sowie die daraus berechneten
Debye-Temperaturen sind in Tabelle~\ref{tab:debye} angegeben.
Anschließend wird die Debye-Temperatur aus dem Mittelwert der Debye-Temperaturen
mit der entsprechenden mittleren Fehler des Mittelwertes zu
\begin{equation}
  \theta_D = \SI[parse-numbers = false]{295 \pm 12}{\kelvin}
\end{equation}
bestimmt.

\begin{table}[htpb]
  \centering
  \begin{tabular}{ccc||c}
    \midrule
    \midrule
    $T~/~\si{\kelvin}$ & $C_V~/~\si{\joule\per\mol}$ & $\theta~/~T$ &
    $\theta~/~\si{\kelvin}$ \\
    \midrule
    \phantom{0}88.84  & 15.9              & 3.1               & 275.40           \\
\phantom{0}98.08  & 16.1              & 3.1               & 304.04           \\
107.12            & 19.0              & 2.4               & 257.09           \\
115.96            & 20.0              & 2.2               & 255.11           \\
125.56            & 17.2              & 2.8               & 351.56           \\
131.09            & 20.4              & 2.0               & 262.19           \\
142.94            & 19.2              & 2.4               & 343.06           \\
150.95            & 20.9              & 1.9               & 286.81           \\
160.95            & 20.7              & 2.0               & 321.90           \\
    \midrule
    \midrule
  \end{tabular}
  \caption{Debye-Temperaturen.}
  \label{tab:debye}
\end{table}

\begin{figure}[htpb]
  \centering
  \includegraphics[scale=0.3]{bilder/debye.png}
  \caption{Zahlenwerte für die Debye-Funktion~\cite{FP}.}
\label{fig:debye}
\end{figure}
