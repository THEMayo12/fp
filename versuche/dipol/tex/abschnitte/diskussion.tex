
% ==================================================
%	Diskussion
% ==================================================

\section{Diskussion}
In Tabelle \ref{tab:Diskussion} sind die Ergebnisse dieser Versuchsdurchführung
zusammengefasst.

\begin{table}[h]
\centering
\begin{tabular}{lr}
\toprule \midrule
Messgröße & Messergebnis\\
\midrule

$W_1$ &$ (2.25 \pm 0.37) \text{ eV}$ \\

$W_2 $& $(2.35\pm 0.26) \text{ eV}$ \\

$W_{S,1}$ &$ (0.76\pm 0.13) \text{ eV}$ \\

$W_{S,2}$ &$ (0.48\pm 0.15) \text{ eV}$ \\

$\tau_{0,1}$ & $(1.8 \pm 9.6) \times 10^{-12}\text{ s}$ \\

$\tau_{0,2} $& $(1.54 \pm 8.5) \times 10^{-7}\text{ s} $\\
\midrule \bottomrule
\end{tabular}
\caption{Zusammenstellung der Messergebnisse.}
\label{tab:Diskussion}
\end{table}

Zunächst ist festzustellen, dass die Werte $W_1,W_2$ und $W_{S,1},W_{S,2}$ nur gering voneinander 
Abweichen, und alle in einer realistischen Größenordnung liegen. Typische Aktivierungsenergien 
leigen im Bereich $10^{-1}-10^1$ eV. Die Werte $W_1$ und $W_2$ resultieren aus einer 
Temperatur-Strom Näherung, welche nur für kleine Temperaturen gültig ist (vgl. Theorie), sodass 
hier, da weniger Messwerte verwendet werden, größere Fehler erwartet werden als bei der 
zweiten Berechnungsmethode. Bei der zweiten Methode, der Berechnung aus dem gesamten 
Kurvenverlauf wird jedoch vorausgesetzt, dass die Heizrate während der Gesamten Messung konstant 
ist. Dies ist jedoch, wie an der Abbildung \ref{fig:Heiz} zu sehen ist, nicht exakt erfüllt. 
Die Heizraten lagen bei $b_{1,\text{real}}=(1.63\pm 0.31)$ K/min und $b_{2,\text{real}}=2.41\pm 
0.30$ K/min, sodass hier entsprechende, allerdings relativ kleine Fehler erwartet werden. Die 
ermittelten Werte für die Relaxationszeiten zeigen sehr große Fehler und weichen außerdem 
um einen Faktor $10^5$ voneinander ab, sodass damit keine zuverlässige Aussage getroffen 
werde kann. Bei der Ermittlung der Relaxationszeiten hat sich gezeigt, dass sie sehr 
empfindlich von den Parametern abhängen, sodass eine kleine Änderung der Aktivierungsenergie 
bereits eine große Veränderung der berechneten Relaxationszeit hervorruft. Um eine bessere 
Aussage über diese Größe zu bekommen, müssten zuerst die Fehler der $W$-Messung verringert 
werden.