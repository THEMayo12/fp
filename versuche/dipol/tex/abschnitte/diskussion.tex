
% ==================================================
%	Diskussion
% ==================================================

\section{Diskussion}
In Tabelle \ref{tab:Diskussion} sind die Ergebnisse dieser Versuchsdurchführung 
zusammengefasst.

\begin{table}[H]
\centering
\begin{tabular}{lr}
\midrule \midrule
Messgröße & Messergebnis\\
\midrule

$W_1$ &$ (0.1273\pm 0.0012) \text{ eV}$ \\ 

$W_2 $& $(0.0786\pm 0.0087) \text{ eV}$ \\ 

$W_{S,1}$ &$ (0.656\pm 0.011) \text{ eV}$ \\ 

$W_{S,2}$ &$ (0.705\pm 0.016) \text{ eV}$ \\ 
 
$\tau_{0,1}$ & $(5.5 \pm 2.1) \times 10^{-9}\text{ s}$ \\ 

$\tau_{0,2} $& $(1.50 \pm 0.73) \times 10^{-9}\text{ s} $\\ 
\midrule \midrule
\end{tabular}
\caption{Zusammenstellung der Messergebnisse.}
\label{tab:Diskussion}
\end{table}

Zunächst sind die großen Abweichungen zwischen den Werten $W$ und $W_S$ zu bewerten. 
Nach der Theorie sollten die letzteren den wahren Wert besser annähern. Dabei 
wird jedoch vorausgesetzt, dass am Ende der Messung der Relaxationsstrom gegen Null 
geht, was in dieser Versuchsdurchführung jedoch nicht der Fall war. 