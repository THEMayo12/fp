\section{Auswertung}
Die gemessenen Wertepaare von Temperatur und Depolarisationsstrom sind in den 
Tabellen \ref{tab:1} und \ref{tab:2} zu sehen. Im Folgenden beziehen sich Größen mit
Index "`1"' auf die Heizrate $b_1=2$ K/min bzw. mit Index "`2"' auf $b_2=2.5$ 
K/min.

%
%280.1             & 0.45              & 291.1             & 1.66              & 303.5             & 1.23             \\
281.8             & 0.48              & 293.1             & 1.61              & 304.9             & 1.27             \\
283.4             & 0.65              & 294.8             & 1.56              & 306.4             & 1.30             \\
285.2             & 0.91              & 296.6             & 1.40              & 307.8             & 1.33             \\
287.1             & 1.15              & 298.5             & 1.23              & 309.3             & 1.45             \\
289.1             & 1.52              & 301.8             & 1.21              & -                 & -                \\
%233.1             & 0.39              & 278.3             & 0.00              & 323.9             & 0.00             \\
235.5             & 0.30              & 281.0             & 0.00              & 326.3             & 0.00             \\
238.1             & 0.25              & 283.1             & 0.00              & 328.8             & 0.00             \\
240.7             & 0.21              & 285.1             & 0.00              & 331.4             & 0.23             \\
243.3             & 0.16              & 287.3             & 0.00              & 333.6             & 0.56             \\
245.9             & 0.11              & 289.9             & 0.07              & 335.9             & 1.07             \\
248.5             & 0.07              & 292.4             & 0.38              & 338.4             & 1.63             \\
251.0             & 0.04              & 294.9             & 0.59              & 341.0             & 2.05             \\
253.5             & 0.00              & 300.4             & 0.20              & 343.5             & 2.55             \\
255.8             & 0.00              & 299.9             & 0.00              & 346.0             & 3.13             \\
258.1             & 0.00              & 302.3             & 0.00              & 348.5             & 3.47             \\
260.3             & 0.00              & 304.6             & 0.00              & 350.9             & 3.59             \\
262.3             & 0.00              & 306.8             & 0.00              & 353.1             & 3.63             \\
265.1             & 0.00              & 309.0             & 0.00              & 355.4             & 3.11             \\
267.0             & 0.00              & 311.3             & 0.00              & 357.8             & 2.55             \\
269.1             & 0.00              & 313.5             & 0.00              & 360.3             & 1.38             \\
271.1             & 0.00              & 316.2             & 0.00              & 363.1             & 0.09             \\
273.4             & 0.00              & 318.6             & 0.00              & 365.6             & 0.00             \\
276.1             & 0.00              & 321.3             & 0.00              & 368.6             & 0.00             \\
%
\subsection{Bestimmung von $W$ aus dem ersten Teil des Kurvenverlaufes}
Für Temperaturen nahe bei $T_0:=\min\{T_1\}$ gilt die Abhängigkeit
\begin{equation}
j(T)\propto \text{e}^{-W/\text{k}_\text{B}T} \quad ,
\end{equation}
wobei $j$ der Depolarisationsstrom, $W$ die Aktivierungsenergie, $T$ die 
Temperatur sind, sowie $\text{e}$ und $\text{k}_\text{B}$ die Eulersche- und 
Boltzmannkonstanten.

Ein linearer Fit für $\ln(j)$ gegen $1/T$ ergibt
%\input{G1}
%\input{G2}
Die Graphen dazu sind in den Abbildungen \ref{fig:G1} und \ref{fig:G2} zu sehen.

Damit ergibt sich
\begin{align}
W_1&= \\
W_2&=   \quad .
\end{align}
Da bei der Rechnung nur eine fehlerbehaftete Größe (die Steigung der 
Ausgleichsgeraden) vorkommt, sind die Fehlerrechnung identisch mit der 
zum Mittelwert.
\subsection{Bestimmung von $W$ aus dem gesamten Kurvenverlauf}
Definiere 
\begin{equation}
S(T)=\int\limits_T^{T^*} j(T') \text{d}T' \frac{1}{j(T)}
\end{equation}
mit dem letzten gemessenen Wert $T^*$.
Bei konstanter Heizrate gilt
\begin{equation}
\frac{W}{\text{k}_\text{B} T}=\ln S(T)
\end{equation}
mit einer Konstanten $k$. Bei der Berechnung wurde die Näherung 
\begin{equation}
S(T)=\int\limits_T^{T^*} j(T') \text{d}T' \approx \sum\limits_{\#\text{Messwerte}} j^n  \hat{b}
\end{equation}
benutzt. Der Index bezieht sich dabei auf den $n$-ten Messwert, $\hat{b}$ ist die 
Heizrate in SI-Einheiten. Ein linearer Fit für $\ln(S(T))$ gegen $1/T$ ergibt 
%\input{GS1}
%\input{GS2}
Die entsprechenden Abbildungen sind \ref{fig:GS1} und \ref{fig:GS2}.
Es folgen
\begin{align}
W_{S,1}=  \\
W_{S,2}=   \quad .
\end{align}
Da wieder nur eine fehlerbehaftete Größe vorkommt, kann der Fehler wie der 
Messwert behandelt werden.

\subsection{Bestimmung der Relaxationszeit $\tau$}
Differentiation von Gleichung \ref{} ergibt
\begin{equation}
1/b + \frac{\text{d}}{\text{d}T}\eval_\text{max}\tau(T)=0
\end{equation}
mit $T_\text{max}=\max\{T \}$. Mit
\begin{equation}
 \frac{\text{d}}{\text{d}T}\eval_\text{max}\tau(T) =-\frac{W}{\text{k}_\text{B} 
 T^2}
\end{equation}
folgt
\begin{equation}
\tau(T_\text{max})=\frac{\text{k}_\text{B} T_\text{max}^2}{W b} \quad .
\end{equation}
Nun kann mit Hilfe von Gleichung \eqref{eq:tau} die Relaxationszeit $\tau_0$ 
gemäß
\begin{equation}
\tau_0=\tau(T_\text{max})\text{e}^{-\frac{W}{\text{k}_\text{B}T_\text{max}}}
\end{equation}
bestimmt werden.

So ergibt sich
\begin{align}
\tau_{0,1}=
\tau_{0,2}=
\end{align}

Die Fehlerrechnung geschieht dabei gemäß der Gaußschen Fehlerfortpflanzung.

