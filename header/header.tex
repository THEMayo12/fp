%        File: header.tex
%     Created: Sa Mär 08 11:00  2014 C
% Last Change: Sa Mär 08 11:00  2014 C
%

% ==================================================
%	Encoding
% ==================================================

\usepackage[utf8]{inputenc}
\usepackage[T1]{fontenc}
\usepackage{lmodern}
\usepackage{textcomp}

% ==================================================
%	Spracheinstellung
% ==================================================

\usepackage[ngerman]{babel}

% ==================================================
%	Referenzen
% ==================================================

\usepackage[ngerman]{varioref}
% Links im pdf
\usepackage[pdfborder={0 0 0}, hypertexnames=false]{hyperref}
% zusammen mit varioref, hyperref folgt damit
% z. B. in \vref{eq:1} -> in Gl. 1 auf Seite 4, cleveres referenzieren
\usepackage{cleveref}

% ==================================================
%	Grafiken, Abbildungen und Tabellen
% ==================================================

\usepackage{graphicx}
\usepackage{xcolor}
\usepackage[font=small, labelfont=bf, format=plain]{caption}
\usepackage{subcaption}
\usepackage{booktabs}
% for floating figures and floating tables
\usepackage[vflt]{floatflt}

\usepackage{rotating}

\usepackage{multicol}
\usepackage{multirow}

\usepackage{tikz}
\usetikzlibrary{shadows, calc, arrows, patterns}

\usepackage{tcolorbox}
\tcbuselibrary{skins, breakable, theorems}

% ==================================================
%	Seiten-Layout und -Definitionen
% ==================================================

% Maße für DIN A4
\usepackage[a4paper]{geometry}

\clubpenalty10000
\widowpenalty10000
\displaywidowpenalty=10000

% ==================================================
%	Float-Parameter
% ==================================================

% minimaler Anteil der Seite für den Text
\renewcommand{\textfraction}{0.05}
% maximaler Anteil der Seite für Floats am Anfang
\renewcommand{\topfraction}{0.95}
% maximaler Anteil der Seite für Floats am Enden
\renewcommand{\bottomfraction}{0.95}
% mininaler Anteil der Float-Seite für Text
\renewcommand{\floatpagefraction}{0.35}
% maximale Anzahl der Floats auf der Seite
\setcounter{totalnumber}{5}

% ==================================================
%	Fancy Header
% ==================================================

\usepackage{fancyhdr}
\pagestyle{fancy}

% Stil für gesamtes Dokument
\fancyhf{}

% Dicke der Linien ändern
\renewcommand{\headrulewidth}{1.0pt}
\renewcommand{\footrulewidth}{1.0pt}

% Abschnitte in Großschrift
\renewcommand{\sectionmark}[1]{\markright{\MakeUppercase{#1}}{}}
\renewcommand{\subsectionmark}[1]{}

% Die Positionen in Kopf- und Fußzeile füllen
\fancyhead[L]{
	TU Dortmund
}
\fancyhead[R]{\tit}
\fancyfoot[L]{\rightmark}
\fancyfoot[R]{Seite \thepage}

\addtolength{\headheight}{2\baselineskip}

% Neudefinition von plain (wird auf Kapitel/Verzeichnis-Seiten verwendet)
\fancypagestyle{plain}{
	\fancyhead[L]{}
	\fancyhead[R]{}
	\fancyfoot[L]{\rightmark}
	\fancyfoot[R]{Seite \thepage}
}

\fancypagestyle{firstpage}{
	\fancyhf{}
	\fancyhead[L]{}
	\renewcommand{\footrulewidth}{0.0pt}
}

% ==================================================
%	Bibliograhphie
% ==================================================

\newcommand{\anhang}{
	\clearpage
	\setcounter{page}{0}
	\pagenumbering{Roman}
	% Kapitelnummerierung in Großbuchstaben statt Zahlen
	\appendix
}

\newcommand{\referenzen}{
	% \renewcommand{\refname}{Quellenverzeichnis}
	\bibliographystyle{utphys}
	\bibliography{literatur}
}

% ========================================
%	Angaben für das Titelblatt
% ========================================

% einfaches Verändern des Titels/Versuchs in der Hauptdatei
\newcommand{\titel}[1]{\newcommand{\tit}{#1}}
\newcommand{\versuch}[1]{\newcommand{\ver}{Versuch #1}}

% ==================================================
%	Mathematik
% ==================================================

\usepackage{amsmath}
\usepackage{amsfonts}
\usepackage{amssymb}
\usepackage{amstext}
\usepackage{mathtools}
% für Einheitsmatrix '1'
\usepackage{bbm}

% Mathematik Font
\usepackage[sc]{mathpazo}
% Palatino needs more leading (space between lines)
\linespread{1.05}

% ==================================================
%	Physik
% ==================================================

\usepackage{upgreek}
\usepackage{physics}
\usepackage[nice]{units}
\usepackage[parse-numbers=false, per-mode=symbol]{siunitx}

% ==================================================
%	Sonstiges
% ==================================================

% rechnen mit LateX-Counter
\usepackage{calc}
% Euro-Zeichen
\usepackage{eurosym}
% ein paar LateX-Fehler beheben
\usepackage{fixltx2e}
% für bessere Darstellung von Text (Abstände)
\usepackage{microtype}
% Einfügen von Text um Layout zu testen
\usepackage{lipsum}
